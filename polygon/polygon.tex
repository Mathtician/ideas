\documentclass{article}
\usepackage{amsmath}
\usepackage{amssymb}
\usepackage{graphicx}
\usepackage[margin=1in]{geometry}
\usepackage{hyperref}
\usepackage{caption}
\usepackage{float}
\graphicspath{{images/}}
\hypersetup{
  colorlinks=true,
  urlcolor=blue,
}
\begin{document}

\title{Rendering Pixel-Perfect Polygon Tilings}
\author{Aresh Pourkavoos}
\maketitle

Given a tiling of the plane by non-self-intersecting polygons,
now do you assign each point to a unique polygon?
The interiors don't overlap, so they can be assigned unambiguously,
but what about the edges and vertices?
It's possible to assign some arbitrary ordering to the polygons
and simply choose the ``first'' one that contains a given point.
However, it would be nice if the assignment
depended only on the shapes themselves
and not on any additional information.
\begin{itemize}
\item
  Non-vertical edges are assigned to the polygon directly below them.
\item
  Vertical edges are assigned to the polygon directly to their right.
\item
  Vertices with no downward-pointing edge are assigned to the polygon directly below them.
\item
  Vertices with a downward-pointing edge are assigned to the polygon directly to the right of said edge.
\end{itemize}
This assignment is invariant under translation and scaling, but not rotation.
\begin{center}
  \begin{tabular}{|c c c c c|}
    \hline
    4 & 3 & 2 & 2 & 1 \\
    4 & 4 & 2 & 1 & 1 \\
    5 & 5 & 7 & 8 & 8 \\
    5 & 6 & 7 & 7 & 8 \\
    6 & 6 & 7 & 7 & 7 \\ \hline
  \end{tabular}
\end{center}

\end{document}
