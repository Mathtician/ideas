\documentclass{article}
\usepackage[margin=1in]{geometry}
\usepackage{amsmath}
\usepackage{amssymb}
\usepackage{graphicx}
\usepackage{mathtools}
\usepackage{tikz}
\usepackage{enumitem}
\usepackage{amsthm}
\graphicspath{images/}

\newcommand{\R}{\mathbb{R}}
\newcommand{\abs}[1]{\left\vert #1 \right\vert}
\DeclareMathOperator{\supp}{supp}
\DeclareMathOperator{\spn}{span}

\begin{document}

\title{Optimal Bump Functions with Lipschitz $n^\text{th}$ Derivatives}
\author{Aresh Pourkavoos}
\maketitle

\newtheorem{theorem}{Theorem}[section]
\newtheorem{lemma}{Lemma}[section]
\newtheorem{corollary}{Corollary}[section]
\newcommand{\norm}[1]{\left\Vert #1 \right\Vert}

%% I originally encountered this problem on my high school robotics team.
%% We wanted to program a routine for our robot
%% to move from one point to another in a straight line in the minimum time $T$.
%% For simplicity, assume the robot
%% is moving in one dimension by a distance $d_0 > 0$.
%% The robot has a top speed $d_1 > 0$,
%% so the most straightforward routine is
%% to calculate the time it takes to travel the desired distance at top speed
%% (i.e. $T=\frac{d_1}{d_0}$)
%% and to run the motors for that length of time.
%% The velocity and position graphs for this routine are below.
%% \begin{center}
%%   \begin{tikzpicture}[scale=0.5]
%%     \begin{scope}[shift={(0,0)}]
%%       \draw[->] (-1.5, 0) -- (4.5, 0) node[right] {$t$};
%%       \draw[->] (0, -0.5) -- (0, 6.5) node[above] {$v$};
%%       \draw (0.25, 2) -- (-0.25, 2) node[left] {$d_1$};
%%       \draw (3, 0.25) -- (3, -0.25) node[below] {$T$};
%%       \draw[very thick, color=red] plot coordinates {(-1, 0) (0, 0) (0, 2) (3, 2) (3, 0) (4, 0)};
%%     \end{scope}
%%     \begin{scope}[shift={(8,0)}]
%%       \draw[->] (-1.5, 0) -- (4.5, 0) node[right] {$t$};
%%       \draw[->] (0, -0.5) -- (0, 6.5) node[above] {$x$};
%%       \draw (0.25, 6) -- (-0.25, 6) node[left] {$d_0$};
%%       \draw (3, 0.25) -- (3, -0.25) node[below] {$T$};
%%       \draw[very thick, color=red] plot coordinates {(-1, 0) (0, 0) (3, 6) (4, 6)};
%%     \end{scope}
%%   \end{tikzpicture}
%% \end{center}
%% However, this solution was unsatisfactory
%% because of the sudden acceleration from rest to top speed,
%% which would cause the wheels to slip
%% and the distance traveled to be inconsistent.
%% So we placed a limit $d_2 > 0$ on the acceleration,
%% had the robot accelerate at that rate to top speed,
%% maintain that speed for as long as possible,
%% then slow down to rest as it approached the destination.
%% The new graphs of acceleration, velocity, and position were:
%% \begin{center}
%%   \begin{tikzpicture}[scale=0.5]
%%     \begin{scope}[shift={(0,0)}]
%%       \draw[->] (-1.5, 0) -- (5.5, 0) node[right] {$t$};
%%       \draw[->] (0, -2.5) -- (0, 6.5) node[above] {$a$};
%%       \draw (0.25, 2) -- (-0.25, 2) node[left] {$d_2$};
%%       \draw (4, -0.25) -- (4, 0.25) node[above] {$T$};
%%       \draw[very thick, color=red] plot coordinates {(-1, 0) (0, 0) (0, 2) (1, 2) (1, 0) (3, 0) (3, -2) (4, -2) (4, 0) (5, 0)};
%%     \end{scope}
%%     \begin{scope}[shift={(9,0)}]
%%       \draw[->] (-1.5, 0) -- (5.5, 0) node[right] {$t$};
%%       \draw[->] (0, -2.5) -- (0, 6.5) node[above] {$v$};
%%       \draw (0.25, 2) -- (-0.25, 2) node[left] {$d_1$};
%%       \draw (4, 0.25) -- (4, -0.25) node[below] {$T$};
%%       \draw[very thick, color=red] plot coordinates {(-1, 0) (0, 0) (1, 2) (3, 2) (4, 0) (5, 0)};
%%     \end{scope}
%%     \begin{scope}[shift={(18,0)}]
%%       \draw[->] (-1.5, 0) -- (5.5, 0) node[right] {$t$};
%%       \draw[->] (0, -2.5) -- (0, 6.5) node[above] {$x$};
%%       \draw (0.25, 6) -- (-0.25, 6) node[left] {$d_0$};
%%       \draw (4, 0.25) -- (4, -0.25) node[below] {$T$};
%%       \draw[very thick, color=red] plot coordinates {(-1, 0) (0, 0)};
%%       \draw[very thick, color=red, domain=0:1] plot (\x, \x*\x);
%%       \draw[very thick, color=red] plot coordinates {(1, 1) (3, 5)};
%%       \draw[very thick, color=red, domain=3:4] plot (\x, -10+8*\x-\x*\x);
%%       \draw[very thick, color=red] plot coordinates {(4, 6) (5, 6)};
%%     \end{scope}
%%   \end{tikzpicture}
%% \end{center}
%% However, over short enough distances,
%% the robot would not have enough time to reach full speed,
%% so it would need to begin decelerating before then:
%% \begin{center}
%%   \begin{tikzpicture}[scale=0.5]
%%     \begin{scope}[shift={(0,0)}]
%%       \draw[->] (-1.5, 0) -- (2.5, 0) node[right] {$t$};
%%       \draw[->] (0, -2.5) -- (0, 2.5) node[above] {$a$};
%%       \draw (0.25, 2) -- (-0.25, 2) node[left] {$d_2$};
%%       \draw (1, -0.25) -- (1, 0.25) node[above] {$T$};
%%       \draw[very thick, color=red] plot coordinates {(-1, 0) (0, 0) (0, 2) (0.5, 2) (0.5, -2) (1, -2) (1, 0) (2, 0)};
%%     \end{scope}
%%     \begin{scope}[shift={(6,0)}]
%%       \draw[->] (-1.5, 0) -- (2.5, 0) node[right] {$t$};
%%       \draw[->] (0, -2.5) -- (0, 2.5) node[above] {$v$};
%%       \draw (0.25, 2) -- (-0.25, 2) node[left] {$d_1$};
%%       \draw (1, 0.25) -- (1, -0.25) node[below] {$T$};
%%       \draw[very thick, color=red] plot coordinates {(-1, 0) (0, 0) (0.5, 1) (1, 0) (2, 0)};
%%     \end{scope}
%%     \begin{scope}[shift={(12,0)}]
%%       \draw[->] (-1.5, 0) -- (2.5, 0) node[right] {$t$};
%%       \draw[->] (0, -2.5) -- (0, 2.5) node[above] {$x$};
%%       \draw (0.25, 0.5) -- (-0.25, 0.5) node[left] {$d_0$};
%%       \draw (1, 0.25) -- (1, -0.25) node[below] {$T$};
%%       \draw[very thick, color=red] plot coordinates {(-1, 0) (0, 0)};
%%       \draw[very thick, color=red, domain=0:0.5] plot (\x, \x*\x);
%%       \draw[very thick, color=red, domain=0.5:1] plot (\x, -0.5+2*\x-\x*\x);
%%       \draw[very thick, color=red] plot coordinates {(1, 0.5) (2, 0.5)};
%%     \end{scope}
%%   \end{tikzpicture}
%% \end{center}

%% $T$ still represents the total time to excecute the maneuver,
%% but its formula has changed.
%% The time to accelerate to top speed is given by $\frac{d_1}{d_2}$,
%% and the distance the robot travels in that time is
%% $\frac{d_2}{2}\left(\frac{d_1}{d_2}\right)^2=\frac{d_1^2}{2d_2}$.
%% If the total distance traveled is at least twice that,
%% i.e. if $d_0 \geq \frac{d_1^2}{d_2}$,
%% then the robot can reach top speed on the way.
%% Thus it spends $\frac{2d_1}{d_2}$ time accelerating and decelerating (altogether),
%% and the distance it covers at top speed is $d_0-\frac{d_1^2}{d_2}$, which takes
%% $\frac{1}{d_1}\left(d_0-\frac{d_1^2}{d_2}\right)=\frac{d_0}{d_1}-\frac{d_1}{d_2}$ time.
%% Then $T=\frac{2d_1}{d_2}+\frac{d_0}{d_1}-\frac{d_1}{d_2}=\frac{d_0}{d_1}+\frac{d_1}{d_2}$.
%% On the other hand, if $d_0 < \frac{d_1^2}{d_2}$,
%% the robot can only accelerate until it is halfway to the destination,
%% which also takes half of the total time
%% (as the other half is spent slowing back down).
%% Thus $\frac{d_0}{2}=\frac{d_2}{2}\left(\frac{T}{2}\right)^2$,
%% so $T=2\sqrt{\frac{d_0}{d_2}}$.

%% This solution worked well enough for our robot,
%% but some applications also place a limit $d_3 > 0$ on jerk,
%% which is the third derivative of position with respect to time,
%% i.e. the rate of change of acceleration.
%% For example, people in elevators feel acceleration as weight,
%% so if the elevator jumps from rest to some nonzero acceleration,
%% the passengers will feel a sudden weight change.
%% As a result, the path of an elevator
%% may have up to 7 polynomial pieces
%% (not including the phases of rest at the beginning and end).
%% Example jerk, acceleration, velocity, and position curves
%% are shown below.

%% \begin{center}
%%   \begin{tikzpicture}[xscale=0.5,yscale=0.0625]
%%     \begin{scope}[shift={(0,0)}]
%%       \draw[->] (-1.5, 0) -- (8.5, 0) node[right] {$t$};
%%       \draw[->] (0, -52) -- (0, 52) node[above] {$j$};
%%       \draw (0.25, 48) -- (-0.25, 48) node[left] {$d_3$};
%%       \draw (7, 2) -- (7, -2) node[below] {$T$};
%%       \draw[very thick, color=red] plot coordinates {(-1, 0) (0, 0) (0, 48) (1, 48) (1, 0) (2, 0) (2, -48) (3, -48) (3, 0) (4, 0) (4, -48) (5, -48) (5, 0) (6, 0) (6, 48) (7, 48) (7, 0) (8, 0)};
%%     \end{scope}
%%     \begin{scope}[shift={(12,0)}]
%%       \draw[->] (-1.5, 0) -- (8.5, 0) node[right] {$t$};
%%       \draw[->] (0, -52) -- (0, 52) node[above] {$a$};
%%       \draw (0.25, 24) -- (-0.25, 24) node[left] {$d_2$};
%%       \draw (7, -2) -- (7, 2) node[above] {$T$};
%%       \draw[very thick, color=red] plot coordinates {(-1, 0) (0, 0) (1, 24) (2, 24) (3, 0) (4, 0) (5, -24) (6, -24) (7, 0) (8, 0)};
%%     \end{scope}
%%   \end{tikzpicture} \\
%%   \vspace{5mm}
%%   \begin{tikzpicture}[xscale=0.5,yscale=0.0625]
%%     \begin{scope}[shift={(0,0)}]
%%       \draw[->] (-1.5, 0) -- (8.5, 0) node[right] {$t$};
%%       \draw[->] (0, -4) -- (0, 52) node[above] {$v$};
%%       \draw (0.25, 24) -- (-0.25, 24) node[left] {$d_1$};
%%       \draw (7, 2) -- (7, -2) node[below] {$T$};
%%       \draw[very thick, color=red] plot coordinates {(-1, 0) (0, 0)};
%%       \draw[very thick, color=red, domain=0:1] plot (\x, 6*\x*\x);
%%       \draw[very thick, color=red] plot coordinates {(1, 6) (2, 18)};
%%       \draw[very thick, color=red, domain=2:3] plot (\x, -6*\x*\x+36*\x-30);
%%       \draw[very thick, color=red] plot coordinates {(3, 24) (4, 24)};
%%       \draw[very thick, color=red, domain=4:5] plot (\x, -6*\x*\x+48*\x-72);
%%       \draw[very thick, color=red] plot coordinates {(5, 18) (6, 6)};
%%       \draw[very thick, color=red, domain=6:7] plot (\x, 6*\x*\x-84*\x+294);
%%       \draw[very thick, color=red] plot coordinates {(7, 0) (8, 0)};
%%     \end{scope}
%%     \begin{scope}[shift={(12,0)}]
%%       \draw[->] (-1.5, 0) -- (8.5, 0) node[right] {$t$};
%%       \draw[->] (0, -4) -- (0, 52) node[above] {$x$};
%%       \draw (0.25, 48) -- (-0.25, 48) node[left] {$d_0$};
%%       \draw (7, 2) -- (7, -2) node[below] {$T$};
%%       \draw[very thick, color=red] plot coordinates {(-1, 0) (0, 0)};
%%       \draw[very thick, color=red, domain=0:1] plot (\x, \x*\x*\x);
%%       \draw[very thick, color=red, domain=1:2] plot (\x, 3*\x*\x-3*\x+1);
%%       \draw[very thick, color=red, domain=2:3] plot (\x, -\x*\x*\x+9*\x*\x-15*\x+9);
%%       \draw[very thick, color=red] plot coordinates {(3, 18) (4, 30)};
%%       \draw[very thick, color=red, domain=4:5] plot (\x, -\x*\x*\x+12*\x*\x-36*\x+46);
%%       \draw[very thick, color=red, domain=5:6] plot (\x, -3*\x*\x+39*\x-79);
%%       \draw[very thick, color=red, domain=6:7] plot (\x, \x*\x*\x-21*\x*\x+147*\x-295);
%%       \draw[very thick, color=red] plot coordinates {(7, 48) (8, 48)};
%%     \end{scope}
%%   \end{tikzpicture}
%% \end{center}

%% At this point, the general version of the problem begins to suggest itself.
%% Given an integer $n \geq 1$ and positive real numbers $d_0, d_1, \ldots, d_n$,
%% the task is to find a function $x : \R \rightarrow \R$ where

%% \begin{enumerate}
%% \item
%%   $x(t)=0$ for all $t \leq 0$;
%% \item
%%   $x(t)=d_0$ for all $t \geq T$, where $T > 0$;
%% \item
%%   $\abs{x^{(i)}(t)} \leq d_i$
%%   for all $t \in \R$ and $i \in \{1, \ldots, n\}$
%%   where $x^{(i)}(t)$ denotes the $i^\text{th}$ derivative of $x$ at $t$;
%% \item
%%   the value of $T$ is minimized among all functions
%%   which satisfy properties 1-3.
%% \end{enumerate}

%% Actually, the problem statement above is slightly inaccurate,
%% specifically property 3,
%% as the $n^\text{th}$ derivative is not defined everywhere.
%% In the figure above, there are $8$ transition points between the phases.
%% At these points, the acceleration $a = x^{(n-1)}$ has no derivative,
%% so the jerk is undefined.
%% Instead of saying that the derivative of $x^{(n-1)}$
%% (i.e. $j = x^{(n)}$) must be bounded by $d_n$,
%% we say that $x^{(n-1)}$ must be \textit{Lipschitz continuous}
%% with Lipschitz constant $d_n$.
%% This means that
%% \[
%% \abs{\frac{x^{(n-1)}(t_2)-x^{(n-1)}(t_1)}{t_2-t_1}} \leq d_n
%% \]
%% for all $t_1, t_2 \in \R$.
%% Geometrically speaking, this is equivalent to
%% every \textit{secant} line of the graph of $x^{(n-1)}$
%% having a slope between $-d_n$ and $d_n$
%% (rather than every \textit{tangent} line as bounding $x^{(n)}$ would suggest).
%% So property 3 may be more accurately written as
%% \begin{enumerate}
%%   \setcounter{enumi}{2}
%% \item
%%   $\abs{x^{(i)}(t)} \leq d_i$
%%   for all $t \in \R$ and $i \in \{1, \ldots, n-1\}$,
%%   and $x^{(n-1)}$ has Lipschitz constant $d_n$.
%% \end{enumerate}

%% I have not yet been able to solve this problem in full,
%% but I have done three things:
%% \begin{enumerate}
%%   \item
%%     Find an algorithm to produce \textit{a} function
%%     which satisfies properties 1-3 for any values of $d_0, \ldots, d_n$.
%%     However, this is not \textit{the} function which minimizes $T$.
%%   \item
%%     Find the (probably) optimal solution
%%     in the case where $x^{(n-1)}$ has Lipschitz constant $d_n$,
%%     but $\abs{x^{(i)}}$ is not restricted for $i \in \{1, \ldots, n-1\}$.
%%     Equivalently, this is the case where $d_1, \ldots, d_{n-1}$ are very large,
%%     so $d_n$ is the only ``bottleneck.''
%%   \item
%%     Combine the previous two cases in a nontrivial way
%%     to take advantage of both of their strengths.
%%     This approach
%%     \begin{itemize}
%%     \item
%%       returns a solution for any values of $d_0, \ldots, d_n$
%%       (which is at least as good as that of case 1);
%%     \item
%%       returns the same thing as case 2 whenever the latter is applicable;
%%     \item
%%       and (this is the nontrivial part)
%%       sometimes returns a solution \textit{better} than that of case 1,
%%       even when case 2 is \textit{not} directly applicable.
%%     \end{itemize}
%% \end{enumerate}
%% The algorithm in case 1 arose from the following observation:
%% the velocity graph for $n=2$
%% looks like two reflected copies of the position graph for $n=1$,
%% separated by a constant function at $d_1$.
%% Likewise, the velocity graph for $n=3$
%% looks like two reflected copies of the position graph for $n=2$,
%% separated by a constant function at $d_1$.  
%% In general, this makes sense:
%% in order to get from $0$ to $d_0$ with $n$ derivatives restricted,
%% the velocity needs to increase from $0$ to $d_1$ (or less),
%% stay at the maximum for some length of time (possibly $0$ time),
%% and then decrease back to $0$.
%% But the increasing portion of the velocity
%% may be seen as a smaller instance of the same problem,
%% where the first $n-1$ derivatives are restricted rather than $n$,
%% and the decreasing portion is a mirror image of that.
%% The base case $n=1$, as seen above, is just a straight line segment.

%% \newpage

%% \
%% Theorem: Let $n \in \mathbb{Z}^+$, and let $x : \mathbb{R} \rightarrow \mathbb{R}$.
%% Suppose that $x(t)$ is eventually constant as $t \rightarrow \infty$ and as $t \rightarrow -\infty$, so any derivatives of $x$ have compact support.
%% Suppose also that $x$ is $n$ times weakly differentiable,
%% where the weak $n^\text{th}$ derivative is bounded.
%% (Equivalently, suppose $x$ is $n-1$ times differentiable,
%% where the $(n-1)^\text{th}$ derivative is Lipschitz.)
%% Then,

%% \[x(t)\Big|_{-\infty}^\infty \leq \frac{1}{4^nn!}\norm{x^{(n)}}_\infty\lambda(\supp(x'))^n.\]

%% Additionally, the constant $\frac{1}{4^nn!}$ is tight.

%% Proof: By scaling and translation of $x$, it suffices to show that
%% if $x(t) = 0$ for all $t \leq 0$, $x(t)$ is constant for all $t \geq 4$
%% (so $\lambda(\supp(x')) \leq 4$)

%% $\omega := \exp\left(\frac{\pi}{n}\sqrt{-1}\right)$ \\
%% $t_i := 4\cos^2\left(\frac{\pi (n-i)}{2n}\right)$ \\
%% $t_{n-i} = 4\cos^2\left(\frac{\pi i}{2n}\right)
%% = \left(2\cos\left(\frac{\pi i}{2n}\right)\right)^2
%% = \left(\omega^{i/2}+\omega^{-i/2}\right)^2$ \\
%% $z_{k,i}(t) := \frac{\max(0, t-t_i)^k}{k!}$ \\
%% $z_{k,i}^{(k-j)} = z_{j,i}$ \\
%% \begin{align*}
%%   x :=& \sum_{i=0}^{n-1}(-1)^i(z_{n,i} - z_{n,i+1}) \\
%%   =& \sum_{i=0}^{n-1}(-1)^iz_{n,i} + \sum_{i=1}^n(-1)^iz_{n,i} \\
%%   =& \sum_{i=0}^{n-1}(-1)^iz_{n,i} + \sum_{i=0}^{n-1}(-1)^{n-i}z_{n,n-i} \\
%%   =& \sum_{i=0}^{n-1} \left((-1)^iz_{n,i} + (-1)^{n-i}z_{n,n-i}\right) \\
%%   x^{(n-k)} =& \sum_{i=0}^{n-1} \left((-1)^iz_{k,i} + (-1)^{n-i}z_{k,n-i}\right) \\
%%   x^{(n-k)}(t_n) =& \sum_{i=0}^{n-1} \left((-1)^iz_{k,i}(t_n) + (-1)^{n-i}z_{k,n-i}(t_n)\right) \\
%%   =& \sum_{i=0}^{n-1} \left((-1)^i\frac{(t_n-t_i)^k}{k!} + (-1)^{n-i}\frac{(t_n-t_{n-i})^k}{k!}\right) \\
%%   k!x^{(n-k)}(t_n) =& \sum_{i=0}^{n-1} \left((-1)^i(t_n-t_i)^k + (-1)^{n-i}(t_n-t_{n-i})^k\right) \\
%%   =& \sum_{i=0}^{n-1} \left((-1)^it_{n-i}^k + (-1)^{n-i}t_i^k\right) \\
%%   t_{n-i}^k =& \left(\omega^{i/2}+\omega^{-i/2}\right)^{2k} \\
%%   =& \sum_{j=0}^{2k}\binom{2k}{j}\omega^{(i/2)(2k-j)}\omega^{(-i/2)j} \\
%%   =& \sum_{j=0}^{2k}\binom{2k}{j}\omega^{(k-j)i} \\
%%   =& \sum_{j=-k}^k\binom{2k}{k-j}\omega^{ji}
%% \end{align*}

%% \newpage
%% Conjecture:
%% Let $n \in \mathbb{Z}^+$ and $p, q : [0, 1] \rightarrow \mathbb{R}$,
%% where $p < 0$ and $q > 0$ a.e..
%% Then there exist $x_1, \ldots, x_{n-1}$ where $0 \leq x_1 \leq x_2 \leq \ldots \leq x_{n-1} \leq 1$ such that for all $k \in$

\begin{theorem}
  Let $n \in \mathbb{N}^{>0}$, $I = (t_0, t_n)$ be an interval,
  $p : I \rightarrow \mathbb{R}^{>0}$, $q : I \rightarrow \mathbb{R}^{<0}$. \\
  Define \[
  \mathcal{F} = \left\{f : I \rightarrow \mathbb{R} \mid
  q \leq f \leq p,
  \forall i \in \{1, \ldots, n-1\} \left(
  \int_I f(t)(t_n-t)^{i-1} dt = 0
  \right)
  \right\}.
  \]
  Then if there exist $t_1 < \ldots < t_{n-1} \in I$ s.t. $f^* \in \mathcal{F}$,
  where
  \[
  f^*(t) =
  \begin{cases}
    p(t) & t_{2i} \leq t < t_{2i+1} \\
    q(t) & t_{2i+1} \leq t < t_{2i+2},
  \end{cases}
  \]
  then
  \begin{enumerate}
    \item
    $f^*$ maximizes $\int_I f(t)(t_n-t)^{n-1} dt$ over $f \in \mathcal{F}$, and
    \item
    $f^*$ is the unique maximizer up to a set of measure $0$.
  \end{enumerate}
\end{theorem}
\begin{proof}
  (Found with the help of Xander Heckett and Robert Trosten)
  Suppose $f^*$ exists as described.
  Let $f \in \mathcal{F}$.
  Then $f = f^*(1-\eta)$ for some $\eta : I \rightarrow \mathbb{R}^{\geq 0}$.
  For 1., suffices to show that
  \[
  \int_I f^*(t) (t_n-t)^{n-1} dt \geq \int_I f(t) (t_n-t)^{n-1} dt,
  \]
  i.e. that 
  \[
  \int_I f^*(t)\eta(t) (t_n-t)^{n-1} dt \geq 0.
  \]
  Let $a(t) = (t_1-t)\cdots(t_{n-1}-t)$.
  Then $f^*(t)a(t) > 0$ everywhere except at $t_i$,
  so
  \[
  \int_I f^*(t)\eta(t) a(t) dt \geq 0.
  \]
  Thus, suffices to show
  \[
  \int_I f^*(t)\eta(t) ((t_n-t)^{n-1}-a(t)) dt = 0.
  \]
  Note $(t_n-t)^{n-1}-a(t) \in \spn\{(t_n-t)^{i-1} | i \in \{1, \ldots, n-1\}\}$,
  so suffices to show
  \[
  \int_I f^*(t)\eta(t) (t_n-t)^{i-1} dt = 0
  \]
  for all $i \in \{1, \ldots, n-1\}$.
  Since $f^*\eta = f^*-f$
  and $f, f^* \in \mathcal{F}$,
  \[
  \int_I f^*(t)\eta(t) (t_n-t)^{i-1} dt =
  \int_I (f^*(t)-f(t)) (t_n-t)^{i-1} dt =  
  \int_I f^*(t)(t_n-t)^{i-1} dt - \int_I f(t)(t_n-t)^{i-1} dt
  = 0.
  \]
  For 2., suppose $f \neq f^*$ on a set of positive measure,
  so $\eta > 0$ on a set of positive measure,
  so inequality becomes strict.
\end{proof}
\newpage
\begin{lemma}
  Let $n \in \mathbb{N}^{>0}$, $I = (0, 4)$,
  $p(t) = 1$, and $q(t) = -1$
  under the conditions of the previous theorem.
  Then the values $t_i = 4\cos^2\left(\frac{i\pi}{2n}\right)$
  yield an $f^* \in \mathcal{F}$.
  Furthermore, $\int_I f^*(t)(t_n-t)^{n-1} dt = 4$.
\end{lemma}
\begin{proof}
  Note
  \begin{align*}
    f^*(t) &= 
    \begin{cases}
      1 & t_{2i} \leq t < t_{2i+1} \\
      -1 & t_{2i+1} \leq t < t_{2i+2}
    \end{cases} \\
    &= \sum_{i=0}^{n-1}(-1)^i\left([t \geq t_i]-[t \geq t_{i+1}]\right),
  \end{align*}
  where $[]$ denotes the Iverson bracket.
  We want to determine
  $\int_0^4f^*(t)(4-t)^{k-1}dt$
  for all $k \in \{1, \ldots, n\}$,
  which should be $4$ for $k = n$ and $0$ otherwise.
  Note
  \begin{align*}
    \int_0^4[t \geq t_i](4-t)^{k-1} dt = \int_{t_i}^4 (4-t)^{k-1} dt
    = -\frac{1}{k}(4-t)^k\Bigg|_{t_i}^4
    = \frac{1}{k}(4-t_i)^k
    = \frac{1}{k}t_{n-i}^k,
  \end{align*}
  so
  \begin{align*}
    \int_0^4f^*(t)(4-t)^{k-1}dt
    &= \int_0^4\left(\sum_{i=0}^{n-1}(-1)^i\left([t \geq t_i]-[t \geq t_{i+1}]\right)\right) (4-t)^{k-1}dt \\
    &= \sum_{i=0}^{n-1}(-1)^i\left(\int_0^4[t \geq t_i] (4-t)^{k-1}dt-\int_0^4[t \geq t_{i+1}] (4-t)^{k-1}dt\right) \\
    &= \sum_{i=0}^{n-1}(-1)^i\left(\frac{1}{k}t_{n-i}^k-\frac{1}{k}t_{n-(i+1)}^k\right) \\
    &= \frac{1}{k}\sum_{i=0}^{n-1}(-1)^i\left(t_{n-i}^k-t_{n-(i+1)}^k\right),
  \end{align*}
  so
  \begin{align*}
    k\int_0^4f^*(t)(4-t)^{k-1}dt
    &= \sum_{i=0}^{n-1}(-1)^i\left(t_{n-i}^k-t_{n-(i+1)}^k\right) \\
    &= \sum_{i=0}^{n-1}(-1)^it_{n-i}^k-\sum_{i=0}^{n-1}(-1)^it_{n-(i+1)}^k \\
    &= \sum_{i=0}^{n-1}(-1)^it_{n-i}^k+\sum_{i=0}^{n-1}(-1)^{i+1}t_{n-(i+1)}^k \\
    &= \sum_{i=0}^{n-1}(-1)^it_{n-i}^k+\sum_{i=0}^{n-1}(-1)^{n-i}t_i^k \\
    &= \sum_{i=0}^{n-1}((-1)^it_{n-i}^k+(-1)^{n-i}t_i^k).
  \end{align*}
  which we WTS is $4k=4n$ for $k = n$ and $0$ otherwise.
  
  Define $\omega = \exp\left(\frac{\pi}{n}\sqrt{-1}\right)$,
  so
  \[
  t_i = 4\cos^2\left(\frac{i\pi}{2n}\right)
  = \left(2\cos\left(\frac{i\pi}{2n}\right)\right)^2
  = (\omega^{i/2}+\omega^{-i/2})^2,
  \]
  so
  \begin{align*}
    t_i^k =& \left(\omega^{i/2}+\omega^{-i/2}\right)^{2k} \\
    =& \sum_{j=0}^{2k}\binom{2k}{j}\omega^{(i/2)(2k-j)}\omega^{(-i/2)j} \\
    =& \sum_{j=0}^{2k}\binom{2k}{j}\omega^{(k-j)i} \\
    =& \sum_{j=-k}^k\binom{2k}{k-j}\omega^{ji}.
  \end{align*}
  Then
  \begin{align*}
    k\int_0^4f^*(t)(4-t)^{k-1}dt
    &= \sum_{i=0}^{n-1}\left((-1)^i\sum_{j=-k}^k\binom{2k}{k-j}\omega^{j(n-i)}+(-1)^{n-i}\sum_{j=-k}^k\binom{2k}{k-j}\omega^{ji}\right).
  \end{align*}
  The rest is trig bashing, which I will copy from an earlier iteration of the writeup.
  There seems to be a missing factor of $(-1)^n$
  which I cannot account for at the moment
  (and which is relevant since the result is nonzero in the $k = n$ case),
  but it is what it is.

  \begin{align*}
  k\int_0^4f^*(t)(4-t)^{k-1}dt =& \sum_{i=0}^{n-1} \left((-1)^i\sum_{j=-k}^k\binom{2k}{k-j}\omega^{ji} + (-1)^{n-i}\sum_{j=-k}^k\binom{2k}{k-j}\omega^{j(n-i)}\right) \\
  =& \sum_{i=0}^{n-1} \left(\sum_{j=-k}^k\binom{2k}{k-j}(-1)^i\omega^{ji} + \sum_{j=-k}^k\binom{2k}{k-j}(-1)^{n-i}\omega^{j(n-i)}\right) \\
  =& \sum_{i=0}^{n-1} \left(\sum_{j=-k}^k\binom{2k}{k-j}(-1)^i\omega^{ji} + \sum_{j=-k}^k\binom{2k}{k+j}(-1)^{n-i}\omega^{-j(n-i)}\right) \\
  =& \sum_{i=0}^{n-1} \left(\sum_{j=-k}^k\binom{2k}{k-j}\omega^{ni}\omega^{ji} + \sum_{j=-k}^k\binom{2k}{k-j}\omega^{-n(n-i)}\omega^{-j(n-i)}\right) \\
  =& \sum_{i=0}^{n-1} \left(\sum_{j=-k}^k\binom{2k}{k-j}\omega^{(n+j)i} + \sum_{j=-k}^k\binom{2k}{k-j}\omega^{-(n+j)(n-i)}\right) \\
  =& \sum_{i=0}^{n-1} \left(\sum_{j=-k}^k\binom{2k}{k-j}\omega^{(n+j)i} + \sum_{j=-k}^k\binom{2k}{k-j}\omega^{(n+j)(i-n)}\right) \\
  =& \sum_{i=0}^{n-1} \left(\sum_{j=-k}^k\binom{2k}{k-j}\left(\omega^{(n+j)i} + \omega^{(n+j)(i-n)}\right)\right) \\
  =& \sum_{j=-k}^k \left(\binom{2k}{k-j}\sum_{i=0}^{n-1}\left(\omega^{(n+j)i} + \omega^{(n+j)(i-n)}\right)\right) \\
  =& \sum_{j=-k}^k \left(\binom{2k}{k-j}\left(\sum_{i=0}^{n-1}\omega^{(n+j)i} + \sum_{i=0}^{n-1}\omega^{(n+j)(i-n)}\right)\right) \\
  =& \sum_{j=-k}^k \left(\binom{2k}{k-j}\left(\sum_{i=0}^{n-1}\omega^{(n+j)i} + \sum_{i=-n}^{-1}\omega^{(n+j)i}\right)\right) \\
  =& \sum_{j=-k}^k \left(\binom{2k}{k-j}\sum_{i=-n}^{n-1}\omega^{(n+j)i}\right) \\
  =& \sum_{j=-k}^k \left(\binom{2k}{k-j}[2n|n+j]2n\right) \\
  =& \sum_{j=-k}^k \left(\binom{2k}{k-j}[k=n]([j=-k]+[j=k])2n\right) \\
  =& [k=n]2n\sum_{j=-k}^k \left(\binom{2k}{k-j}([j=-k]+[j=k])\right) \\
  =& [k=n]2n\left(\binom{2k}{k-(-k)}+\binom{2k}{k-k}\right) \\
  =& [k=n]2n(2) \\
  =& [k=n]4n,
  \end{align*}
  which is $4n$ when $k=n$ and $0$ otherwise, as desired.
\end{proof}
\newpage

\begin{theorem}
  Let $n \in \mathbb{N}^{>1}$, $g : \mathbb{R} \rightarrow \mathbb{R}$.
  Suppose $g$ is supported on an open interval $I = (a, b)$,
  $g$ is $n-2$ times differentiable,
  and $g^{(n-2)}$ is Lipschitz with constant $L$.
  Then
  \[
  \int_\mathbb{R} g(t) dt \leq \frac{1}{4^{n-1}(n-1)!}L(b-a)^n.
  \]
  Furthermore, this bound is tight.
\end{theorem}
\begin{proof}
  Since $g$ is supported on $I$, so is $g^{(n-2)}$.
  Since $g^{(n-2)}$ is also Lipschitz,
  there is some $f$ supported on $I$ such that
  $g^{(n-2)}(t) = \int_a^t f(t) dt$
  and $-L \leq f \leq L$.
  By the Cauchy formula for repeated integration,
  and since $g(t) = 0$ for $t \leq a$,
  \[
  g^{(n-2-k)}(t) = \frac{1}{k!}\int_a^t f(s)(t-s)^kds
  \]
  for $k \in \{0, \ldots, n-1\}$,
  where $g^{(-1)}(t) = \int_{-\infty}^tg(s)ds = \int_a^tg(s)ds$.
  Let $f_{norm}$ be $f$ normalized to fit the previous lemma,
  i.e. $f_{norm}(s) = \frac{1}{L}f\left(\frac{b-a}{4}s+a\right)$,
  so $f(t) = Lf_{norm}\left(\frac{4}{b-a}(t-a)\right)$.
  Then
  \[
  \int_0^4f_{norm}(s)(4-s)^{n-1}ds \leq 4,
  \]
  so substituting $t = b$, $k = n-1$,
  \begin{align*}
    \int_\mathbb{R}g(t)dt
    &= \frac{1}{(n-1)!}\int_a^bf(t)(b-t)^{n-1}dt \\
    &= \frac{1}{(n-1)!}\int_a^bLf_{norm}\left(\frac{4}{b-a}(t-a)\right)(b-t)^{n-1}dt \\
    &= \frac{1}{(n-1)!}\int_0^4Lf_{norm}(s) \left(\frac{b-a}{4}(4-s)\right)^{n-1}\frac{b-a}{4} ds \\
    &= \frac{1}{4^n(n-1)!}L(b-a)^n\int_0^4f_{norm}(s) (4-s)^{n-1} ds \\
    &\leq \frac{1}{4^{n-1}(n-1)!}L(b-a)^n.
  \end{align*}
  The bound may be shown tight by transforming $f^*$ from the previous lemma
  and repeatedly integrating to get $g$:
  $f(t) = Lf^*\left(\frac{4}{b-a}(t-a)\right)$,
  $g(t) = \frac{1}{(n-2)!}\int_a^tf(s)(t-s)^{n-2}ds$.
\end{proof}
\end{document}
