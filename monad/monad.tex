\documentclass{article}
\usepackage[margin=1in]{geometry}
\usepackage{amsmath}
\usepackage{amssymb}
\usepackage{graphicx}
\graphicspath{images/}

\DeclareMathOperator{\Cont}{Cont}
\DeclareMathOperator{\return}{return}
\DeclareMathOperator{\bind}{bind}
\DeclareMathOperator{\Type}{Type}

\begin{document}

\title{The Continuation Monad}
\author{Aresh Pourkavoos}
\maketitle

Within System F,
there is a monad $\Cont$
which maps type $\alpha$
to $\Cont\alpha := (\omega : \Type) \rightarrow (\alpha \rightarrow \omega) \rightarrow \omega$,
i.e. a polymorphic function parameterized by type $\omega$.
In this monad, $\return : (\alpha : \Type) \rightarrow \alpha \rightarrow \Cont\alpha$
is defined by $\return \alpha\ a\ \omega\ f := f\ a$,
where $a : \alpha$ and $f : \alpha \rightarrow \omega$.
%% Since the types of $a$ and $f$ are sufficient to infer $\alpha$ and $\omega$,
%% the type arguments may be omitted,
%% so $\return$ may just as easily be defined in the HM type system.
%% However, we will include the type arguments for explicitness.
The operation $\bind$ has type
\[(\alpha : \Type) \rightarrow (\beta : \Type) \rightarrow (\alpha \rightarrow \Cont\beta) \rightarrow \Cont\alpha \rightarrow \Cont\beta.\]
In order to see how it should be defined,
we may unpack its arguments:
\begin{align*}
  \alpha &: \Type \\
  \beta &: \Type \\
  f &: \alpha \rightarrow \Cont \beta \\
  g &: \Cont \alpha
\end{align*}
The return type, then, is $\Cont \beta = (\omega : \Type) \rightarrow (\alpha \rightarrow \omega) \rightarrow \omega$,
so we may take two more arguments:
\begin{align*}
  \omega &: \Type \\
  h &: \beta \rightarrow \omega
\end{align*}
So the new return type is $\omega$.
Given what we have, there are a few potential ways to obtain a term of type $\omega$:
\begin{itemize}
\item
  The type of $f$ may be expanded to
  $\alpha \rightarrow \l$
\end{itemize}

is defined by
\[
\bind \alpha\ \beta\ f\ g\ \omega\ h := g\ \omega\ (\lambda a.(f\ a\ \omega\ h)).
\]

%% Let's ensure that this typechecks.
%% From the type of $f$, we may infer that $a : \alpha$,
%% so

%% where $f : \alpha \rightarrow \Cont \beta = \alpha \rightarrow (\beta \rightarrow \omega) \rightarrow \omega$,

\end{document}
