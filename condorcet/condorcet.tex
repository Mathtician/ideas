\documentclass{article}
\usepackage{amsmath}
\usepackage{amssymb}
\usepackage{graphicx}
\usepackage[margin=1in]{geometry}
\usepackage{hyperref}
\usepackage{caption}
\usepackage{float}
\graphicspath{{images/}}
\hypersetup{
  colorlinks=true,
  urlcolor=blue,
}
\begin{document}

\title{Condorcet's Nightmare}
\author{Aresh Pourkavoos}
\maketitle

The field of social choice theory deals with
how to take the preferences of a group of people
and combine them into a single decision for everyone.
This includes voting systems,
where each voter ranks a set of candidates first to last
and a single candidate is chosen as the winner.
A simple example is plurality:
take the candidate who is ranked first by the most people.
However, plurality is terrible for a number of reasons.
For example, if there are more extreme candidates A and C
who divide the voter base almost evenly for first choice,
but also a moderate candidate B who is most people's second choice,
only A and C have a real chance to win the election
despite B arguably being the ``best'' decision.

In place of plurality, a number of other systems have been proposed,
all with various ``desirable'' properties.
At the same time, it has been proven that
certain combinations of these properties cannot be achieved simultaneously,
the most famous example of which is Arrow's theorem.
One property is known as Condorcet consistency,
which states that if a candidate beats every other candidate in a ``head-to-head'' election,
i.e. an election with all but the two candidates removed,
then they should also win the overall election.
In this case, the candidate is called the Condorcet winner.
The mathematician for whom this is named, Marquis de Condorcet,
also showed that a Condorcet winner does not always exist
by providing the following counterexample
with 3 candidates A, B, and C and 3 voters:

\begin{center}
  \begin{tabular}{|c|c c c|}
    \hline
    Ranks & \#1 & \#2 & \#3 \\ \hline
    Voter 1 & A & B & C \\ \hline
    Voter 2 & B & C & A \\ \hline
    Voter 3 & C & A & B \\ \hline
  \end{tabular}
\end{center}

In head-to-head elections,
A beats B, B beats C, and C beats A
like a game of rock-paper-scissors,
which comes from the symmetric nature of the preferences:
Voters 2 and 3 have preferences which are ``rotations'' of Voter 1's,
so

But can similar things be done for other non-transitive games?
For example, there is a variant of RPS
which adds two new options: lizard and Spock.
Each option beats two others and loses to two others.
We can construct an analogous example with 5 candidates and 5 voters:

\begin{center}
  \begin{tabular}{|c|c c c c c|}
    \hline
    Ranks & \#1 & \#2 & \#3 & \#4 & \#5 \\ \hline
    Voter 1 & A & B & C & D & E \\ \hline
    Voter 2 & B & C & D & E & A \\ \hline
    Voter 3 & C & D & E & A & B \\ \hline
    Voter 4 & D & E & A & B & C \\ \hline
    Voter 5 & E & A & B & C & D \\ \hline
  \end{tabular}
\end{center}

As in the example with 3 candidates,
the voters' preferences are the rotations of Voter 1's,
so each candidate beats the next two candidates in the alphabet,
with A coming directly ``after'' E,
e.g. D beats E and A.
This can be done for any odd number of candidates and the same number of voters,
so that each candidate beats half of the others and loses to the other half.
However, it's also possible to construct the same outcome for 5 candidates
with only 3 voters:
\begin{center}
  \begin{tabular}{|c|c c c c c|}
    \hline
    Ranks & \#1 & \#2 & \#3 & \#4 & \#5 \\ \hline
    Voter 1 & A & B & C & D & E \\ \hline
    Voter 2 & C & D & E & A & B \\ \hline
    Voter 3 & E & B & D & A & C \\ \hline    
  \end{tabular}
\end{center}

\begin{center}
  \begin{tabular}{|c|c c c c c c c|}
    \hline
    Ranks & \#1 & \#2 & \#3 & \#4 & \#5 & \#6 & \#7 \\ \hline
    Voter 1 & A & B & C & D & E & F & G \\ \hline
    Voter 2 & F & C & G & D & A & E & B \\ \hline
    Voter 3 & E & G & B & D & F & A & C \\ \hline
  \end{tabular}
\end{center}

\end{document}
