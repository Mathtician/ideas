\documentclass{article}
\usepackage{amsmath}
\usepackage{amssymb}
\usepackage{amsthm}
\usepackage{graphicx}
\usepackage[margin=1in]{geometry}
\usepackage{hyperref}
\usepackage{caption}
\usepackage{float}
\graphicspath{{images/}}
\hypersetup{
  colorlinks=true,
  urlcolor=blue,
}
\begin{document}

\title{The Generalized Hockey-stick Identity}
\author{Aresh Pourkavoos}
\maketitle

\newtheorem*{GHSI}{Generalized Hockey-stick Identity}

\begin{GHSI}

  For all natural variables,
  \[\sum_{j_1+j_2=J}\binom{j_i+k_1}{j_1}\binom{j_2+k_2}{j_2} = \binom{J+k_1+k_2+1}{J}.\]

\end{GHSI}

\begin{proof}
  (based on Pascal's rule)
  Let $J, k_1, k_2 \in \mathbb{N}$. Define
  \[f(j, k) = \frac{(j+k)!}{j!k!} \binom{j+k}{k} = \binom{j+k}{j}\]
  so that the identity may be rewritten 
  \[\sum_{j_1+j_2=J}f(j_1, k_1)f(j_2, k_2) = f(J, k_1+k_2+1).\]
  Note $f(0, k)=f(j, 0)=1$ and $f(j+1, k+1)=f(j, k+1)+f(j+1, k)$ for all $j, k\in\mathbb{N}$.

  Proof by induction over $J$ and $k_2$:
  \begin{enumerate}
  \item $J=0$ is trivial:
    \begin{align*}
      \sum_{j_1+j_2=0}f(j_1, k_1)f(j_2, k_2) &= f(0, k_1+k_2+1) \\
      f(0, k_1)f(0, k_2) &= f(0, k_1+k_2+1) \\
      1\times1 &= 1
    \end{align*}
  \item $k_2=0$ reduces to the normal hockey stick identity:
    \begin{align*}
      \sum_{j_1+j_2=J}f(j_1, k_1)f(j_2, 0) &= f(J, k_1+0+1) \\
      \sum_{j_1=0}^{J}f(j_1, k_1) &= f(J, k_1+1)
    \end{align*}
  \item $J=J'+1$, $k_2=k_2'+1$: Assume
    \begin{align*}
      \sum_{j_1+j_2=J'}f(j_1, k_1)f(j_2, k_2) = f(J', k_1+k_2+1), \\
      \sum_{j_1+j_2=J}f(j_1, k_1)f(j_2, k_2') = f(J, k_1+k_2'+1).
    \end{align*}
    Then
    \begin{align*}
      & \sum_{j_1+j_2=J}f(j_1, k_1)f(j_2, k_2) \\
      &= \sum_{j_1=0}^{J}f(j_1, k_1)f(J-j_1, k_2) \\
      &= \sum_{j_1=0}^{J'}f(j_1, k_1)f(J-j_1, k_2)+f(J, k_1) \\
      &= \sum_{j_1=0}^{J'}f(j_1, k_1)(f(J'-j_1, k_2)+f(J-j_1, k_2'))+f(J, k_1) \\
      &= \sum_{j_1=0}^{J'}(f(j_1, k_1)f(J'-j_1, k_2)+f(j_1, k_1)f(J-j_1, k_2'))+f(J, k_1) \\
      &= \sum_{j_1=0}^{J'}f(j_1, k_1)f(J'-j_1, k_2)+\sum_{j_1=0}^{J'}f(j_1, k_1)f(J-j_1, k_2')+f(J, k_1) \\
      &= \sum_{j_1=0}^{J'}f(j_1, k_1)f(J'-j_1, k_2)+\sum_{j_1=0}^{J}f(j_1, k_1)f(J-j_1, k_2') \\
      &= \sum_{j_1+j_2=J'}f(j_1, k_1)f(j_2, k_2)+\sum_{j_1+j_2=J}f(j_1, k_1)f(j_2, k_2') \\
      &= f(J', k_1+k_2+1)+f(J, k_1+k_2'+1) \\
      &= f(J, k_1+k_2+1). \\
    \end{align*}
  \end{enumerate}
\end{proof}

\begin{proof}
  (based on polynomial expansion)
\end{proof}

\begin{proof}
  (based on combinatorics)
  The left and right sides of the equation can be interpreted as cardinalities of sets,
  and a bijection between these sets is sufficient to show that their cardinalities are equal.
  $\binom{j+k}{j}$ becomes the set of ways to arrange $j$ red and $k$ blue objects in a line,
  multiplication becomes the Cartesian product of sets, and summation becomes the disjoint union.
  Then the left side represents the number of pairs of
  arrangements of $j_1$ red and $k_1$ blue objects and
  arrangements of $j_2$ red and $k_2$ blue objects
  such that $j_1+j_2=J$.
  Likewise, the right side is the number of arrangements of $J$ red and $k_1+k_2+1$ blue objects.

  Left$\rightarrow$right: Take the first arrangement in the given pair (with $j_1+k_1$ objects),
  add a blue object to its right as a separator,
  and add the second arrangement (with $j_2+k_2$ objects) to the right of that
  to obtain an arrangement of $j_1+j_2=J$ red and $k_1+k_2+1$ blue objects.

  Right$\rightarrow$left: Take the $(k_1+1)^{th}$ blue object in the given arrangement
  (with $J$ red and $k_1+k_2+1$ blue objects) to act as a separator,
  so the arrangement to its left has $k_1$ blue objects
  and the arrangement to its right has $k_2$ blue objects.
  The number of red objects on the left and right ($j_1$ and $j_2$, respectively)
  are not determined by the split, but they must still add to $J$.
  
\end{proof}

\end{document}
