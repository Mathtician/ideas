\documentclass{article}
\usepackage{amsmath}
\usepackage{amssymb}
\usepackage{graphicx}
\usepackage[margin=1in]{geometry}
\usepackage{hyperref}
\usepackage{caption}
\usepackage{float}
\graphicspath{{images/}}
\hypersetup{
  colorlinks=true,
  urlcolor=blue,
}
\begin{document}

\title{Deriving a Trig Identity}
\author{Aresh Pourkavoos}
\maketitle

\newcommand{\cis}{\mathrm{cis}}

In acoustics, there is a phenomenon known as ``beating,''
which occurs when two tones with similar frequencies are played together.

\begin{align*}
  \cos(ax)+\cos(bx) &= 2\cos\left(\frac{a+b}{2}x\right)\cos\left(\frac{a-b}{2}x\right) \\
  \sin(ax)+\sin(bx) &= 2\sin\left(\frac{a+b}{2}x\right)\cos\left(\frac{a-b}{2}x\right)
\end{align*}

\begin{align*}
  \cos(2ax)+\cos(2bx) &= 2\cos((a+b)x)\cos((a-b)x) \\
  \sin(2ax)+\sin(2bx) &= 2\sin((a+b)x)\cos((a-b)x)
\end{align*}

\[(\cos(2a)+\cos(2b))+i(\sin(2a)+\sin(2b))=2\cos(a+b)\cos(a-b)+2i\sin(a+b)\cos(a-b)\]

\[\cis(2a)+\cis(2b)=2\cis(a+b)\cos(a-b)\]

\[\cis(2a)+\cis(2b)=\cis(a+b)(\cis(a-b)+\cis(b-a))\]

\end{document}
