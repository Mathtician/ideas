\documentclass{article}
\usepackage{amsmath}
\usepackage{amssymb}
\usepackage{graphicx}
\usepackage[margin=1in]{geometry}
\usepackage{hyperref}
\usepackage{caption}
\usepackage{float}
\graphicspath{{images/}}
\hypersetup{
  colorlinks=true,
  urlcolor=blue,
}
\begin{document}

\title{Deriving a Trig Identity}
\author{Aresh Pourkavoos}
\maketitle

\newcommand{\cis}{\mathrm{cis}}

In acoustics, there is a phenomenon known as ``beating,''
which occurs when two tones with similar frequencies are played together.
Instead of two separate notes,
we hear a single note whose frequency is the average of the given notes
and which periodically becomes quieter and louder.
The phenomenon may be explained by the following trigonometric identity:
\[
\frac{1}{2}(\cos(ax)+\cos(bx))
= \cos\left(\frac{a+b}{2}x\right)\cos\left(\frac{a-b}{2}x\right)
\]
The left side describes a waveform as a function of $x$ (time)
as an average of two waves with (angular) frequencies $a$ and $b$.
The right side, on the other hand, is a product of functions,
the first of which has the average frequency of the two tones
and the second of which describes the beating:
if $a$ and $b$ are close, this function has a low frequency
and may be heard not as a pitch,
but as a gradual change in volume of the note.
To avoid fractions,
we may double $a$, $b$, and both sides of this identity
to obtain:
\[\cos(2ax)+\cos(2bx) = 2\cos((a+b)x)\cos((a-b)x)\]
Since $x$ only exists when multiplied by $a$ or $b$,
We may replace $ax$ with $a$ and $bx$ with $b$
to obtain the equivalent identity
\[\cos(2a)+\cos(2b) = 2\cos(a+b)\cos(a-b)\]
An analogous identity for sines also exists:
\[\sin(2a)+\sin(2b) = 2\sin(a+b)\cos(a-b)\]
Both may be derived simultaneously
by interpreting them as the real and imaginary parts of a unified equation:
\[(\cos(2a)+\cos(2b))+i(\sin(2a)+\sin(2b))=2\cos(a+b)\cos(a-b)+2i\sin(a+b)\cos(a-b)\]
\[\cis(2a)+\cis(2b)=2\cis(a+b)\cos(a-b)\]

\[\cis(2a)+\cis(2b)=\cis(a+b)(\cis(a-b)+\cis(b-a))\]

Envelope of $c\cos(ax)+d\cos(bx)$: $\sqrt{c^2+2cd\cos((a-b)x)+d^2}$
\begin{align*}
  (c\cos(a)+d\cos(b))^2 &= c^2\cos^2(a)+2cd\cos(a)\cos(b)+d^2\cos^2(b) \\
  (c\cos(a)+d\cos(b))^2 &= c^2\cos^2(a)+2cd\cos(a)\cos(b)+d^2\cos^2(b) \\
\end{align*}

\end{document}
