\documentclass{article}
\usepackage{amsmath}
\usepackage{amssymb}
\usepackage{graphicx}
\usepackage[margin=1in]{geometry}
\usepackage{hyperref}
\usepackage{caption}
\usepackage{float}
\graphicspath{{images/}}
\hypersetup{
  colorlinks=true,
  urlcolor=blue,
}
\begin{document}

\title{A Projection}
\author{Aresh Pourkavoos}
\maketitle

\[
Q =
\frac{1}{2\sqrt{2+\varphi}}
\begin{pmatrix}
  \varphi & \varphi & 1 & -1 & 0 & 0 \\
  1 & -1 & 0 & 0 & \varphi & \varphi \\
  0 & 0 & \varphi & \varphi & 1 & -1 \\
  -1 & -1 & \varphi & -\varphi & 0 & 0 \\
  \varphi & -\varphi & 0 & 0 & -1 & -1 \\
  0 & 0 & -1 & -1 & \varphi & -\varphi
\end{pmatrix}
\]

The $6 \times 6$ matrix above is orthonormal,
meaning that it represents a rotation of 6-dimensional space.
$\varphi$ is the golden ratio $\frac{1+\sqrt{5}}{2}$,
the positive solution to the equation $\varphi^2 = \varphi+1$.
I found this matrix while considering the cut-and-project method
of generating aperiodic tilings,
and I think it illustrates an element of this method
that is often overlooked.

\end{document}
