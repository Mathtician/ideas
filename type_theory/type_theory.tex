\documentclass{article}
\usepackage{amsmath}
\usepackage{graphicx}
\usepackage{bm}
\usepackage[margin=1in]{geometry}
\graphicspath{{images/}}

\begin{document}

\title{Notes on Type Theory}
\author{Aresh Pourkavoos}
\maketitle

Base types, sum/product/function types, dependent types, inductive types, universes \\
Base types: $\bot$ AKA false, $\top$ AKA true, bool \\
\#:$\top$, 0:bool, 1:bool ($\bot$ has no terms), $\bot$:Type, $\top$:Type, bool:Type \\

Dependent types: $\sum$ AKA $\exists$, $\prod$ AKA $\forall$ \\

\begin{align}
  & A+B & A \lor B \\
  & A \times B & A \land B \\
  & B^A & A \rightarrow B
\end{align}

(1):
The union type of types A and B has terms which are either of type A or of type B.
For example, if A had 2 terms (and thus stored one bit of information) and B had 3 terms (one trit),
there would be $2+3=5$ possible values of the union.

(2):
The pair type of A and B has terms which contain both a term of type A and a term of type B.
Using the previous examples of A and B, there would be $2 \times 3 = 6$ possible pairs.

(3):
The function type that goes from A to B has terms which associate exactly one (non-unique) element of B with each element of A.
Since for each of the 2 terms of A (input), there are 3 choices for the corresponding element of B (output),
there would be $3^2=9$ possible functions.

\begin{align}
  & A \times (B+C) = (A \times B)+(A \times C) & A \land (B \lor C) \equiv (A \land B) \lor (A \land C) \label{distribution} \\
  & C^{A+B} = C^A \times C^B & (A \lor B) \rightarrow C \equiv (A \rightarrow C) \land (B \rightarrow C) \label{cases} \\
  & (B \times C)^A = B^A \times C^A & A \rightarrow (B \land C) \equiv (A \rightarrow B) \land (A \rightarrow C) \\
  & C^{A \times B} = (C^B)^A & (A \land B) \rightarrow C \equiv A \rightarrow (B \rightarrow C) \label{currying} \\
  & A+0 = A & A \lor F \equiv A \label{false} \\
  & A \times 1 = A & A \land T \equiv A \label{true}
\end{align}
\ref{distribution}

\end{document}
