\documentclass{article}
\usepackage[margin=1in]{geometry}
\usepackage{amsmath}
\usepackage{amssymb}
\usepackage{graphicx}
\usepackage{hyperref}
\usepackage{tikz}
\usepackage{algorithm}
\usepackage[noend]{algpseudocode}

\graphicspath{images/}
\newcommand{\abs}[1]{\left\vert #1 \right\vert}

\begin{document}

\title{Common Nets}
\author{Aresh Pourkavoos}
\maketitle

\section{Introduction}

A video by Matt Parker
\footnote{
Parker 2022, `` Can the Same Net Fold into Two Shapes?'',
\href{https://youtu.be/jOTTZtVPrgo}{youtu.be/jOTTZtVPrgo}
}
explores the following problem:
are there multiple boxes which may be cut and unfolded
to produce the same 2D net?
Unless mentioned otherwise,
a \textit{box} is a rectangular prism with positive integer side lengths,
and a \textit{net} is a connected shape made of unit squares
whose edges are aligned with those of the box when folded.
Parker summarizes the following results,
mostly from Ryuhei Uehara and collaborators:
\begin{itemize}
\item
  The $1 \times 1 \times 5$ and $1 \times 2 \times 3$ boxes
  have a common net consisting of $22$ unit squares.
  In fact, all $2263$ such nets have been enumerated.
  \footnote{
  Matsui and Uehara, \href{
    https://www.jaist.ac.jp/~uehara/etc/origami/nets/index-e.html}{
    jaist.ac.jp/~uehara/etc/origami/nets/index-e.html}}
\item
  For all positive integers $k$,
  the $1 \times 5 \times 2k$ and $1 \times 1 \times (6k+2)$ boxes
  have a common net (which also tiles the plane),
  so there are infinitely many pairs of distinct boxes with a common net.
  \footnote{
  Uehara 2008, ``Polygons Folding to Plural Incongruent Orthogonal Boxes''
  }
\item
  One of the $2263$ nets mentioned above
  also folds into a double-covered $1 \times 11$ rectangle,
  which may be considered a $0 \times 1 \times 11$ ``box.''
  However, some of the unit squares are folded in half in the process.
  To avoid this, we may scale the net by a factor of $2$,
  producing a common net of the $0 \times 2 \times 22$,
  $2 \times 2 \times 10$, and $2 \times 4 \times 6$ boxes.
  \footnote
      {Abel et al 2011,
        ``Common Developments of Several Different Orthogonal Boxes''
      }
    \item
      The $1 \times 3 \times 3$ and $1 \times 1 \times 7$ boxes
      have a common net consisting of $30$ unit squares.
      This net also folds along diagonal lines
      to produce a $\sqrt{5} \times \sqrt{5} \times \sqrt{5}$ ``box.''
      \footnote{
      Horiyama et al 2015,
      ``Common Developments of Three Incongruent Boxes of Area 30''
      }
    \item
      For all positive integers $k$,
      the following three boxes have a common net:
      \begin{alignat*}{3}
        &2k &\times& (4k+9) &\times& 2(16k+13) \\
        &(4k+3) &\times& 2(k+3) &\times& 8(4k+3) \\
        &(4k+3) &\times& 2(4k+3) &\times& 2(7k+12)
      \end{alignat*}
      So there are infinitely many triples of distinct boxes with a common net.
      \footnote{
      Shirakawa and Uehara 2013,
      ``Common Developments of Three Different Orthogonal Boxes''
      }
\end{itemize}
Parker concludes with the observation that
the smallest surface area shared by three boxes is $46$,
the area of the $1 \times 1 \times 11$, $1 \times 2 \times 7$,
and $1 \times 3 \times 5$ boxes.
However, it is unknown whether these boxes have a common net.
Here, we present an algorithm to search for these nets.

\section{Definitions}

First, we define a net more precisely.
Consider the infinite square grid as a graph $G$,
where each vertex represents a square
and each edge connects a pair of adjacent squares.
(In other words, $G$ is the $1$-skeleton of
the dual of the square grid.)
To maintain visual clarity,
the vertices of $G$ will be referred to as squares.
A \textit{generalized polyomino}, or \textit{GP},
is a connected subgraph $(S, E)$ of $G$,
up to isomorphisms of $G$
(i.e. translation, $90$-degree rotation, and reflection),
where $S$ and $E$ are the GP's square and edge sets, respectively.
In other words, it is a polyomino
with additional information about which squares are considered adjacent.
(It is important to specify that
only isomorphisms of $G$ can make two GPs equivalent,
since many distinct polyominoes are isomorphic as graphs,
e.g. the two trominoes.)
Squares which are adjacent in $G$ but not in the GP
may be considered to have a cut between them.

%% WLOG, we may fix a square $o$ of $G$, dubbed the \textit{origin},
%% and require all GPs to include it.

A \textit{partial net} of an $a \times b \times c$ box
is a GP $(S, E)$
equipped with a map from $S$ to the surface of the box
which represents a folding procedure, i.e.
\begin{itemize}
\item
  Each square in $S$ is mapped to one of the $2ab+2ac+2bc$ squares on the box,
  as well as one of $4$ orientations.
  (Since boxes are not chiral,
  we do not need all $8$ ways to lay one square on another.
  This may be thought of as coloring one side of the plane
  from which the GP is cut,
  then requiring that color to face the outside of the box.)
\item
  No two squares in $S$ map to the same square on the box,
  even if their orientations are different.
\item
  Adjacent squares in $S$ map to adjacent squares on the box
  with aligned orientations,
  e.g. if square $a$ is to the left of $b$ in $G$,
  and $a$ and $b$ are connected by an edge in $E$,
  then the right side of $a$'s corresponding oriented square on the box
  overlaps with the left side of $b$'s square.
\end{itemize}
A \textit{net} is a partial net such that all squares on the given box are covered.

We wish to answer the following question:

\begin{center}
  \fbox{
    Does there exist a GP which is a net of all three of
    the $1 \times 1 \times 11$, $1 \times 2 \times 7$,
    and $1 \times 3 \times 5$ boxes?
  }
\end{center}

Since the boxes have surface area $46$,
the shared net would be a generalized $46$-omino.
However, only recently
has the total number of (non-generalized) $46$-ominoes been computed,
at approximately $8.57 \times 10^{24}$.
\footnote{
Mason 2023, ``Counting polyominoes of size 50'',
\href{https://oeis.org/A000105/a000105\_1.pdf}{oeis.org/A000105/a000105\_1.pdf}
}
Including the edges of the graph to account for all GPs of size $46$
would raise this number significantly,
so we should try to restrict the search space.

First, note that removing edges from a net always results in another net
(as long as the graph remains connected),
since edges only constrain how squares may map to the surface of the box.
Also, every GP has a spanning tree,
so we only need to consider the set of GPs which are also trees.
Additionally, (Uehara 2008) observed that
if four squares are connected by three edges
to form a larger square with an internal cut,
then the fourth edge may be added to the net
(i.e. the last pair of adjacent squares may be glued together),
as shown below,
without affecting its ability to fold onto the box.

\begin{center}
  \begin{tikzpicture}
    \draw[dashed] (0,0) -- (0,2) -- (2,2) -- (2,0) -- (0,0);
    \draw[dotted] (1,0) -- (1,2) (0,1) -- (1,1);
    \draw (1,1) -- (2,1);
    \draw[->] (2.25,1) -- (2.75,1);
    \draw[dashed] (3,0) -- (3,2) -- (5,2) -- (5,0) -- (3,0);
    \draw[dotted] (4,0) -- (4,2) (3,1) -- (5,1);
  \end{tikzpicture} \\
  Solid lines are cut edges,
  dotted lines are joined edges,
  and dashed lines may be either.  
\end{center}

Given a GP, we may repeat this gluing process
until the configuration on the left of the above figure no longer exists.
Since gluing can map many trees to the same GP,
the size of the search space is further reduced.
Let us call these glued trees
\textit{weakly generalized polyominoes} or \textit{WGPs}.
Then the answer to the central question above
is not changed if we replace ``GP'' with ``WGP,''
so from here, the definition of ``partial net'' (and thus of ``net'')
is restricted to refer to WGPs only.

WGPs have no cuts which terminate on the interior,
i.e. all internal cuts terminate on the edges.
Then if a given WGP has an underlying polyomino which is simple
(i.e. has no holes),
then an internal cut would split it in two,
so the WGP is the polyomino itself with all adjacent squares joined.
For this reason, previous work was focused on simple polyominoes.
In particular, the 2263 common nets
of the $1 \times 1 \times 5$ and $1 \times 2 \times 3$ boxes
listed by (Matsui and Uehara)
were found by a search of simple polyominoes.
However, there are WGPs whose underlying polyominoes have holes,
such as the one below, which is a net of a $1 \times 2 \times 2$ box.
\begin{center}
  \begin{tikzpicture}
    \draw
    (0,0) -- (2,0) -- (2,1) -- (1,1) -- (1,2) --
    (2,2) -- (2,0) -- (3,0) -- (3,1) -- (6,1) --
    (6,4) -- (4,4) -- (4,3) -- (0,3) -- (0,0);
    \draw[dashdotted] (1,0) -- (1,1) (1,2) -- (1,3);
    \draw[dotted] (2,2) -- (2,3);
    \draw[dashdotted] (3,1) -- (3,3) (4,1) -- (4,3);
    \draw[dotted] (5,1) -- (5,4);
    \draw[dashdotted] (0,1) -- (1,1);
    \draw[dotted] (2,1) -- (3,1);
    \draw[dotted] (0,2) -- (1,2) (2,2) -- (6,2);
    \draw[dashdotted] (4,3) -- (6,3);
  \end{tikzpicture} \\
  Dashdotted lines are joined edges which are folded to produce the box.
  Note the internal cut edge.
\end{center}
Crucially, the underlying polyomino (with the internal cut removed)
is not a net of any box.

\section{Algorithm}
\subsection{Depth-first search}
The first algorithm to find common nets is based on
%% a more general search algorithm which is given:
%% \begin{itemize}
%% \item
%%   A partially ordered set $(S, \leq)$
%% \item
%%   A predicate $A$ on $S$ which is downward closed,
%%   i.e. 
%% \end{itemize}
a depth-first search function which accepts:
\begin{itemize}
\item
  $\mathcal{N}$, a set of common partial nets which
  do not contain each other,
  i.e. for all distinct $Q, R \in \mathcal{N}$,
  $Q$ does not contain $R$.
\item
  $P$, a common partial net
  which does not contain, and is not contained by, any $Q \in \mathcal{N}$.
\end{itemize}
The function returns the set of all common nets
which contain $P$ but do not contain any $Q \in \mathcal{N}$.

\begin{algorithm}
  \caption{Common net search}
  \begin{algorithmic}[1]
    \Function{Search}{$P, \mathcal{N}$}
    \If {$P$ has $46$ squares}
    \Return $\{P\}$
    \EndIf
    \State $\mathcal{R} \gets \{\}$
    \State $\mathcal{M} \gets \{\}$
    \For{$e \in \{$edges from a square in $P$ to a square not in $P\}$}
    \State $P' \gets \Call{Glue}{P \cup \{e\}}$
    \If {$P'$ is not a common partial net}
    \textbf{continue}
    \EndIf
    \If {$P'$ contains any $Q \in \mathcal{N} \cup \mathcal{M}$}
    \textbf{continue}
    \EndIf
    \State $\mathcal{R} \gets \mathcal{R} \cup \Call{Search}{P',\mathcal{N} \cup \mathcal{M}}$
    \State $\mathcal{M} \gets \mathcal{M} \cup \{P'\}$
    \EndFor
    \Return $\mathcal{R}$
    \EndFunction
  \end{algorithmic}
\end{algorithm}

In other words, the algorithm repeatedly tries
to extend $P$ by one square,
searches the resulting WGP $P'$,
and then eliminates $P'$ for the rest of the search of $P$.
The correctness of Algorithm 1 relies on the fact that
every WGP which strictly contains $P$
must contain one of the $P'$.

Thanks to line 8,
each common net will be found exactly once during the search,
so there is no duplication of effort.
Further refinements should strive to maintain this property.
However, depending on $\abs{\mathcal{N} \cup \mathcal{M}}$,
line 8 may be very expensive.
In this case, the size is bounded by the depth of recursion ($46$)
and the number of $P'$ added at each level $i$,
which is no more than $2i+2$, the maximum perimeter of an $i$-omino.
Thus the extra time spent at each node
is quadratic in the recursion depth,
whereas the overlaps that are prevented
decrease the number of nodes exponentially.
Contrast this with a breadth-first search,
where $\mathcal{N}$ would contain an entire layer of the search tree,
i.e. the set of all common partial nets with $i$ squares.

\subsection{Restricting common partial nets}

In theory, Algorithm 1 would suffice to find all common nets of the
$1 \times 1 \times 11$, $1 \times 2 \times 7$, and $1 \times 3 \times 5$ boxes,
when called with $P$ as a single square and $\mathcal{N}$ as the empty set.
However, it suffers from two major problems:
\begin{itemize}
\item
  It is called for all common partial nets $P$ at some point during the search.
  While this guarantees that all common nets will be found,
  it would be ideal to search only a subset of common partial nets
  which still includes all common (total) nets.
\item
  Checking whether one partial net $A$ contains another $B$ is expensive,
  as $B$ must be overlaid on $A$
  with all combinations of translation, rotation, and reflection.
\end{itemize}
To ameliorate these issues,
we can try to find a set $\mathcal{S} = \{S_1, \ldots, S_k\}$ of partial nets
such that for all common nets $P$,
there exists an $S \in \mathcal{S}$ such that $P$ contains $S$.
Then we can run Algorithm 1 $k$ times on the inputs
\[
(S_1, \{\}), (S_2, \{S_1\}), (S_3, \{S_1, S_2\}), \ldots,
(S_k, \{S_1, S_2, S_3, \ldots, S_{k-1}\})
\]
to obtain each net exactly once.

\newpage

\newcommand{\code}{\texttt}

Edgemap consists of:
\begin{itemize}
\item
  \code{cx, cy : u8} - 
\item
  \code{rows: Vec<u64>} - array of rows of edges
  \begin{itemize}
  \item
    \code{(rows[2*y]<<x)>>63 = } presence of edge
    from square \code{(x, y)} to \code{(x+1, y)} relative to top left
  \item
    \code{(rows[2*y+1]<<x)>>63 = } presence of edge
    from \code{(x, y)} to \code{(x, y+1)}
  \end{itemize}
\end{itemize}

Partial net consists of:
\begin{itemize}
\item
  Edgemap for joined edges
\item
  Horizontal reflection of previous map
\item
  Edgemap for cut edges
\end{itemize}



%% recall that a WGP is a tree after applying the gluing operation.
%% A tree may be ``grown'' from any of its vertices,
%% i.e. given a tree $T$ and a vertex $v$ of $T$,
%% there is a sequence of trees
%% starting with the graph containing only $v$ and ending with $T$,
%% where each tree is created
%% by adding a single vertex to the previous one.
%% At each node:
%% \begin{itemize}
%% \item
%%   Sets of joined and cut edges
%%   \begin{itemize}
%%     \item
%%       All joined edges are true,
%%       all cut edges are false,
%%   \end{itemize}
%% \item
%%   Set of interior squares
%% \item
%%   For each valid box-orbit,
%%   mapping from plane to box (position and orientation)
%% \item
%%   List of stages used to construct current shape
%%   \begin{itemize}
%%   \item
%%     For each stage, list of no-gos made by adding a single (joined) edge
%%   \end{itemize}
%% \end{itemize}

%% Assurances:
%% \begin{itemize}
%% \item
%%   The three edge sets are disjoint
%% \item
%%   The joined edges form a connected graph
%% \item
%%   Each boundary edge has exactly one endpoint inside the net and one outside
%%   \begin{itemize}
%%   \item
%%     ``Inside'' points are endpoints of the joined edges
%%   \end{itemize}
%% \item
%%   All edges between an inside and an outside point
%% \end{itemize}

%% Search function pseudocode:
%% \begin{itemize}
%% \item
%%   Initialize empty list of candidates
%% \item
%%   Let ``inside'' points be endpoints of the joined edges
%% \item
%%   Let a boundary edge
%% \item
%%   For each non-cut boundary edge:
%%   \begin{itemize}
%%   \item
%%     Try to add it to the graph, along with its new vertex
%%   \item
%%     Infer new edges to join by square completion
%%     \begin{itemize}
%%     \item
%%       If a cut edge is inferred,
%%       turn the boundary edge into a cut edge; skip this candidate
%%     \item
%%       If another boundary edge is inferred,
%%       turn it into a cut edge (for all other candidates)
%%       \begin{itemize}
%%       \item
%%         Inference is two-sided:
%%         if one boundary edge infers another,
%%         their cases are equivalent
%%       \end{itemize}
%%     \end{itemize}
%%   \item
%%     Check if adding the vertex created a no-go,
%%     or anything already in the candidate list
%%   \item
%%     If so, turn the boundary edge into a cut edge;
%%     skip this candidate
%%   \item
%%     For each valid orbit:
%%     \begin{itemize}
%%     \item
%%       If the new vertex maps to an occupied square,
%%       eliminate the orbit
%%     \item
%%       Else, add the new location to the mapping
%%     \end{itemize}
%%   \item
%%     If any box has all orbits eliminated,
%%     turn the boundary edge into a cut edge;
%%     skip this candidate
%%   \item
%%     Else, add the new graph to the candidate list
%%   \end{itemize}
%% \item
%%   Sort the candidate list by $
%%   (\text{box }1\text{ orbits}) \times
%%   (\text{box }2\text{ orbits}) \times
%%   (\text{box }3\text{ orbits})$, increasing
%% \item
%%   Initialize empty list of no-gos derived from current stage
%% \item
%%   For each candidate:
%%   \begin{itemize}
%%   \item
%%     Recursively call search with the candidate's
%%     edges, mappings, and stages/no-gos
%%   \item
%%     Add the candidate to the list of no-gos
%%   \end{itemize}
%% \end{itemize}

%% \newpage

%% SAT formulation \\
%% Variables:
%% \begin{itemize}
%% \item
%%   For each edge in (part of) the grid, a variable for whether it is joined
%% \item
%%   For each square in the grid, a variable for whether it is part of the net
%% \item
%%   For each box:
%%   \begin{itemize}
%%   \item For each square of the box:
%%   \begin{itemize}
%%   \item
%%     Variables describing the coordinates of the corresponding grid square in binary
%%     (Gray code?)
%%   \item
%%     Variables describing the orientation of the corresponding grid square
%%   \end{itemize}
%%   \end{itemize}
%% \end{itemize}

%% Constraints:
%% \begin{itemize}
%% \item
%%   For each grid square:
%%   \begin{itemize}
%%     \item
%%       If three of the edges around the square are joined, then so is the fourth one
%%     \item
%%       The square is part of the net iff it is adjacent to a joined edge
%%     \item
%%       For each box, the square is part of the net iff one of the squares of the box has its coordinates
%%   \end{itemize}
%% \item
%%   For each box:
%%   \begin{itemize}
%%     \item
%%     No two of the box's squares have the same grid coordinates
%%   \end{itemize}
%% \end{itemize}

\newpage

SAT formulation (2 boxes) divides each square into 4 square-edges \\
Variables:
\begin{itemize}
\item
  For each square-edge $a$ of the first box and square-edge $b$ of the second box,
  a variable $M_{ab}$ for whether $a$ maps to $b$
  \begin{itemize}
  \item
    Preprocessing: sets of 4 variables for same square of both boxes are equivalent
  \item
    Equivalence classes map a square of the first box to a square of the second box
    in one of $4$ orientations
  \item
    Fixes an orientation of the mapping: must be accounted for in symmetry breaking
  \end{itemize}
\item
  For each edge $e$ of the first box and edge $f$ of the second box,
  a variable $N_{efo}$ for whether $e$ maps to $f$
  in one of two orientations $o$
\item
  For each edge $e$ of the first box, a variable $J_e$ for whether $e$ is joined
\item
  For each edge $f$ of the second box, a variable $K_f$ for whether $f$ is joined
\end{itemize}
Constraints:
\begin{itemize}
\item $M_{ab}$ defines a bijection:
  \begin{itemize}
  \item
    For each square-edge $a$ of the first box,
    exactly one of $\{M_{ab}\ |\ b$ square-edge of second box$\}$
  \item
    For each square-edge $b$ of the second box,
    exactly one of $\{M_{ab}\ |\ a$ square-edge of first box$\}$
    %% \item
    %%   Could also reduce number of formulas
    %%   by only requiring at least one in one direction
    %%   and at most one in the other
    %% \item
    %%   May slow down inference
  \end{itemize}
\item
  For each edge $e$ of the first box, exactly one of
  $\{\lnot J_e\} \cup \{N_{efo}\ |\ f$ edge of second box, $o$ orientation$\}$
\item
  For each edge $f$ of the second box, exactly one of
  $\{\lnot K_f\} \cup \{N_{efo}\ |\ e$ edge of first box, $o$ orientation$\}$
\item
  If $N_{efo}$,
  then the corresponding square-edges on either side of $e$ and $f$
  are mapped to each other
  \begin{itemize}
  \item
    $\lnot N_{efo} \lor M_{e_0f_o}$,
    $\lnot N_{efo} \lor M_{e_1f_{1-o}}$
  \end{itemize}
\item
  %% The converse of the previous:
  %% $\lnot M_{e_0f_o} \lor \lnot M_{e_1f_{1-o}} \lor N_{efo}$
  %% (ensure solution is maximal)
  Gluing: if 3 edges around a degree-4 vertex are joined,
  then so is the 4th one (for both boxes)
\item
  The joined edges of the first box form a connected graph:
  \begin{itemize}
  \item
    Original: cs.stackexchange.com/questions/111410
  \item
    Doesn't work for partial assignments:
    want a contradiction as soon as one part of the graph
    is cut off from another
  \item
    Graph of squares and edges is planar:
    dual has vertices and edges
  \item
    Primal graph is connected iff dual complement graph is acyclic
  \item
    Cuts are dual to cycles: want a contradiction if dual non-joined edges form a cycle
  \item
    Given (simple) graph $G = (V, E)$ and $F \subseteq E$, determine if $(V, F)$ acyclic:
    \begin{itemize}
    \item
      Order $E = \{e_1, \ldots, e_n\}$, ideally by BFS
    \item
      For $i \in \{0, \ldots, n\}$, define $E_i = \{e_j \in E \mid j \leq i\}$,
      $F_i = F \cap E_i$
    \item
      For $i \in \{0, \ldots, n\}$, $u, v \in V$, add aux var $C_{iuv}$:
      whether $u$ is connected to $v$ by $F_i$
    \item
      For $v \in V$, $C_{0vv}$
    \item
      For $i \in \{0, \ldots, n-1\}$, if $e_{i+1} = \{a_i, b_i\}$, for $u, v \in V$:\\
      $C_{(i+1)uv} \leftrightarrow C_{iuv} \lor (C_{iua} \land e_i \in F \land C_{ibv}) \lor (C_{iub} \land e_i \in F \land C_{iav})$
    \end{itemize}
    \item
      Simplification:
      \begin{itemize}
      \item
        For $i \in \{0, \ldots, n\}$, $C_{ivv}$
      \item
      \end{itemize}
      %% For $i \in \{0, \ldots, n\}$, define $V_i = \{v \in V \mid \exists u \in V (\{u, v\} \in E_i) \land \exists w \in V (\{v, w\} \in E \setminus E_i)\}$
  \end{itemize}
%% \item
%%   Symmetry-breaking: choose a representative of each orbit
%%   of the square-edges of the second box
%%   and a single square-edge of the first box,
%%   and require the base square-edge to be mapped to (at least) one representative
  
\end{itemize}

Unit prop during problem generation:
``literal'' can be (possibly negated) variable or truth value \\
Function to create a logic gate accounts for these options:
can return one of its arguments back, or a truth value \\
Clauses must be either a sequence of (possibly negated) variables
or ``satisfied'' (not added to final problem) \\
Inserting false into a clause does nothing,
inserting true makes it satisfied,
inserting into a satisfied clause does nothing \\

Symmetry breaking: \\
Start with arbitrary polyomino that can fold around all 3 boxes \\
Fix a ``base squedge'' on one of the $1 \times 1$ faces of box 1 \\
Fix a representative of each of the 29 orbits of the squedges of box 2
under rotation \textit{and reflection} \\
Ensure that the base squedge maps to a representative
by rotating/reflecting the net around box 2 \\
Ensure that the mapping preserves orientation
by reflecting the nets around boxes 1 and/or 3 if necessary \\
Fix a representative of each of the 46 orbits of the squedges of box 3
under rotation \\
Ensure that the base squedge maps to a representative
by rotating the net around box 3 \\
Draw a path following the net from the face of box 1 containing the base squedge
to the opposite face (without moving backwards) \\
Ensure that the path (when interpreted as a sequence of moves)
is lexicographically earlier than its reversal
by rotating one $1 \times 1$ face to the other if necessary \\
Ensure that the base squedge of box 1 points towards the opposite face
when the net is unfolded 
by rotating the net around the face


\end{document}
