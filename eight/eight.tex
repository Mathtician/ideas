\documentclass{article}
\usepackage{amsmath}
\usepackage{amssymb}
\usepackage{graphicx}
\usepackage[margin=1in]{geometry}
\usepackage{hyperref}
\usepackage{caption}
\usepackage{float}
\graphicspath{{images/}}
\hypersetup{
    colorlinks=true,
    urlcolor=blue,
}
\begin{document}

\title{Why 8?}
\author{Aresh Pourkavoos}
\maketitle

This leans further toward the garden-variety recreational math
involving specific numbers rather than very general formulas,
and it answers the question: why does
\[\frac{987654321}{123456789} \approx 8.0000000729?\]
Since base 10 is a bit unwieldy to work with in this case,
we will instead note that similar identities seem to hold in other bases.
To avoid confusion, numbers which are not in decimal
will be given a subscript denoting the base,
which is itself written in decimal.
The exception is single-digit numbers, which are unambiguous
and never have subscripts.

Instead of the decimal identity,
we will show its analog in seximal, or base 6:
\[\frac{54321_6}{12345_6} \approx 4.000325_6\]
It might be hard to see what's special
about a number whose digits are increasing (or decreasing),
but they also appear in another place,
namely the repeating decimal (or seximal) expansions of some fractions.
For example,
\[\frac{1}{81} = 0.\overline{012345679}\]
and
\[\frac{1}{41_6} = 0.\overline{01235}_6.\]
Already there are some tenuous links to the original problem:
we see the 8 and the 4 appear on the left-hand side
in both decimal and seximal.
The expansion doesn't look quite right, though:
it skips the corresponding digit.
Also, it raises the further question
of why \textit{this} is even true.
The next thing to see is that both of the denominators are square
and their roots are 1 less than the base:
$81=9^2, 41_6=25=5^2$.

Thus
\[\frac{1}{25}=\left(\frac{1}{5}\right)^2=(0.\overline{1}_6)^2.\]
To understand why $\frac{1}{5}=0.\overline{1}\ldots_6$,
multiply both sides by 5:
$1=0.\overline{5}\ldots_6$,
the seximal version of the identity $1=0.\overline{9}\ldots$.

Given this, we can find the seximal expansion of $(0.\overline{1}_6)^2$
as follows.
$0.\overline{1}_6=0.1_6+0.01_6+0.001_6+\ldots$,
so each term can be multiplied by $0.\overline{1}_6$:

\[0.\overline{1}_6^2 = 0.0\overline{1}_6+0.00\overline{1}_6+0.000\overline{1}_6+\ldots\]

To see what this converges to, we can look at the partial sums:

\begin{align*}
  & 0.0\overline{1}_6 \\
  + 0.00\overline{1}_6 =& 0.01\overline{2}_6 \\
  + 0.000\overline{1}_6 =& 0.012\overline{3}_6 \\
  + 0.0000\overline{1}_6 =& 0.0123\overline{4}_6 \\
  + 0.00000\overline{1}_6 =& 0.01234\overline{5}_6
\end{align*}
At this point, though, something interesting happens.
The repeating 5 is equal to a repeating 0,
and it carries over a 1 to the previous place.
Thus the fifth partial sum is just $0.01235_6$.

\[\frac{(b-1)^3}{b^b-(b^2-b+1)}\]

\[\frac{1}{b^b-(b^2-b+1)} = \sum_{n=0}^\infty \frac{(b^2-b+1)^n}{b^{b(n+1)}} \]

\end{document}
