\documentclass{article}
\usepackage{amsmath}
\usepackage{amssymb}
\usepackage{graphicx}
\usepackage[margin=1in]{geometry}
\usepackage{hyperref}
\usepackage{caption}
\usepackage{float}
\graphicspath{{images/}}
\hypersetup{
  colorlinks=true,
  urlcolor=blue,
}
\begin{document}

\title{Why 8?}
\author{Aresh Pourkavoos}
\maketitle

This leans further toward the garden-variety recreational math
involving specific numbers rather than very general formulas,
and it answers the question: why does
\[\frac{987654321}{123456789} \approx 8.0000000729?\]
Since base 10 is a bit unwieldy to work with in this case,
we will instead note that similar identities seem to hold in other bases.
To avoid confusion, numbers which are not in decimal
will be given a subscript denoting the base,
which is itself written in decimal.
The exception is single-digit numbers, which are unambiguous
and never have subscripts.

Instead of the decimal identity,
we will show its analog in seximal, or base 6:
\[\frac{54321_6}{12345_6} \approx 4.000325_6\]
It might be hard to see what's special
about a number whose digits are increasing (or decreasing),
but they also appear in another place,
namely the repeating decimal (or seximal) expansions of some fractions.
For example,
\begin{align*}
  \frac{1}{81} &= 0.\overline{012345679} \\
  \frac{1}{41_6} &= 0.\overline{01235}_6
\end{align*}
Already there are some tenuous links to the original problem:
we see the 8 and the 4 appear on the left-hand side
in both decimal and seximal.
The expansion doesn't look quite right, though:
it skips a digit as it increases.
Also, it raises the further question
of why \textit{this} is even true.
The next thing to see is that both of the denominators are square
and their roots are 1 less than the base:
$81=9^2, 41_6=25=5^2$.

Thus
\[\frac{1}{25}=\left(\frac{1}{5}\right)^2=(0.\overline{1}_6)^2.\]
To understand why $\frac{1}{5}=0.\overline{1}\ldots_6$,
multiply both sides by 5:
$1=0.\overline{5}\ldots_6$,
the seximal version of the identity $1=0.\overline{9}\ldots$.
Given this, we can find the seximal expansion of $(0.\overline{1}_6)^2$
as follows.
$0.\overline{1}_6=0.1_6+0.01_6+0.001_6+\ldots$,
so each term can be multiplied by $0.\overline{1}_6$:

\[0.\overline{1}_6^2 = 0.0\overline{1}_6+0.00\overline{1}_6+0.000\overline{1}_6+\ldots\]

To see what this converges to, we can look at the partial sums:
\begin{align*}
  & 0.0\overline{1}_6 \\
  + 0.00\overline{1}_6 =& 0.01\overline{2}_6 \\
  + 0.000\overline{1}_6 =& 0.012\overline{3}_6 \\
  + 0.0000\overline{1}_6 =& 0.0123\overline{4}_6 \\
  + 0.00000\overline{1}_6 =& 0.01234\overline{5}_6
\end{align*}
At this point, though, something interesting happens.
since $0.\overline{5}_6=1$,
The repeating 5 carries over a 1 to the previous place,
which happens to have a 4 in it.
Thus the $5^\text{th}$ partial sum is just $0.01235_6$.
In the next term to add,
the first 1 is in the $7^\text{th}$ place after the decimal point,
so it comes 2 places after the 5 and leaves a 0 in between:
\begin{align*}
  & 0.01235_6 \\
  + 0.000000\overline{1}_6 =& 0.012350\overline{1}_6 \\
  + 0.0000000\overline{1}_6 =& 0.0123501\overline{2}_6 \\
  \ldots &
\end{align*}
From here, we can see that the limit of this sum is
$0.\overline{01235}_6$,
where each 5 new terms added turn a 4 into a 5
(effectively skipping the 4)
and resets the repeating portion to 0.

With that out of the way, we can move on to the almost-equation at the beginning.
The increasing sequence $12345_6$ seems related to the repeating expansion $0.\overline{01235_6}$,
but how do we get the decreasing one?
Remembering that $0.\overline{55555}_6=1$,
we can subtract one from the other digit by digit
to get that $0.\overline{55555}_6-0.\overline{01235_6}=0.\overline{54320}_6$,
whose expansion is close to the decreasing sequence $54321_6$.
Since $0.\overline{01235}_6=\frac{1}{41_6}$,
the new number is
\[1-\frac{1}{41_6}=\frac{40_6}{41_6}=\frac{24}{25}.\]
To line up the increasing and decreasing numbers,
we can multiply $0.\overline{01235}_6$ by $10_6=6$
to get
\[0.\overline{12350}_6 = \frac{10_6}{41_6} = \frac{6}{25}.\]

But in order to get to the original almost-equation,
the decimals need to be turned back into integers somehow.
First, we can multiply by $100000_6$
to shift the first cycle of 5 digits to the left of the decimal point:
\[0.\overline{12350}_6 \times 100000_6 = 12350.\overline{12350}_6.\]
Close, but we still have the repeating decimal to deal with.
Fortunately, we can just subtract $0.\overline{12350}_6$ from both sides:
\[0.\overline{12350}_6 \times (100000_6-1) = 12350_6,\]
or in other words,
\[0.\overline{12350}_6 = \frac{12350_6}{55555_6}.\]
Similarly,
\[0.\overline{54320}_6 = \frac{54320_6}{55555_6}.\]
Almost there! All we need to do now is divide one by the other
and cancel out the denominators:
\[\frac{0.\overline{12350}_6}{0.\overline{54320}_6}=
\frac{\frac{12350_6}{55555_6}}{\frac{54320_6}{55555_6}}
=\frac{12350_6}{54320_6},\]
which is very close to the left-hand side of the goal!
We can't expect to get both sides exactly,
since the ``equation'' is not exact,
but we can at least rewrite this fraction as
\[\frac{12345_6+1}{54321_6-1}.\]
On the other hand,
we know how the decimals simplify,
so it is also true that
\[\frac{0.\overline{12350}_6}{0.\overline{54320}_6}
=\frac{\frac{40_6}{41_6}}{\frac{10_6}{41_6}}=\frac{40_6}{10_6}=4.\]
Thus
\[\frac{12345_6+1}{54321_6-1}=4,\]
the exact version of the almost-equation at the beginning.
An analogous argument holds for base 10.

\newpage

But the original ``equation'' isn't what we got at the end:
there is still a small error term.
But what is that term exactly?
The way the decimal and seximal were truncated after $729$ and $325_6$, respectively,
makes it appear that the digits look random after the initial run of 0s.
However, that's not actually the case:

\[\frac{987654321}{123456789} \approx 8.0000000729000006633900060368490549353263999114702391943791766688\ldots\]

After the 729, there are 5 more 0s in a row!
After that, the nonzero string 66339 appears,
followed by 3 more zeros,
after which the digits really do seem chaotic.
But why 729 and 66339?

%% \[\frac{(b-1)^3}{b^b-b^2+b-1}\]

%% \[\frac{1}{b^b-b^2+b-1} = \sum_{n=0}^\infty \frac{(b^2-b+1)^n}{b^{b(n+1)}} \]

\end{document}
