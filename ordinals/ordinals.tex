\documentclass{article}
\usepackage{amsmath}
\usepackage{amssymb}
\usepackage{graphicx}
\usepackage{bm}
\usepackage[margin=1in]{geometry}
\graphicspath{{images/}}

\newcommand{\w}{\omega}
\newcommand{\e}{\varepsilon}

\begin{document}

Infinite lists \\
An ordinal is either 0 or an infinite list of ordinals written right to left,
which is all 0 after some point \\
$\gamma = (\ldots, 0, 0, \gamma_n, \ldots, \gamma_0)$,
so $\gamma_k = 0$ for all $k > n$ \\
For all $i$ (including $n$ and larger),
$\gamma_i < (\ldots, \gamma_{i+3}, \gamma_{i+2}, \gamma_{i+1}+1, 0, \ldots, 0)$ \\
Should match the behavior of finitary Veblen functions
while retaining lexicographical order \\
$(\ldots, 0)=1$, $(\ldots, 0, 1)=2$, $(\ldots, 0, 1, 0)=\w$ \\

Finite lists \\
Leading 0s may be removed from infinite list and length of remaining list put in front:
$(n: \gamma_{n-1}, \ldots, \gamma_0)$ \\
Requires construction of naturals first, unlike infinite lists \\
Still lexicographic since lengths are compared first \\
No need for base case since list may be empty:
could set $(0:)=0$, $(1: (0:))=(1: 0)=1$, $(1: 1)=2$, etc. \\
This would offset naturals by 1 from infinite list representation,
but otherwise identical \\

Ordinal-indexed lists \\
``Redundant'' finite lists: place index of each element before it:
$(n: \gamma_n, \ldots, \gamma_0) \rightarrow (n: \gamma_n, \ldots, 0: \gamma_0)$ \\
``Sparse'' finite lists: remove all 0 entries from list (along with their indices), \\
turn all entries $\alpha \rightarrow -1+\alpha$ to fill gap:
$(2: 1, 1: 0, 0: \w) \rightarrow (2: 0, 0: \w)$ \\
Ordinal-indexed lists: allow indices to be ordinals themselves:
$(\omega: 0) = \sup(\{(0: 0), (1: 0), (2: 0), \ldots\})$ \\
No longer requires naturals to be constructed first \\

Formal definition \\
Ordinal is a finite list of pairs
$(\beta_n: \gamma_n, \beta_{n-1}: \gamma_{n-1}, \ldots, \beta_0: \gamma_0)$,
ordered lexicographically \\
Indices are strictly decreasing: $\beta_n > \beta_{n-1} > \ldots > \beta_0$ \\
For all $i$, if $i < n$ and $\beta_{i+1}=\beta_i+1$, then
$\gamma_i < (\beta_n: \gamma_n, \ldots, \beta_{i+1}: \gamma_{i+1}+1)$ \\
Otherwise, $\gamma_i < (\beta_n: \gamma_n, \ldots, \beta_{i+1}: \gamma_{i+1}, \beta_i+1: 0)$ \\

%% Ternary representation \\
%% ``Tail'' of list may be bundled into another ordinal: \\
%% $(\beta_n: \gamma_n, \beta_{n-1}: \gamma_{n-1}, \ldots, \beta_0: \gamma_0)
%% \rightarrow (\beta, \gamma, \alpha)$ \\
%% $\beta = \beta_n, \gamma = \gamma_n,
%% \alpha = (\beta_{n-1}: \gamma_{n-1}, \ldots, \beta_0: \gamma_0)$ \\
%% Constraints: $\alpha < (\beta, 0, 0)$, $\gamma < ???$

%% Ternary representation \\
%% ``Init'' of list may be bundled into another ordinal: \\
%% $(\beta_n: \gamma_n, \ldots, \beta_{1}: \gamma_{1}, \beta_0: \gamma_0)
%% \rightarrow (\alpha, \beta, \gamma)$ \\
%% $\alpha = (\beta_{n}: \gamma_{n}, \ldots, \beta_1: \gamma_1),
%% \beta = \beta_0, \gamma = \gamma_0$ \\
%% Constraints: ??? \\

Examples \\
\begin{align*}
  () &= 0 \\
  (0: 0) &= 1 \\
  (0: 1) &= 2 \\
  (0: n) &= 1+n, & n < (1: 0) \\
  (1: 0) &= \w \\
  (1: 0, 0: \gamma) &= \w+(1+\gamma), & \gamma < (1: 1) \\
  (1: 1) &= \w^2 \\
  (1: \gamma) &= \w^{1+\gamma}, & \gamma < (2: 0) \\
  (1: \gamma_1, 0: \gamma_0) &= \w^{1+\gamma_1}+(1+\gamma_0),
  & \gamma_1 < (2: 0), \gamma_0 < (1: \gamma_1+1) = \w^{1+\gamma_1+1} \\
  (2: 0) &= \e_0 \\
  (2: 0, 1: \gamma_1) &= \e_0\w^{1+\gamma_1}
  & \gamma_1 < (2: 1) \\
  (2: 0, 1: \gamma_1, 0: \gamma_0) &= \e_0\w^{1+\gamma_1}+(1+\gamma_0)
  & \gamma_1 < (2: 1), \gamma_0 < (2: 0, 1: \gamma_1+1) = \e_0\w^{1+\gamma_1+1} \\
\end{align*}

\end{document}
