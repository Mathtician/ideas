\documentclass{article}
\usepackage{amsmath}
\usepackage{amssymb}
\usepackage{graphicx}
\usepackage{bm}
\usepackage[margin=1in]{geometry}
\graphicspath{{images/}}

\newcommand{\w}{\omega}
\newcommand{\e}{\varepsilon}

\begin{document}

\newcommand{\hopen}[1]{[#1)}
  \newcommand{\indentsarewack}{]}

Infinite lists \\
An ordinal is either 0 or an infinite list of ordinals written right to left,
which is all 0 after some point \\
$\gamma = (\ldots, 0, 0, \gamma_n, \ldots, \gamma_0)$,
so $\gamma_k = 0$ for all $k > n$ \\
For all $i$ (including $n$ and larger),
$\gamma_i < (\ldots, \gamma_{i+3}, \gamma_{i+2}, \gamma_{i+1}+1, 0, \ldots, 0)$ \\

Finite lists \\
Leading 0s may be removed from infinite list and length of remaining list put in front:
$(n: \gamma_{n-1}, \ldots, \gamma_0)$ \\
Requires construction of naturals first, unlike infinite lists \\
Still lexicographic since lengths are compared first \\
No need for base case since list may be empty:
could set $(0:)=0$, $(1: (0:))=(1: 0)=1$, $(1: 1)=2$, etc. \\
This would offset naturals by 1 from infinite list representation,
but otherwise identical \\

Ordinal-indexed lists \\
``Redundant'' finite lists: place index of each element before it:
$(n: \gamma_n, \ldots, \gamma_0) \rightarrow (n: \gamma_n, \ldots, 0: \gamma_0)$ \\
``Sparse'' finite lists: remove all 0 entries from list (along with their indices), \\
turn all entries $\alpha \rightarrow -1+\alpha$ to fill gap:
$(2: 1, 1: 0, 0: \w) \rightarrow (2: 0, 0: \w)$ \\
Ordinal-indexed lists: allow indices to be ordinals themselves:
$(\omega: 0) = \sup(\{(0: 0), (1: 0), (2: 0), \ldots\})$ \\
No longer requires naturals to be constructed first \\

Formal definition \\
Ordinal is a finite list of pairs
$(\beta_n: \gamma_n, \beta_{n-1}: \gamma_{n-1}, \ldots, \beta_0: \gamma_0)$,
ordered lexicographically \\
Indices are strictly decreasing: $\beta_n > \beta_{n-1} > \ldots > \beta_0$ \\
For all $i$, if $i < n$ and $\beta_{i+1}=\beta_i+1$, then
$\gamma_i < (\beta_n: \gamma_n, \ldots, \beta_{i+1}: \gamma_{i+1}+1)$ \\
Otherwise, $\gamma_i < (\beta_n: \gamma_n, \ldots, \beta_{i+1}: \gamma_{i+1}, \beta_i+1: 0)$ \\

%% Ternary representation \\
%% ``Tail'' of list may be bundled into another ordinal: \\
%% $(\beta_n: \gamma_n, \beta_{n-1}: \gamma_{n-1}, \ldots, \beta_0: \gamma_0)
%% \rightarrow (\beta, \gamma, \alpha)$ \\
%% $\beta = \beta_n, \gamma = \gamma_n,
%% \alpha = (\beta_{n-1}: \gamma_{n-1}, \ldots, \beta_0: \gamma_0)$ \\
%% Constraints: $\alpha < (\beta, 0, 0)$, $\gamma < ???$

%% Ternary representation \\
%% ``Init'' of list may be bundled into another ordinal: \\
%% $(\beta_n: \gamma_n, \ldots, \beta_{1}: \gamma_{1}, \beta_0: \gamma_0)
%% \rightarrow (\alpha, \beta, \gamma)$ \\
%% $\alpha = (\beta_{n}: \gamma_{n}, \ldots, \beta_1: \gamma_1),
%% \beta = \beta_0, \gamma = \gamma_0$ \\
%% Constraints: ??? \\

Examples \\
\begin{align*}
  () &= 0 \\
  (0: 0) &= 1 \\
  (0: 1) &= 2 \\
  (0: n) &= 1+n, & n < (1: 0) \\
  (1: 0) &= \w \\
  (1: 0, 0: \gamma) &= \w+(1+\gamma), & \gamma < (1: 1) \\
  (1: 1) &= \w^2 \\
  (1: \gamma) &= \w^{1+\gamma}, & \gamma < (2: 0) \\
  (1: \gamma_1, 0: \gamma_0) &= \w^{1+\gamma_1}+(1+\gamma_0),
  & \gamma_1 < (2: 0), \gamma_0 < (1: \gamma_1+1) = \w^{1+\gamma_1+1} \\
  (2: 0) &= \e_0 \\
  (2: 0, 1: \gamma_1) &= \e_0\w^{1+\gamma_1}
  & \gamma_1 < (2: 1) \\
  (2: 0, 1: \gamma_1, 0: \gamma_0) &= \e_0\w^{1+\gamma_1}+(1+\gamma_0)
  & \gamma_1 < (2: 1), \gamma_0 < (2: 0, 1: \gamma_1+1) = \e_0\w^{1+\gamma_1+1} \\
  (3: 0) &= \zeta_0 \\
  (\beta: \gamma) &= \varphi_{-1+\beta}(\gamma) & \beta \geq 2
\end{align*} \\

Relation to Veblen functions \\
The notation goes up to $\Gamma_0=\varphi(1, 0, 0)$,
since each lower-$\beta$ term essentially applies
another two-variable Veblen function,
except for the last, which is added to the result.
It is closely related to Veblen normal form,
but restricted to two arguments
and preserving lexicographical order.

\newpage

\begin{align*}
  [0, \varepsilon_0) &= \{0\} \cup \bigcup_{\alpha \in [0, \varepsilon_0)}[\omega^\alpha, \omega^{\alpha+1}) \\
  [\omega^\alpha, \omega^{\alpha+1}) &= \{\omega^\alpha+\beta \mid \beta \in [0, \omega^{\alpha+1})\} \\
  [0, \omega^{\alpha+1}) &= \{0\} \cup \bigcup_{\beta \in [0, \alpha+1)}[\omega^\beta, \omega^{\beta+1}) \\
  [0, 0+1) &= \{0\} \\
  [0, \omega^\alpha+\beta+1) &= [0, \omega^\alpha) \cup [\omega^\alpha, \omega^\alpha+\beta+1) \\
  [\omega^\alpha, \omega^\alpha+\beta+1) &= \{\omega^\alpha+\gamma \mid \gamma \in [0, \beta+1)\} \\
  [0, \omega^\alpha) &= \{0\} \cup \bigcup_{\beta \in [0, \alpha)}[\omega^\beta, \omega^{\beta+1}) \\
  [0, 0) &= \varnothing \\
  [0, \omega^\alpha+\beta) &= [0, \omega^\alpha) \cup [\omega^\alpha, \omega^\alpha+\beta) \\
  [\omega^\alpha, \omega^\alpha+\beta) &= \{\omega^\alpha+\gamma \mid \gamma \in [0, \beta)\}
\end{align*}
May cast $\beta$ from $[0, \omega^{\alpha+1})$ to $[0, \varepsilon_0)$
so $[0, \beta+1)$ may be defined
\end{document}
