\documentclass{article}
\usepackage{amsmath}
\usepackage{amssymb}
\usepackage{graphicx}
\usepackage[margin=1in]{geometry}
\usepackage{hyperref}
\usepackage{caption}
\usepackage{float}
\graphicspath{{images/}}
\hypersetup{
    colorlinks=true,
    urlcolor=blue,
}
\begin{document}

\title{Rendering zonotope cross-sections}
\author{Aresh Pourkavoos}
\maketitle

This project is the culmination of a train of thought
that started around a year ago
with the question of whether I could build an icosahedron in Minecraft.
I had learned that the coordinates of its vertices
were based on the golden ratio $\varphi = \frac{1+\sqrt{5}}{2}$,
the positive solution to the equation $\varphi^2 = \varphi+1$.
More specifically,
an icosahedron can be constructed with three perpendicular golden rectangles.

[Image]

If the icosahedron has a side length of 2,
then the rectangles are $2 \times 2\varphi$,
and the coordinates of their vertices are
\[
(0, \pm 1, \pm \varphi),
(\pm 1, \pm \varphi, 0),
(\pm \varphi, 0, \pm 1),
\]
or more succinctly,
all permutations and sign changes of
\[(0, 1, \varphi).\]
Since $\varphi$ is irrational,
there is no way to scale these coordinates
such that they fall exactly on integers.
(For the readers who are considering rotations,
the proof that they don't work either is left as an exercise.)
However, it is possible to approximate it using the Fibonacci numbers,
since they obey the same ``power law'' as $\varphi$,
i.e. $\varphi^{n}+\varphi^{n+1}=\varphi^{n+2}$ and $F_n+F_{n+1}=F_{n+2}$.
For this reason, the ratio of consecutive Fibonacci numbers approaches $\varphi$,
so we may approximate the coordinates of an icosahedron with something like
\[(0, 3, 5).\]
Placed into Minecraft, the vertices look like this:

[Image]

But how do we fill in the edges?

What is a zonotope?

Construction method

C program

The omnitruncated 120-cell

The golden ring

The omnitruncated 24-cell

The $\sqrt{2}$ ring

\end{document}
