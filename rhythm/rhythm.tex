\documentclass{article}
\usepackage{amsmath}
\usepackage{amssymb}
\usepackage{graphicx}
\usepackage[margin=1in]{geometry}
\usepackage{hyperref}
\usepackage{caption}
\usepackage{float}
\graphicspath{{images/}}
\hypersetup{
  colorlinks=true,
  urlcolor=blue,
}
\begin{document}

\title{Irrational Rhythms}
\author{Aresh Pourkavoos}
\maketitle

One of the most fundamental features of Western music is rhythm,
the spacing between notes in time
that come in multiples (or fractions) of a fixed interval.
A whole note divides into two half notes,
which divide into two quarter notes each,
which divide into eighth notes, etc.

\begin{tabular}{|c|c|c|c|c|c|c|c|}
  \hline
  \multicolumn{8}{|c|}{whole} \\ \hline
  \multicolumn{4}{|c|}{half} & \multicolumn{4}{|c|}{half} \\ \hline
  \multicolumn{2}{|c|}{quarter} & \multicolumn{2}{|c|}{quarter} & \multicolumn{2}{|c|}{quarter} & \multicolumn{2}{|c|}{quarter} \\ \hline
  eighth & eighth & eighth & eighth & eighth & eighth & eighth & eighth \\ \hline
\end{tabular}

This binary system carries many interesting properties
that we often take for granted.
For example, it is possible to speed up or slow down a melody by a factor of 2
just by changing the notation and keeping the tempo the same,
so two copies of the same music at different speeds may be placed on the same score.
This allows for so-called canons in augmentation or diminution.
An instant in time may be given a ``strength''
based on how many binary subdivisions it takes to produce it:
the start of a measure is stronger than the center of one,
which is in turn stronger than the beats immediately to its left and right.
(In 4/4 time, theis means that beat 1 is stronger than 3,
which is stronger than 2 and 4).

The binary system of subdivision is often broken on a few specific scales,
namely with time signatures which are not powers of 2
and with tuplets, which can in principle subdivide a time interval
into any number of equally-spaced parts.
However, all of these still fall within the realm of rational numbers:
any piece of music using tuplets is based on some equally-spaced (isochronic) underlying beat
of which all time intervals are a multiple.

$0$: length $1+\sqrt{2}$ (long), $1$: length $1$ (short), $2$: length $\sqrt{2}$ (medium)
\begin{itemize}
\item
  $0 \rightarrow 001$, $1 \rightarrow 0$:
  $1, 0, 001, 0010010, 00100100010010001, \ldots$ (a)
  \begin{itemize}
  \item
    $0 \hookrightarrow 21$:
    $1, 21, 21211, 212112121121, 21211212112121211212112121211, \ldots$ (b) \\
    No rule
  \item
    $0 \rightarrow 12$:
    $1, 12, 12121, 121211212112, 12121121211212121121211212121, \ldots$ (c) \\
    $1 \rightarrow 12$, $2 \rightarrow 121$
  \end{itemize}
\item
  $0 \rightarrow 010$, $1 \rightarrow 0$:
  $1, 0, 010, 0100010, 01000100100100010, \ldots$ (d)
  \begin{itemize}
  \item
    $0 \hookrightarrow 21$:
    $1, 21, 21121, 211212121121, 21121212112121121211212121121, \ldots$ (e) \\
    $1 \rightarrow 21$, $2 \rightarrow 211$
  \item
    $0 \hookrightarrow 12$:
    $1, 12, 12112, 121121212112, 12112121211212112121121212112, \ldots$ (f) \\
    $1 \rightarrow 12$, $2 \rightarrow 112$
  \end{itemize}
\item
  $0 \rightarrow 100$, $1 \rightarrow 0$:
  $1, 0, 100, 0100100, 10001001000100100, \ldots$ (g)
  \begin{itemize}
  \item
    $0 \hookrightarrow 21$:
    $1, 21, 12121, 211212112121, 12121211212112121211212112121, \ldots$ (h) \\
    $1 \rightarrow 21$, $2 \rightarrow 121$
  \item
    $0 \hookrightarrow 12$:
    $1, 12, 11212, 121121211212, 11212121121211212121121211212, \ldots$ (i) \\
    No rule
  \end{itemize}
\item $1 \rightarrow 12, 2 \rightarrow 211$:
  $1, 12, 12211, 122112111212, 12211211121221112121221112211, \ldots$ (j)
\item $1 \rightarrow 21, 2 \rightarrow 112$:
  $1, 21, 11221, 212111211221, 11221112212121112212111211221, \ldots$ (k)
\end{itemize}
Reverses: (a)-(g), (b)-(i), (c)-(h), (d)-(d), (e)-(f), (j)-(k) \\
(g) and (k) oscillate between 2 limit words, all others converge \\
(j) and (k) do not have maximally even spacing (MOS-like), all others do \\
Personal favorite: (e) - derived from palindromic (d), long beat first (swing)

\end{document}
