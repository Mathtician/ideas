\documentclass{article}
\usepackage{amsmath}
\usepackage{amssymb}
\usepackage{graphicx}
\usepackage[margin=1in]{geometry}
\usepackage{hyperref}
\usepackage{caption}
\usepackage{float}
\usepackage{harmony}
\graphicspath{{images/}}
\hypersetup{
  colorlinks=true,
  urlcolor=blue,
}
\begin{document}

\title{Irrational Rhythms}
\author{Aresh Pourkavoos}
\maketitle

One of the most fundamental features of music (particularly in the West) is rhythm,
the spacing between notes in time
that come in multiples (or fractions) of a fixed interval.
A whole note divides into two half notes,
which divide into two quarter notes each,
which divide into eighth notes, etc.

\newcommand{\note}[2]{\multicolumn{#1}{|c|}{#2}}
\newcommand{\fullnote}{\Ganz}
\newcommand{\halfnote}{\Halb}
\newcommand{\quarternote}{\Vier}
\newcommand{\eighthnote}{\Acht}
\newcommand{\sixteenthnote}{\Sech}

\begin{center}
  \begin{tabular}{|c|c|c|c|c|c|c|c|}
    \hline
    \note{8}{\fullnote} \\ \hline
    \note{4}{\halfnote} &
    \note{4}{\halfnote} \\ \hline
    \note{2}{\quarternote} &
    \note{2}{\quarternote} &
    \note{2}{\quarternote} &
    \note{2}{\quarternote} \\ \hline
    \note{1}{\eighthnote} &
    \note{1}{\eighthnote} &
    \note{1}{\eighthnote} &
    \note{1}{\eighthnote} &
    \note{1}{\eighthnote} &
    \note{1}{\eighthnote} &
    \note{1}{\eighthnote} &
    \note{1}{\eighthnote} \\ \hline
  \end{tabular}
\end{center}

This binary system carries many interesting properties
that we often take for granted.
For example, it is possible to double or halve the speed of a melody
just by changing the notation and keeping the tempo the same.
This allows for so-called canons in augmentation or diminution.
where two copies of the same music at different speeds are played at the same time.
An instant in time may be given a ``strength''
based on how many binary subdivisions it takes to produce it:
the start of a measure is stronger than the center of one,
which is in turn stronger than the beats immediately to its left and right,
which are stronger than the times in the center of each beat.
In 4/4 time, this means that beat 1 is stronger than beat 3,
which is stronger than beats 2 and 4,
which are stronger than the ``and''s of each beat.

Not all music follows this binary system exactly.
Many pieces have time signatures other than powers of 2 like 3/4 or 5/4,
and tuplets can in principle subdivide a time interval
into any number of equal parts.
In swing, each beat is subdivided unequally, and the dividing line is not exactly specified:
musicians simply ``feel'' the rhythm and coordinate with each other.
Some music is even written in free time,
in which note lengths in general are left to the performer.

However, all of these fall within one of two categories:
either lengths are constrained to rational numbers (and usually simple ones at that),
or they are not given at all.
But could there be something in between?
A system which has some precise structure to it,
but which is not based on a single metronome?
A few avant-garde musicians have experimented with irrational numbers in rhythm,
notably Conlon Nancarrow, who used them for canons.
His music, which was written for player piano due to human infeasibility,
includes tempo ratios of $\sqrt{2}$, $\frac{e}{\pi}$, and the monstrous
\[\frac{\frac{1}{\sqrt[3]{\pi}}/\sqrt[3]{13/16}}{\frac{1}{\sqrt{\pi}}/\sqrt{2/3}}.\]
Before you rush to simplify the latter,
it comes from his Study No. 41,
which has three movements.
41a has two copies of a melody whose tempo ratio is the numerator of the fraction,
41b is similar but with the denominator,
and 41c consists of 41a and 41b played simultaneously,
indirectly setting up the entire expression as a ratio.
But, I mean, come on.
Is $\pi$ \textit{really} significant to the overall structure of the piece?
Why not just round to 3 or 4 decimal places
(which works out to $\frac{0.7317}{0.6910} \approx 1.0589$)
and call it a day?
Having read some analysis of Nancarrow
and suffered through Study No. 41 myself,
I believe that the exact ratio does not matter,
since the melodies are (by design) either too fast or too slow to feel any sort of ``beat,''
and the canons build and release tension mainly by having the lines drift in and out of sync.
Despite the exactness of their definitions,
Nancarrow's canons are closer to free time than to traditional rhythm.

\begin{center}
  \begin{tabular}{|c|c|c|c|c|c|c|c|}
    \hline
    \note{8}{\fullnote} \\ \hline
    \note{5}{\halfnote} &
    \note{3}{\quarternote} \\ \hline
    \note{3}{\quarternote} &
    \note{2}{\eighthnote} &
    \note{3}{\quarternote} \\ \hline
    \note{2}{\eighthnote} &
    \note{1}{\sixteenthnote} &
    \note{2}{\eighthnote} &
    \note{2}{\eighthnote} &
    \note{1}{\sixteenthnote} \\ \hline
  \end{tabular}
\end{center}

$0$: length $1+\sqrt{2}$ (long), $1$: length $1$ (short), $2$: length $\sqrt{2}$ (medium)
\begin{itemize}
\item
  $0 \rightarrow 001$, $1 \rightarrow 0$:
  $1, 0, 001, 0010010, 00100100010010001, \ldots$ (a)
  \begin{itemize}
  \item
    $0 \hookrightarrow 21$:
    $1, 21, 21211, 212112121121, 21211212112121211212112121211, \ldots$ (b) \\
    No rule
  \item
    $0 \rightarrow 12$:
    $1, 12, 12121, 121211212112, 12121121211212121121211212121, \ldots$ (c) \\
    $1 \rightarrow 12$, $2 \rightarrow 121$
  \end{itemize}
\item
  $0 \rightarrow 010$, $1 \rightarrow 0$:
  $1, 0, 010, 0100010, 01000100100100010, \ldots$ (d)
  \begin{itemize}
  \item
    $0 \hookrightarrow 21$:
    $1, 21, 21121, 211212121121, 21121212112121121211212121121, \ldots$ (e) \\
    $1 \rightarrow 21$, $2 \rightarrow 211$
  \item
    $0 \hookrightarrow 12$:
    $1, 12, 12112, 121121212112, 12112121211212112121121212112, \ldots$ (f) \\
    $1 \rightarrow 12$, $2 \rightarrow 112$
  \end{itemize}
\item
  $0 \rightarrow 100$, $1 \rightarrow 0$:
  $1, 0, 100, 0100100, 10001001000100100, \ldots$ (g)
  \begin{itemize}
  \item
    $0 \hookrightarrow 21$:
    $1, 21, 12121, 211212112121, 12121211212112121211212112121, \ldots$ (h) \\
    $1 \rightarrow 21$, $2 \rightarrow 121$
  \item
    $0 \hookrightarrow 12$:
    $1, 12, 11212, 121121211212, 11212121121211212121121211212, \ldots$ (i) \\
    No rule
  \end{itemize}
\item $1 \rightarrow 12, 2 \rightarrow 211$:
  $1, 12, 12211, 122112111212, 12211211121221112121221112211, \ldots$ (j)
\item $1 \rightarrow 21, 2 \rightarrow 112$:
  $1, 21, 11221, 212111211221, 11221112212121112212111211221, \ldots$ (k)
\end{itemize}
Reverses: (a)-(g), (b)-(i), (c)-(h), (d)-(d), (e)-(f), (j)-(k) \\
(g) and (k) oscillate between 2 limit words, all others converge \\
(j) and (k) do not have maximally even spacing (MOS-like), all others do \\
Personal favorite: (e) - derived from palindromic (d), long beat first (swing)

\end{document}
