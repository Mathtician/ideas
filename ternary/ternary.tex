\documentclass{article}
\usepackage{amsmath}
\usepackage{amssymb}
\usepackage{graphicx}
\usepackage[margin=1in]{geometry}
\usepackage{hyperref}
\usepackage{caption}
\usepackage{float}
\graphicspath{{images/}}
\hypersetup{
  colorlinks=true,
  urlcolor=blue,
}
\begin{document}

\title{A Floating-Point Standard for Balanced Ternary}
\author{Aresh Pourkavoos}
\maketitle

\newcommand{\T}{\mathrm{T}}

Balanced ternary is a base 3 number system,
i.e. numbers in balanced ternary are written as a string of digits
whose contributions to the number are scaled by powers of 3.
However, rather than the usual 0, 1, and 2,
the digits of balanced ternary are $-1$, 0, and 1.
$-1$ is usually written as T, as it looks like a 1 with a bar or negative sign on top.
It can represent negative integers without the use of a negative sign,
e.g. $\T1 = (-1) \times 3 + 1 \times 1= -2$.

In the early days of computers,
this system was considered as an alternative to binary,
and many balanced ternary computers were built,
most notably the Setun in Russia in the early 1960s.
A brief side-by-side comparison follows.
First of all, since there is no real distinction between positive and negative integers,
there is no need for additional circuits to handle each case.
However, the same is also true for all modern binary computers
thanks to the two's complement implementation of signed integers,
which takes advantage of the fact that logic gates on a finite number of bits
behaves like modular arithmetic.
In two's complement, the larger half of what would be unsigned integers,
e.g. 8 through 15 (inclusive) for 4-bit integers,
are reassigned to negative numbers (-8 through -1 in this case).
So in this case, the advantage evens out.

Next, addition and multiplication in balanced ternary
both rely on their respective tables for a single digit.
In binary, adding two bits requires a carryover $\frac{1}{4}$ of the time,
namely when performing $1+1$.
In balanced ternary, however, this occurrs only $\frac{2}{6}$ of the time.


Where I believe balanced ternary shines is in how it handles floating point numbers.


Float $f$ comprised of exponent $e$ (int) and significand $s$ (-1.5 to 1.5) \\
If $s$ starts with 1 or T, $f = 3^e \times s$ (normal) \\
Else if $e=e_{min}$ (all Ts), $f$ is denormal (see formula for normal) \\
Else, $f =$ sNaN, qNaN, or $\pm\infty$ (TBD when operations are implemented)
\setlength{\tabcolsep}{5pt}
\renewcommand{\arraystretch}{1.25}

\noindent
Example with 1 trit for $e$ and 2 trits for $s$: \\
\hspace{-2cm}
\begin{tabular}{*{10}{|c}|}
  \hline
  Memory
  & $\T\T\T$ & $\T\T0$ & $\T\T1$
  & $\T0\T$ & $\T00$ & $\T01$
  & $\T1\T$ & $\T10$ & $\T11$ \\ \hline
  Formula
  & $10^\T \times \T.\T$ & $10^\T \times \T.0$ & $10^\T \times \T.1$
  & $10^\T \times 0.\T$ & $10^\T \times 0.0$ & $10^\T \times 0.1$
  & $10^\T \times 1.\T$ & $10^\T \times 1.0$ & $10^\T \times 1.1$ \\ \hline
  Result
  & $.\T\T$ & $.\T0$ & $.\T1$
  & $.0\T$ & $.00$ & $.01$
  & $.1\T$ & $.10$ & $.11$ \\ \hline
  Decimal
  & $-4/9$ & $-1/3$ & $-2/9$
  & $-1/9$ & $0$ & $1/9$
  & $2/9$ & $1/3$ & $4/9$ \\ \hline
  \hline
  Memory
  & $0\T\T$ & $0\T0$ & $0\T1$
  & $00\T$ & $000$ & $001$
  & $01\T$ & $010$ & $011$ \\ \hline
  Formula
  & $10^0 \times \T.\T$ & $10^0 \times \T.0$ & $10^0 \times \T.1$
  & ? & ? & ?
  & $10^0 \times 1.\T$ & $10^0 \times 1.0$ & $10^0 \times 1.1$ \\ \hline
  Result
  & $\T.\T$ & $\T.0$ & $\T.1$
  & ? & ? & ?
  & $1.\T$ & $1.0$ & $1.1$ \\ \hline
  Decimal
  & $-4/3$ & $-1$ & $-2/3$
  & ? & ? & ?
  & $2/3$ & $1$ & $4/3$ \\ \hline
  \hline
  Memory
  & $1\T\T$ & $1\T0$ & $1\T1$
  & $10\T$ & $100$ & $101$
  & $11\T$ & $110$ & $111$ \\ \hline
  Formula
  & $10^1 \times \T.\T$ & $10^1 \times \T.0$ & $10^1 \times \T.1$
  & ? & ? & ?
  & $10^1 \times 1.\T$ & $10^1 \times 1.0$ & $10^1 \times 1.1$ \\ \hline
  Result
  & $\T\T.$ & $\T0.$ & $\T1.$
  & ? & ? & ?
  & $1\T.$ & $10.$ & $11.$ \\ \hline
  Decimal
  & $-4$ & $-3$ & $-2$
  & ? & ? & ?
  & $2$ & $3$ & $4$ \\ \hline
\end{tabular}
\end{document}
