\documentclass{article}
\usepackage[margin=1in]{geometry}
\usepackage{amsmath}
\usepackage{amssymb}
\usepackage{graphicx}
\graphicspath{images/}

\begin{document}

\title{Solving Minesweeper}
\author{Aresh Pourkavoos}
\maketitle

The ``Minesweeper problem'' consists of a set $S$,
a family of sets $\mathcal{F} \subseteq \mathcal{P}(S)$,
and a function $g : \mathcal{F} \rightarrow \mathbb{N}$.
$S$ is the set of cells on the board,
and $\mathcal{F}$ contains the set of neighbors of each square.
For all $A \in \mathcal{F}$, $g(A)=n$ is the number of elements of $A$
(which are squares) that contain mines,
so $0 \leq n \leq \vert A \vert$.

Perform all possible intersections of elements of $\mathcal{F}$ \\
Assign variable to each intersection representing number of mines \\
Sums of these variables corresponding to a single $A \in \mathcal{F}$ are known \\
Variables themselves are only given ranges \\
Need to find a variable with value 0 (if one exists)
so its cells can be revealed

\end{document}
