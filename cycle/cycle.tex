\documentclass{article}
\usepackage[margin=1in]{geometry}
\usepackage{amsmath}
\usepackage{amssymb}
\usepackage{graphicx}
\usepackage{tensor}
\usepackage{enumitem}
\graphicspath{images/}

\newcommand{\ternary}[4]{\hspace{-1mm}\tensor[_{#3}]{\overset{#2}{#1}}{_{#4}}}
\newcommand{\cyc}[3]{\ternary{\circlearrowleft}{#1}{#2}{#3}}
\newcommand{\po}[1]{\overset{#1}{\leq}}

\begin{document}

\title{Cycle Relations}
\author{Aresh Pourkavoos}
\maketitle

\textit{Problem.} Let $S$ be a set.
A ternary relation $\circlearrowleft$ on $S$ is a \textit{cycle relation} when:
\begin{enumerate}[label=(\arabic*)]
\item
  For all $a \in S$, $\cyc{a}{a}{a}$.
\item
  For all $a, b, c \in S$, $\cyc{a}{b}{c}$ implies $\cyc{c}{a}{b}$.
\item
  For all $a \in S$, if we define the relation $\po{a}$ on $S \setminus \{a\}$
  so that $b \po{a} c$ when $\cyc{a}{b}{c}$,
  then $\po{a}$ is a partial order.
\end{enumerate}
Let $z \in S$, and let $\preceq$ be a partial order on $S \setminus \{z\}$.
\begin{enumerate}[label=(\roman*)]
\item
  Does there exist a cycle relation $\circlearrowleft$ on $S$
  such that $\po{z}$ as defined above is equal to $\preceq$?
\item
  Assuming that such a $\circlearrowleft$ exists, is it unique?
\end{enumerate}

\newpage

\textit{Solution.}
\begin{enumerate}[label=(\roman*)]
\item
  We show that a cycle relation exists.
  Define $\circlearrowleft$ as follows:
  \begin{enumerate}[label=(\Alph*)]
  \item
    For all $a, b \in S$,
    $\cyc{a}{b}{b}$.
  \item
    For all $w, x \in S \setminus \{z\}$
    where $w \preceq x$, $\cyc{z}{w}{x}$.
  \item
    For all $w, x, y \in S \setminus \{z\}$
    where $w \preceq x \preceq y$, $\cyc{w}{x}{y}$.
  \item
    All cycles of the above satisfy $\circlearrowleft$,
    e.g. $\cyc{x}{z}{w}$ and $\cyc{w}{x}{z}$ where $w \preceq x$,
    and no other triples do.
  \end{enumerate}
  By (A) with $b = a$, $\circlearrowleft$ satisfies property (1),
  and by (D), it also satisfies (2).

  To show (3), we find the relations $\po{a}$ for $a \in S$.
  In the first case, $a = z$.
  Neither (A) nor (C) contributes to $\po{z}$,
  even when cycled by (D).
  However, by (B), for all $w, x \in S \setminus \{z\}$
  where $w \preceq x$, $w \po{z} x$,
  and the cycles of (B) again contribute nothing.
  Thus $\po{z} = \preceq$, so $\po{z}$ is a partial order
  (and has the desired agreement with $\preceq$).

  In the second case, let $w \in S \setminus \{z\}$.
  Then $\po{w}$ is defined as follows:
  \begin{enumerate}[label=(\Roman*)]
  \item
    By (A) with $a = w$, for all $b \in S \setminus \{w\}$, $b \po{w} b$.
  \item
    By (B), for all $x \in S \setminus \{w, z\}$,
    if $w \preceq x$, then $\cyc{z}{w}{x}$,
    so $\cyc{w}{x}{z}$, so $x \po{w} z$. \\
    Also, if $x \preceq w$, then $\cyc{z}{x}{w}$,
    so $\cyc{w}{z}{x}$, so $z \po{w} x$.
  \item
    By (C), for all $x, y \in S \setminus \{w, z\}$,
    if $w \preceq x \preceq y$, then $\cyc{w}{x}{y}$,
    so $x \po{w} y$. \\
    Also, if $y \preceq w \preceq x$, then $\cyc{y}{w}{x}$,
    so $\cyc{w}{x}{y}$, so $x \po{w} y$. \\
    Finally, if $x \preceq y \preceq w$, then $\cyc{x}{y}{w}$,
    so $\cyc{w}{x}{y}$, so $x \po{w} y$.
  \end{enumerate}
  We show that $\po{w}$ is a partial order.
  By (I), $\po{w}$ is reflexive.
  To show that it is antisymmetric,
  let $a, b \in S \setminus \{w\}$,
  and suppose $a \po{w} b$ and $b \po{w} a$.
  We case on the values of $a$ and $b$.
  \begin{itemize}
  \item
    If both are $z$, $a = b$ trivially.
  \item
    If one is $z$ and the other is not,
    this contradicts the goal $a = b$,
    so we want to show that this case cannot occur.
    WLOG $a = z$ and $b \in S \setminus \{w, z\}$,
    so $z \po{w} b$ and $b \po{w} z$.
    This must arise from (II),
    so $b \preceq w$ and $w \preceq b$, respectively.
    Since $\preceq$ is antisymmetric, $b = w$,
    which contradicts $b \in S \setminus \{w, z\}$.
  \item
    If neither is $z$,
    i.e. $a, b \in S \setminus \{w, z\}$,
    then $a \po{w} b$ and $b \po{w} a$
    must arise from either (I),
    which would immediately imply that $a = b$,
    or (III).
    For the latter, since $a \po{w} b$,
    one of the following is true:
    \begin{itemize}
    \item $w \preceq a \preceq b$
    \item $b \preceq w \preceq a$
    \item $a \preceq b \preceq w$
    \end{itemize}
    Likewise, since $b \po{w} a$,
    one of the following is true:
    \begin{itemize}
    \item $w \preceq b \preceq a$
    \item $a \preceq w \preceq b$
    \item $b \preceq a \preceq w$
    \end{itemize}
    From here, it may be verified
    that any of the $3 \times 3 = 9$ cases
    implies one of the following:
    \begin{itemize}
    \item
      $a \preceq b$ and $b \preceq a$,
      so $a = b$ because $\preceq$ is antisymmetric.
    \item
      $a \preceq w$ and $w \preceq a$,
      so $a = w$, which contradicts $a \in S \setminus \{w, z\}$.
    \item
      $b \preceq w$ and $w \preceq b$,
      so $b = w$, which contradicts $b \in S \setminus \{w, z\}$.
    \end{itemize}
  \end{itemize}

  It remains to show that $\po{w}$ is transitive.
  Let $a, b, c \in S \setminus \{w\}$,
  and suppose $a \po{w} b$ and $b \po{w} c$.
  If any pair of $a$, $b$, and $c$ are equal,
  then $a \po{w} c$ follows immediately,
  so we assume from here that all three are distinct.
  Then at most one of them can be $z$.
  \begin{itemize}
  \item
    If $a = z$, then $b, c \in S \setminus \{w, z\}$.
    Also, $z = a \po{w} b$ must arise from (II),
    so $b \preceq w$,
    and $b \po{w} c$ must arise from (III),
    so one of the following is true:
    \begin{itemize}
    \item $w \preceq b \preceq c$
    \item $c \preceq w \preceq b$
    \item $b \preceq c \preceq w$
    \end{itemize}
    In all cases, $w \preceq b$ or $c \preceq w$.
    If $w \preceq b$, then $b = w$,
    which contradicts $b \in S \setminus \{w, z\}$.
    If $c \preceq w$, then by (II), $z \po{w} c$,
    i.e. $a \po{w} c$.
  \item
    If $b = z$, then $a, c \in S \setminus \{w, z\}$.
    Also, $a \po{w} b = z$ must arise from (II),
    so $w \preceq a$,
    and $z = b \po{w} c$ must arise from (II),
    so $c \preceq w$.
    Thus $c \preceq w \preceq a$,
    so by (III), $a \po{w} c$.
  \item
    If $c = z$, then $a, b \in S \setminus \{w, z\}$.
    Also, $b \po{w} c = z$ must arise from (II),
    so $w \preceq b$,
    and $a \po{w} b$ must arise from (III),
    so one of the following is true:
    \begin{itemize}
    \item $w \preceq a \preceq b$
    \item $b \preceq w \preceq a$
    \item $a \preceq b \preceq w$
    \end{itemize}
    In all cases, $b \preceq w$ or $w \preceq a$.
    If $b \preceq w$, then $b = w$,
    which contradicts $b \in S \setminus \{w, z\}$.
    If $w \preceq a$, then by (II), $a \po{w} z$,
    i.e. $a \po{w} c$.
  \item
    If none of them is $z$, then $a, b, c \in S \setminus \{w, z\}$.
    $a \po{w} b$ must arise from (III),
    so one of the following is true:
    \begin{itemize}
    \item $w \preceq a \preceq b$
    \item $b \preceq w \preceq a$
    \item $a \preceq b \preceq w$
    \end{itemize}
    $b \po{w} c$ must also arise from (III),
    so one of the following is true:
    \begin{itemize}
    \item $w \preceq b \preceq c$
    \item $c \preceq w \preceq b$
    \item $b \preceq c \preceq w$
    \end{itemize}
    From here, it may be verified
    that any of the $3 \times 3 = 9$ cases
    implies one of the following:
    \begin{itemize}
    \item
      $b \preceq w$ and $w \preceq b$,
      so $b = w$, contradicting $b \in S \setminus \{w, z\}$.
    \item $w \preceq a \preceq c$, so by (III), $a \po{w} c$.
    \item $c \preceq w \preceq a$, so by (III), $a \po{w} c$.
    \item $a \preceq c \preceq w$, so by (III), $a \po{w} c$.
    \end{itemize}
  \end{itemize}
  Since $\po{w}$ is transitive,
  $\po{w}$ is a partial order on $S \setminus \{w\}$,
  so $\circlearrowleft$ is a cycle relation.
  
  \newpage

\item
  We show that the cycle relation is not unique if it exists.
  Let $S = \{w, x, y, z\}$,
  and let $\preceq = \{(w, w), (x, x), (y, y)\}$.
  Define $\circlearrowleft$ as follows:
  \begin{enumerate}[label=(\Alph*)]
  \item
    For all $a, b \in S$, $\cyc{a}{b}{b}$.
  \item
    $\cyc{w}{x}{y}$.
  \item
    All cycles of the above satisfy $\circlearrowleft$,
    and no other triples do.
  \end{enumerate}
  Then by (A) with $a = b$, $\circlearrowleft$ satisfies property (1),
  and by (C), it also satisfies (2).
  To show (3), we find the relations $\po{a}$ for $a \in S$:
  \begin{itemize}
  \item $\po{w} = \{(x, x), (y, y), (z, z), (x, y)\}$
  \item $\po{x} = \{(w, w), (y, y), (z, z), (y, w)\}$
  \item $\po{y} = \{(w, w), (x, x), (z, z), (w, x)\}$
  \item $\po{z} = \{(w, w), (x, x), (y, y)\}$
  \end{itemize}
  All of these are partial orders, so (3) is satisfied.
  Thus $\circlearrowleft$ is a cycle relation.
  Also, $\po{z} = \preceq$.

  \newcommand{\backcyc}[3]{\ternary{\circlearrowright}{#1}{#2}{#3}}
  \newcommand{\notbackcyc}[3]{\ternary{\not\circlearrowright}{#1}{#2}{#3}}
  \newcommand{\backpo}[1]{\overset{#1}{\geq}}

  Next, define $\circlearrowright$ (with the direction of the arrow reversed)
  so that $\backcyc{a}{b}{c}$ when $\cyc{a}{c}{b}$.
  Since $\circlearrowleft$ satisfies (1) and (2), so does $\circlearrowright$.
  Also, each relation $\backpo{a}$ for $\circlearrowright$
  is the reverse of the corresponding $\po{a}$ for $\circlearrowleft$.
  Since the reverse of a partial order is also a partial order,
  $\circlearrowright$ is a cycle relation.
  Also, $\backpo{z} = \po{z} = \preceq$.

  Finally, $\cyc{w}{x}{y}$, but $\notbackcyc{w}{x}{y}$,
  so the cycle relation is not unique.
\end{enumerate}
%% First, we prove uniqueness
%% by attempting to define $\circlearrowleft$
%% by cases on the number of arguments equal to $z$.
%% \begin{itemize}
%% \item
%%   By 1, $\cyc{z}{z}{z}$.
%% \item
%%   Let $a \neq z$.
%%   By 3, $\cyc{a}{z}{z}$ if and only if $z \po{a} z$.
%%   Since partial orders are reflexive, $z \po{a} z$,
%%   so $\cyc{a}{z}{z}$.
%%   By 2, $\cyc{z}{a}{z}$ and $\cyc{z}{z}{a}$.
%% \item
%%   Let $a, b \neq z$.
%%   By 3, $\cyc{z}{a}{b}$ if and only if $a \po{z} b$.
%%   By 2, $\cyc{b}{z}{a}$ and $\cyc{a}{b}{z}$ under the same conditions
%%   (as 2 may be strengthened to a biimplication).
%% \item
%%   Let $a, b, c \neq z$.
%%   We find $\cyc{a}{b}{c}$
%%   by cases on the ordering of $a$, $b$, and $c$ under $\po{z}$.
%%   \begin{itemize}
%%   \item
%%     If $c \po{z} a \po{z} b$,
%%     then $\cyc{z}{c}{a}$ and $\cyc{z}{a}{b}$.
%%     By 2, $\cyc{a}{z}{c}$ and $\cyc{a}{b}{z}$,
%%     so by 3, $z \po{a} c$ and $b \po{a} z$.
%%     Since partial orders are transitive,
%%     $b \po{a} c$,
%%     so by 3, $\cyc{a}{b}{c}$.
%%   \item
%%     If $a \po{z} b \po{z} c$,
%%     then by the previous case, $\cyc{b}{c}{a}$.
%%     By 2, $\cyc{a}{b}{c}$.
%%   \item
%%     If 
%%   \end{itemize}
%% \end{itemize}

\end{document}
