\documentclass{article}
\usepackage[margin=1in]{geometry}
\usepackage{amsmath}
\usepackage{amssymb}
\usepackage{amsthm}
\usepackage{graphicx}
\usepackage{multicol}
\setlength{\columnseprule}{0.4pt}
\graphicspath{images/}

\newtheorem*{theorem*}{Theorem}

\begin{document}

\title{SPOTD 18}
\author{Aresh Pourkavoos}
\maketitle
\begin{multicols}{2}
\begin{theorem*}
  For $x \in (0, 1)$, define
  \[f(x) := \sum_{i=0}^\infty (-1)^i x^{2^i}.\]
  Then $\lim\limits_{x \rightarrow 1^-} f(x)$ does not exist.
\end{theorem*}

\begin{proof}
  First, for $n \in \mathbb{N}$, define
  \[x_n := 2^{-2^{-n}}.\]
  Then $x_n < (0, 1)$,
  so we may also define $y_n := f(x_n)$.
  Also, $\lim\limits_{n \rightarrow \infty} x_n = 1$,
  so it suffices to show that $\lim\limits_{n \rightarrow \infty} y_n$ does not exist.
  Note
  \begin{align*}
    y_n &= \sum_{i=0}^\infty (-1)^i x_n^{2^i} \\
    &= \sum_{i=0}^\infty (-1)^i \left(2^{-2^{-n}}\right)^{2^i} \\
    &= \sum_{i=0}^\infty (-1)^i 2^{(-2^{-n})(2^i)} \\
    &= \sum_{i=0}^\infty (-1)^i 2^{-2^{i-n}} \\
    &= \sum_{i=-n}^\infty (-1)^{i+n} 2^{-2^i}.
  \end{align*}
  It suffices to show that $\lim\limits_{k \rightarrow \infty} y_{2k} < \frac{1}{2} < \lim\limits_{k \rightarrow \infty} y_{2k+1}$ (if both exist).
  \vfill\null
  By the previous,
  \begin{align*}
    y_{2k} &= \sum_{i=-2k}^\infty (-1)^i 2^{-2^i} \\
    &= \sum_{i=-2k}^{-1} (-1)^i 2^{-2^i} + \sum_{i=0}^\infty (-1)^i 2^{-2^i} \\
    &= \sum_{i=1}^{2k} (-1)^i 2^{-2^{-i}} + y_0 \\
    &= \sum_{i=1}^{2k} (-1)^i x_i + y_0 \\
    &= \sum_{i=1}^{k} (x_{2i}-x_{2i-1}) + y_0.
  \end{align*}
  Similarly,
  \begin{align*}
    y_{2k+1} &= \sum_{i=-(2k+1)}^\infty (-1)^{i+1} 2^{-2^i} \\
    &= \sum_{i=-(2k+1)}^{-2} (-1)^{i+1} 2^{-2^i} + \sum_{i=-1}^\infty (-1)^{i+1} 2^{-2^i} \\
    &= \sum_{i=2}^{2k+1} (-1)^{i+1} 2^{-2^{-i}} + y_1 \\
    &= \sum_{i=2}^{2k+1} (-1)^{i+1} x_i + y_1 \\
    &= \sum_{i=1}^{k} (x_{2i+1}-x_{2i}) + y_1. \\
  \end{align*}
  Since $x_{n+1} - x_n > 0$ for all $n$,
  $(y_{2k})_k$ and $(y_{2k+1})_k$ are both (strictly) increasing.
  \newpage
  First, we show that $\lim\limits_{k \rightarrow \infty} y_{2k+1} > \frac{1}{2}$.
  Since the sequence is increasing, it suffices to find such a $k$.
  (In fact, this alone shows that $\lim\limits_{x \rightarrow 1^-} f(x) \neq \frac{1}{2}$,
  per the original problem.)
  
  Take $k = 2$, so $y_5 = y_{2k+1} = $
  
  \vfill\null
\end{proof}

\end{multicols}
\end{document}
