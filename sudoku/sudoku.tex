\documentclass{article}
\usepackage{amsmath}
\usepackage{amssymb}
\usepackage{graphicx}
\usepackage[margin=1in]{geometry}
\usepackage{hyperref}
\usepackage{caption}
\usepackage{float}
\graphicspath{{images/}}
\hypersetup{
  colorlinks=true,
  urlcolor=blue,
}
\begin{document}

\title{SAT Solving Sudoku}
\author{Aresh Pourkavoos}
\maketitle

Sudoku is a puzzle game which consists of a $9 \times 9$ square grid,
subdivided into $3 \times 3$ blocks
and partially filled with the digits 1-9.
The goal is to fill in the rest of the digits such that
every row, column and block contains each digit exactly once.
Sudoku is a well-studied problem,
and there are many algorithms out there to solve it,
from pencil-and-paper tricks used mostly by human players
to trial-and-error methods used by computers, such as backtracking.
However, I will focus on one particular approach: SAT solving.

SAT is short for ``satisfiability,''
and a SAT solver is a program that checks
whether a given statement is satisfiable.
In this case, a statement is a logical proposition
such as ``P and Q'',
and a statement is satisfiable
if there is some way to assign each variable
(P and Q, in this case) to either true or false
such that the entire statement is true.
``P and Q,'' for example, is satisfiable:
it has exactly one solution, where P and Q are both true.
``P and not P,'' on the other hand, is not satisfiable,
since for both possible values of P, the statement is false.
A brute-force SAT solver would simply check every possible combination,
of which there are $2^n$, where $n$ is the number of variables.
However, there are much more optimized algorithms in use,
making SAT solvers practical for huge formulas with thousands of variables or more.

Usually, the logical statements that SAT solvers accept
are in a restricted format known as
conjunctive normal form, or CNF.
A CNF statement consists of clauses joined by $\land$,
where each clause consists of terms joined by $\lor$,
where each term is either a variable or its negation.
For example,
\[(p_1 \lor \lnot p_2) \land (p_2 \lor p_3)\]
is in CNF, but
\[\lnot (p_1 \lor \lnot p_2) \land (p_2 \lor p_3)\]
is not, since CNF does not allow the clause
$(p_1 \lor p_2)$ to be negated.
However, it is possible to turn this into CNF by using De Morgan's laws:
\[\lnot p_1 \land p_2 \land (p_2 \lor p_3)\]
Now there are three clauses to satisfy:
$\lnot p_1$, $p_2$, and $p_2 \lor p_3$.

This article will consist of two main sections:
translating Sudoku into a form a SAT solver can recognize,
and how the solver actually finds the answer.
However, it's not immediately obvious
how Sudoku could be turned into a SAT problem,
since each cell is filled with one of 9 possible digits,
not one of 2 truth values.
We can represent a choice of 9 by having 9 variables for each cell,
one for each possible digit,
but this creates another problem.
There are 512 possible assignments to these variables,
which include all of them being false, i.e. leaving a cell empty,
and more than one being true, i.e. placing two or more digits in the same cell.
We need to constrain these variables (in typical SAT solver fashion)
to ensure that exactly one variable is true.
There are two parts to this problem: forbidding empty cells
and preventing two digits in the same cell.
The former can be expressed as an or statment containing the given variables:
\[p_1 \lor p_2 \lor \ldots \lor p_n\]
The latter looks at all pairs of variables,
saying that they can't both be true:
\[\lnot (p_1 \land p_2) \land \ldots \land \lnot (p_1 \land p_n)
\land \lnot (p_2 \land p_3) \land \ldots \land \lnot (p_2 \land p_n) \land \ldots
\land \lnot (p_{n-1} \land p_n)\]
For $n$ variables, there are $\binom{n}{2} = \frac{n(n-1)}{2}$ pairs.
For a given Sudoku cell, where there are 9 variables,
it takes $\binom{9}{2}$ of them
to say that at most one variable may be true,
and the or clause containing all 9 variables makes 37.
Since there are 81 cells,
it takes $37 \times 81 = 2997$ formulas
just to say that every cell is filled with exactly one digit.
To express the rest of the puzzle,
similar constraints must be applied to other sets of 9 variables:
\begin{itemize}
\item
  Every row of cells contains every digit exactly once,
  i.e. for every row and digit,
  exactly 1 of the 9 variables that represent
  filling a cell within the given row with a given digit
  must be true.
\item
  Every column contains every digit exactly once,
  i.e. for every column and digit,
  exactly 1 of the 9 variables that represent
  filling a cell within the given column with a given digit
  must be true.
\item
  Every 3 $\times$ 3 block contains every digit exactly once,
  i.e. for every block and digit,
  exactly 1 of the 9 variables that represent
  filling a cell within the given block with a given digit
  must be true.
\end{itemize}
Each of these 3 items places constraints on 81 sets of 9 variables each,
just like the requirement that every cell contains exactly one digit.
Thus each one adds another 2997 formulas
for a total of $2997 \times 4 = 11988$.

Strictly speaking, these variables and formulas are all that are necessary
to define the constraints of a Sudoku puzzle
formatted in the way described.
But is it the only way, or even the best way?
In SAT solving, a trick that is often used to speed up computation
at the cost of increasing memory use
is to introduce \textit{auxiliary variables},
which are defined to be equivalent to some expression of the original input.
Among other things,
their efficiency can come from reducing the number of formulas,
as we will do with Sudoku.

At the moment, each set of 9 variables requires 37 formulas
to say that exactly one of them must be true.
We will lower that number to 25
by introducing 3 auxiliary variables to each set,
each representing the fact that
one of a set of 3 of the 9 is true.
In other words, given $p_1$ through $p_9$,
we define
\begin{align*}
  q_1 &\leftrightarrow p_1 \lor p_2 \lor p_3, \\
  q_2 &\leftrightarrow p_4 \lor p_5 \lor p_6, \\
  q_3 &\leftrightarrow p_7 \lor p_8 \lor p_9.
\end{align*}
However, $\leftrightarrow$ is not an operation recognized by most SAT solvers.
Instead, the biimplication must be broken up:
\begin{align*}
  q_1 &\rightarrow (p_1 \lor p_2 \lor p_3), \\
  (p_1 \lor p_2 \lor p_3) &\rightarrow q_1.
\end{align*}

We can turn the first implication into CNF
by using the equivalence
\[(a \rightarrow b) \cong (\lnot a \lor b),\]
turning it into
\[\lnot q_1 \lor p_1 \lor p_2 \lor p_3.\]
Applying this to the reverse implication, on the other hand, returns
\[\lnot (p_1 \lor p_2 \lor p_3) \lor q_1,\]
which is still not in CNF.
De Morgan's laws yield
\[(\lnot p_1 \land \lnot p_2 \land \lnot p_3) \lor q_1,\]
and the inner expression may be distributed:
\[(\lnot p_1 \lor q_1) \land (\lnot p_2 \lor q_1) \land (\lnot p_3 \lor q_1).\]
This is a CNF expression consisting of 3 clauses.
Note that this expression is also equivalent to
\[(p_1 \rightarrow q_1) \land (p_1 \rightarrow q_1) \land (p_1 \rightarrow q_1),\]
which makes sense intuitively:
any of $p_1$, $p_2$, and $p_3$ is enough to imply $q_1$ on its own.
Thus the ``definition'' of $q_1$ takes 4 clauses,
and so do those of $q_2$ and $q_3$,
for a total of 12.

From there, the fact that at least one $p$ is true
may be written as $q_1 \lor q_2 \lor q_3$,
which is 1 clause.
The savings come from the statement that
at most one $p$ is true,
which is broken into two main parts.
The first of these is that
at most one $q$ can be true,
since otherwise,
there would be two $p$s ``contained in'' different $q$s which are both true.
Thus
\begin{align*}
  \lnot (q_1 \land q_2), \\
  \lnot (q_1 \land q_3), \\
  \lnot (q_2 \land q_3).
\end{align*}
The second is that ``within'' each $q$,
at most one $p$ is true.
For example, for $q_2$, the statmements would be
\begin{align*}
  \lnot (p_4 \land p_5), \\
  \lnot (p_4 \land p_6), \\
  \lnot (p_5 \land p_7).
\end{align*}
Similar statements may be written for $p_1$ through $p_3$
and for $p_7$ through $p_9$,
bringing the total to 12.

%% For example, we might define an auxiliary variable $q_1$
%% to be equivalent to $p_1 \lor p_2$,
%% since it might be possible in some cases
%% to infer that one of them is true, but not which.
%% Without $q_1$, a naive solver would be stuck
%% and have to guess that $p_1$ is true before proceeding,
%% possibly backtracking later to check if $p_2$ is true.
%% On the other hand, $q_1$ might be used to infer facts about other variables
%% which are true as long as $p_1 \lor p_2$ is true,
%% regardless of which.

%% The following table shows a way to formulate the statement
%% that at most one of 8 variables $a_0, \ldots, a_7$ is true,
%% using a binary divide-and-conquer method.
%% \begin{align*}
%%   && a_0, a_1 \text{ free}|
%%   && a_2, a_3 \text{ free}|
%%   && a_4, a_5 \text{ free}
%%   && a_6, a_7 \text{ free}| \\
%%   && \underline{\lnot a_0 \lor \lnot a_1|}
%%   && \lnot a_2 \lor \lnot a_3|
%%   && \lnot a_4 \lor \lnot a_5
%%   && \lnot a_6 \lor \lnot a_7| \\
%%   && a_{0-1} := a_0 \lor a_1
%%   && a_{2-3} := a_2 \lor a_3|
%%   && a_{4-5} := a_4 \lor a_5
%%   && a_{6-7} := a_6 \lor a_7| \\
%%   && \lnot a_{0-1} \lor a_0 \lor a_1
%%   && \lnot a_{2-3} \lor a_2 \lor a_3|
%%   && \lnot a_{4-5} \lor a_4 \lor a_5
%%   && \lnot a_{6-7} \lor a_6 \lor a_7| \\
%%   && \lnot a_0 \lor a_{0-1}
%%   && \lnot a_2 \lor a_{2-3}|
%%   && \lnot a_4 \lor a_{4-5}
%%   && \lnot a_6 \lor a_{6-7}| \\
%%   && \lnot a_1 \lor a_{0-1}
%%   && \lnot a_3 \lor a_{2-3}|
%%   && \lnot a_5 \lor a_{4-5}
%%   && \lnot a_7 \lor a_{6-7}| \\
%%   && && \underline{\lnot a_{0-1} \lor \lnot  a_{2-3}|}
%%   && && \lnot a_{4-5} \lor \lnot  a_{6-7}| \\
%%   && && a_{0-3} := a_{0-1} \lor a_{2-3}
%%   && && a_{4-7} := a_{4-5} \lor a_{6-7}| \\
%%   && && \lnot a_{0-3} \lor a_{0-1} \lor a_{2-3}
%%   && && \lnot a_{4-7} \lor a_{4-5} \lor a_{6-7}| \\
%%   && && \lnot a_{0-1} \lor a_{0-3}
%%   && && \lnot a_{4-5} \lor a_{4-7}| \\
%%   && && \lnot a_{2-3} \lor a_{0-3}
%%   && && \lnot a_{6-7} \lor a_{4-7}| \\
%%   && && && && \underline{\lnot a_{0-3} \lor \lnot a_{4-7}|}
%% \end{align*}
%% \begin{center}
%%   \begin{tabular}{|c|c|c|c|}
%%     \hline
%%     $i$ & Free vars & Aux vars & Formulas \\ \hline
%%     1 & 2 & 0 & 1 \\ \hline
%%     2 & 4 & 2 & 9 \\ \hline
%%     3 & 8 & 6 & 25 \\ \hline
%%   \end{tabular}
%% \end{center}
%% At level $i$ of the tree,
%% there are $n=2^i$ free variables,
%% $n-2$ auxiliary variables
%% (for a total of $2n-2$ variables),
%% and $4n-7$ formulas.
%% In contrast,
%% the naive method with no auxiliary variables
%% uses $1+(n^2-n)/2$ formulas.
\begin{align*}
  && \lnot p_1 \lor \lnot p_2
  && \lnot p_4 \lor \lnot p_5
  && \lnot p_7 \lor \lnot p_8 \\
  && \lnot p_1 \lor \lnot p_3
  && \lnot p_4 \lor \lnot p_6
  && \lnot p_7 \lor \lnot p_9 \\
  && \lnot p_2 \lor \lnot p_3
  && \lnot p_5 \lor \lnot p_6
  && \lnot p_8 \lor \lnot p_9 \\
  && \lnot q_1 \lor p_1 \lor p_2 \lor p_3
  && \lnot q_2 \lor p_4 \lor p_5 \lor p_6
  && \lnot q_3 \lor p_7 \lor p_8 \lor p_9 \\
  && \lnot p_1 \lor q_1
  && \lnot p_4 \lor q_2
  && \lnot p_7 \lor q_3 \\
  && \lnot p_2 \lor q_1
  && \lnot p_5 \lor q_2
  && \lnot p_8 \lor q_3 \\
  && \lnot p_3 \lor q_1
  && \lnot p_6 \lor q_2
  && \lnot p_9 \lor q_3 \\
  && && && \lnot q_1 \lor \lnot q_2 \\
  && && && \lnot q_1 \lor \lnot q_3 \\
  && && && \lnot q_2 \lor \lnot q_3 \\
  && && && q_1 \lor q_2 \lor q_3 \\
\end{align*}
%% \begin{center}
%%   \begin{tabular}{|c|c|c|c|}
%%     \hline
%%     $i$ & Free vars & Aux vars & Formulas \\ \hline
%%     1 & 3 & 0 & 3 \\ \hline
%%     2 & 9 & 3 & 24 \\ \hline
%%   \end{tabular}
%% \end{center}

%% At level $i$ of the tree,
%% there are $n=3^i$ free variables,
%% $(n-3)/2$ auxiliary variables
%% (for a total of $(3n-3)/2$ variables),
%% and $(7n-15)/2$ formulas.

% Explain unit prop before this
In this representation,
there are not only fewer clauses to check
but also a Sudoku solving trick that we get for free:
the technique of intersection removal.
Unit propagation is able to reason with this
and proceed further toward a solution
before having to guess than before,
which is great because backtracking is expensive.
% Explain intersection removal

\end{document}
