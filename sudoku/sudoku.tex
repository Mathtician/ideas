\documentclass{article}
\usepackage{amsmath}
\usepackage{amssymb}
\usepackage{graphicx}
\usepackage[margin=1in]{geometry}
\usepackage{hyperref}
\usepackage{caption}
\usepackage{float}
\graphicspath{{images/}}
\hypersetup{
  colorlinks=true,
  urlcolor=blue,
}
\begin{document}

\title{SAT Solving Sudoku}
\author{Aresh Pourkavoos}
\maketitle

Sudoku is a puzzle game which consists of a $9 \times 9$ square grid,
subdivided into $3 \times 3$ blocks
and partially filled with the digits 1-9.
The goal is to fill in the rest of the digits such that
every row, column and block contains each digit exactly once.
Sudoku is a well-studied problem,
and there are many algorithms out there to solve it,
from pencil-and-paper tricks used mostly by human players
to trial-and-error methods used by computers, such as backtracking.
However, I will focus on one particular approach: SAT solving.

SAT is short for ``satisfiability,''
and a SAT solver is a program that checks
whether a given statement is satisfiable.
In this case, a statement is a logical proposition
such as ``P and Q'',
and a statement is satisfiable
if there is some way to assign each variable
(P and Q, in this case) to either true or false
such that the entire statement is true.
``P and Q,'' for example, is satisfiable:
it has exactly one solution, where P and Q are both true.
``P and not P,'' on the other hand, is not satisfiable,
since for both possible values of P, the statement is false.
A brute-force SAT solver would simply check every possible combination,
of which there are $2^n$, where $n$ is the number of variables.
\begin{align*}
  && \underline{a_0 \text{ free}}|
  && a_1 \text{ free}|
  && a_2 \text{ free}
  && a_3 \text{ free}|
  && a_4 \text{ free}
  && a_5 \text{ free}
  && a_6 \text{ free}
  && a_7 \text{ free}| \\
  && && \underline{\lnot a_0 \lor \lnot a_1}|
  && && \lnot a_2 \lor \lnot a_3|
  && && \lnot a_4 \lor \lnot a_5
  && && \lnot a_6 \lor \lnot a_7| \\
  && && a_{01} := a_0 \lor a_1
  && && a_{23} := a_2 \lor a_3|
  && && a_{45} := a_4 \lor a_5
  && && a_{67} := a_6 \lor a_7| \\
  && && \lnot a_{01} \lor a_0 \lor a_1
  && && \lnot a_{23} \lor a_2 \lor a_3|
  && && \lnot a_{45} \lor a_4 \lor a_5
  && && \lnot a_{67} \lor a_6 \lor a_7| \\
  && && \lnot a_0 \lor a_{01}
  && && \lnot a_2 \lor a_{23}|
  && && \lnot a_4 \lor a_{45}
  && && \lnot a_6 \lor a_{67}| \\
  && && \lnot a_1 \lor a_{01}
  && && \lnot a_3 \lor a_{23}|
  && && \lnot a_5 \lor a_{45}
  && && \lnot a_7 \lor a_{67}| \\
  && && && && \underline{\lnot a_{01} \lor \lnot  a_{23}}|
  && && && && \lnot a_{45} \lor \lnot  a_{67}| \\
  && && && && a_{03} := a_{01} \lor a_{23}
  && && && && a_{47} := a_{45} \lor a_{67}| \\
  && && && && \lnot a_{03} \lor a_{01} \lor a_{23}
  && && && && \lnot a_{47} \lor a_{45} \lor a_{67}| \\
  && && && && \lnot a_{01} \lor a_{03}
  && && && && \lnot a_{45} \lor a_{47}| \\
  && && && && \lnot a_{23} \lor a_{03}
  && && && && \lnot a_{67} \lor a_{47}| \\
  && && && && && && && && \underline{\lnot a_{03} \lor \lnot a_{47}}|
\end{align*}

\end{document}
