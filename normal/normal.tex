\documentclass{article}
\usepackage{amsmath}
\usepackage{amssymb}
\usepackage{graphicx}
\usepackage[margin=1in]{geometry}
\usepackage{hyperref}
\usepackage{caption}
\usepackage{float}
\graphicspath{{images/}}
\hypersetup{
    colorlinks=true,
    urlcolor=blue,
}
\begin{document}

\title{A Normal(?) Sequence}
\author{Aresh Pourkavoos}
\maketitle

This idea is about OEIS entry
\href{https://oeis.org/A330731}{A330731},
which I submitted,
and most of the information here
is also available there.
Nonetheless, I wanted to explore the sequence
in a bit more detail with less of a wait time before publication,
hence the separate source.

The sequence is infinite and binary,
i.e. its entries are all either 0 or 1.
I designed it to be \textit{normal},
meaning that as the number of terms increases,
every subsequence of the same length appears with equal frequency.
For example, a randomly chosen subsequence of length 5
has a 1 in 32 chance of being 01001.
At the moment, I still have not proven this property,
but experimentally, this seems to be the case.
However, I hope to prove that it is normal eventually,
and in fact, I believe that the frequency of a given substring
converges to its final value
\textit{faster} than a ``random'' sequence.

The sequence is defined by...

The naive algorithm to generate the sequence per the definition
takes $O(n^3)$ time for the first $n$ terms.
This is because to generate each new term,
the frequencies of $O(n)$ different tails are checked
(from the whole sequence so far down to a relatively short tail),
and each counting requires a pass over all terms,
which takes $O(n)$ time.
In other words, each term takes $O(n^2)$ time to generate.
However, with the right data structures,
I was able to reduce the time per term to $O(n)$,
bringing the time for the first $n$ terms to $O(n^2)$.

\end{document}
