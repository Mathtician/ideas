\documentclass{article}
\usepackage{amsmath}
\usepackage{graphicx}
\usepackage{bm}
\usepackage[margin=1in]{geometry}
\graphicspath{{images/}}
\begin{document}

Paper: Discrete Fourier transform methods in the theory of equations, BR Neelley, 1992, https://ttu-ir.tdl.org/bitstream/handle/2346/60037/31295007093692.pdf?sequence=1

Solving Quadratics, Cubics, and Quartics with the Discrete Fourier Transform

Quadratics (Po-Shen Loh)
\begin{align*}
  &\bm{Ax^2+Bx+C=0, A\neq0} \\
  &\bm{b=\frac{B}{A}, c=\frac{C}{A}} \\
  &x^2+bx+c=0 \\
  &\bm{x=x_1,x_2} \\
  &\bm{x_1=p+q, x_2=p-q} \\
  &\bm{p=-\frac{b}{2}}, b=-2p \\
  &y=x-p, x=y+p
\end{align*}
\begin{align*}
  &0=(y+p)^2-2p(y+p)+c=y^2+(c-p^2) \\
  &\bm{c'=c-p^2}
\end{align*}
\begin{align*}
  &0=(x-x_1)(x-x_2)=(x-(p+q))(x-(p-q))=(y-q)(y+q)=y^2-q^2
\end{align*}
\begin{align*}
  &c'=-q^2 \\
  &q^2=-c' \\
  &\bm{q=(-c')^\frac{1}{2}}
\end{align*}
\newpage
Cubics
\begin{align*}
  &\bm{Ax^3+Bx^2+Cx+D=0, A\neq0} \\
  &\bm{b=\frac{B}{A}, c=\frac{C}{A}, d=\frac{D}{A}} \\
  &x^3+bx^2+cx+d=0 \\
  &\bm{x=x_1,x_2,x_3} \\
  &\bm{\omega=\frac{\sqrt{3}i-1}{2}} \\
  &\omega^2+\omega+1=0, \omega^2+\omega=-1, \omega^3=1 \\
  &\bm{x_1=p+q+r, x_2=p+\omega q+\omega^2r, x_3=p+\omega^2q+\omega r} \\
  &\bm{p=-\frac{b}{3}} \\
  &y=x-p, x=y+p
\end{align*}
\begin{align*}
  0&=(y+p)^3-3p(y+p)^2+c(y+p)+d \\
  %&=y^3+3py^2+3p^2y+p^3-3py^2-6p^2y-3p^3+cy+pc+d \\
  %&=y^3+(3p-3p)y^2+(3p^2-6p^2+c)y+(p^3-3p^3+pc+d) \\
  &=y^3+(c-3p^2)y+(d-2p^3+pc) \\
  &\bm{c'=c-3p^2, d'=d-2p^3+pc}
\end{align*}  
\begin{align*}
  0&=(x-x_1)(x-x_2)(x-x_3) \\
  &=(x-(p+q+r))(x-(p+\omega q+\omega^2r))(x-(p+\omega^2q+\omega r)) \\
  %&=((x-p)-(q+r))((x-p)-(\omega q+\omega^2r))((x-p)-(\omega^2q+\omega r)) \\
  &=(y-(q+r))(y-(\omega q+\omega^2r))(y-(\omega^2q+\omega r)) \\
  %&=(y-(q+r))(y^2-(\omega^2q+\omega r)y-(\omega q+\omega^2r)y+(\omega q+\omega^2r)(\omega^2q+\omega r)) \\
  %&=(y-(q+r))(y^2-(\omega^2q+\omega r+\omega q+\omega^2r)y+(\omega^3q^2+\omega^2qr+\omega^4qr+\omega^3r^2)) \\
  %&=(y-(q+r))(y^2-(\omega^2q+\omega r+\omega q+\omega^2r)y+(q^2+\omega^2qr+\omega qr+r^2)) \\
  %&=(y-(q+r))(y^2-((\omega^2+\omega)q+(\omega+\omega^2)r)y+(q^2+(\omega^2+\omega)qr+r^2)) \\
  %&=(y-(q+r))(y^2-(-q-r)y+(q^2-qr+r^2)) \\
  %&=(y-(q+r))(y^2+(q+r)y+(q^2-qr+r^2)) \\
  %&=y^3+(q+r)y^2+(q^2-qr+r^2)y-(q+r)y^2-(q+r)^2y-(q+r)(q^2-qr+r^2) \\
  %&=y^3+((q^2-qr+r^2)-(q+r)^2)y-(q+r)(q^2-qr+r^2) \\
  %&=y^3+((q^2-qr+r^2)-(q^2+2qr+r^2))y-(q^3-q^2r+qr^2+q^2r-qr^2+r^3) \\
  &=y^3-3qry-(q^3+r^3)
\end{align*}
\begin{align*}
  &c'=-3qr, d'=-(q^3+r^3) \\
  &qr=-\frac{c'}{3} \\
  &u_1=q^3, u_2=r^3 \\
  &\bm{u=u_1, u_2} \\
  &0=(u-u_1)(u-u_2)=u^2-(u_1+u_2)u+u_1u_2=u^2-(q^3+r^3)u+q^3r^3=u^2-(q^3+r^3)u+(qr)^3 \\
  &\bm{u^2+d'u-\left(\frac{c'}{3}\right)^3=0} \\
  &\bm{q=u_1^\frac{1}{3}} \\
  &q=0: \bm{r=u_2^\frac{1}{3}} \\
  &q\neq0: \bm{r=-\frac{c'}{3q}}
\end{align*}
\newpage
Quartics
\begin{align*}  
  &\bm{Ax^4+Bx^3+Cx^2+Dx+E=0, A\neq0} \\
  &\bm{b=\frac{B}{A}, c=\frac{C}{A}, d=\frac{D}{A}, e=\frac{E}{A}} \\
  &x^4+bx^3+cx^2+dx+e=0 \\
  &\bm{x=x_1,x_2,x_3,x_4} \\
  &\bm{x_1=p+q+r+s} \\
  &\bm{x_2=p+qi-r-si} \\
  &\bm{x_3=p-q+r-s} \\
  &\bm{x_4=p-qi-r+si} \\
  &\bm{p=-\frac{b}{4}} \\
  &y=x-p, x=y+p
\end{align*}
\begin{align*}
  0&=(y+p)^4-4p(y+p)^3+c(y+p)^2+d(y+p)+e \\
  &=y^4+(c-6p^2)y^2+(d-8p^3+2pc)y+(e-3p^4+p^2c+pd) \\
  &\bm{c'=c-6p^2, d'=d-8p^3+2pc, e'=e-3p^4+p^2c+pd}
\end{align*}
\begin{align*}
  0&=(x-x_1)(x-x_2)(x-x_3)(x-x_4) \\
  &=(x-(p+q+r+s))(x-(p+qi-r-si))(x-(p-q+r-s))(x-(p-qi-r+si)) \\
  &=(y-(q+r+s))(y-(qi-r-si))(y-(-q+r-s))(y-(-qi-r+si)) \\
  &=y^4-(2r^2+4qs)y^2-4r(q^2+s^2)y-(q^2+s^2)^2+r^4-4qr^2s
  %u&=q^2+s^2, v=4qs \\
  %0&=y^4-(2r^2+v)y^2-4ruy-u^2+r^4-r^2v
\end{align*}
\begin{align*}
  c'&=-(2r^2+4qs), d'=-4r(q^2+s^2), e'=-(q^2+s^2)^2+r^4-4qr^2s\\
  t&=q^2+s^2, u=4qs \\
  qs&=\frac{u}{4} \\
  v_1&=q^2, v_2=s^2 \\
  &\bm{v=v_1, v_2} \\
  0&=(v-v_1)(v-v_2)=v^2-(v_1+v_2)v+v_1v_2=v^2-(q^2+s^2)v+q^2s^2=v^2-(q^2+s^2)v+(qs)^2 \\
  &\bm{v^2-tv+\left(\frac{u}{4}\right)^2=0} \\
  &\bm{q=v_1^\frac{1}{2}} \\
  &q=0: \bm{s=v_2^\frac{1}{2}} \\
  &q\neq0: \bm{s=\frac{u}{4q}}
\end{align*}
\begin{align*}
  c'&=-(2r^2+u), d'=-4rt, e'=-t^2+r^4-r^2u \\
  u&=-(2r^2+c') \\
  -\frac{d'}{4}&=rt \\
  w&=r^2 \\
  \left(\frac{d'}{4}\right)^2&=r^2t^2=t^2w \\
  e'&=-t^2+r^4-r^2(-2r^2+c')=3r^4-c'r^2-t^2=3w^2-c'w-t^2 \\
  e'w&=3w^3-c'w^2-t^2w=3w^3-c'w^2-\left(\frac{d'}{4}\right)^2 \\
  &\bm{3w^3-c'w^2-e'w-\left(\frac{d'}{4}\right)^2=0} \\
  &\bm{r=w^\frac{1}{2}} \\
  &\bm{t=-\frac{d'}{4r}} \\
  &\bm{u=-(2w+c')} \\
\end{align*}
%% \begin{align*}
%%   0&=(x-x_1)(x-x_2)(x-x_3)(x-x_4) \\
%%   &=(x-(p+q+r+s))(x-(p+qi-r-si))(x-(p-q+r-s))(x-(p-qi-r+si)) \\
%%   &=((x-p)-(q+r+s))((x-p)-(qi-r-si))((x-p)-(q+r-s))((x-p)-(-qi-r+si)) \\
%%   &=(y-(q+r+s))(y-(qi-r-si))(y-(-q+r-s))(y-(-qi-r+si)) \\
%%   &=((y-r)-(q+s))((y+r)-(q-s)i)((y-r)+(q+s))((y+r)+(q-s)i)) \\
%%   &=((y-r)-(q+s))((y-r)+(q+s))((y+r)-(q-s)i)((y+r)+(q-s)i)) \\
%%   &=((y-r)^2-(q+s)^2)((y+r)^2-((q-s)i)^2) \\
%%   &=((y-r)^2-(q+s)^2)((y+r)^2+(q-s)^2) \\
%%   &=(y^2-2ry+r^2-q^2-2qs-s^2)(y^2+2ry+r^2+q^2-2qs+s^2) \\
%%   &=((y^2+r^2-2qs)-(2ry+q^2+s^2))((y^2+r^2-2qs)+(2ry+q^2+s^2)) \\
%%   &=(y^2+r^2-2qs)^2-(2ry+q^2+s^2)^2
%% \end{align*}
%% \begin{align*}
%%   &u=r^2-2qs, v=q^2+s^2
%% \end{align*}
%% \begin{align*}
%%   0&=(y^2+u)^2-(2ry+v)^2 \\
%%   &=(y^4+2uy^2+u^2)-(4r^2y^2+4rvy+v^2) \\
%%   &=y^4+2uy^2+u^2-4r^2y^2-4rvy-v^2 \\
%%   &=y^4+(2u-4r^2)y^2-4rvy+u^2-v^2
%% \end{align*}
%% \begin{align*}
%%   &c'=2u-4r^2, d'=-4rv, e'=u^2-v^2
%% \end{align*}
%% \begin{align*}
%%   0&=y^4+c'y^2+d'y+e' \\
%%   &=(x-p)^4+c'(x-p)^2+d'(x-p)+e' \\
%%   &=(x^4-4px^3+6p^2x^2-4p^3x+p^4)+c'(x^2-2px+p^2)+d'(x-p)+e' \\
%%   &=x^4-4px^3+6p^2x^2-4p^3x+p^4+c'x^2-2pc'x+p^2c'+d'x-pd'+e' \\
%%   &=x^4-4px^3+(6p^2+c')x^2+(-4p^3-2pc'+d')x+p^4+p^2c'-pd'+e' \\
%%   &=x^4+bx^3+cx^2+dx+e
%% \end{align*}
%% \begin{align*}
%%   &b=-4p, c=6p^2+c', d=-4p^3-2pc'+d', e=p^4+p^2c'-pd'+e' \\
%%   &c'=c-6p^2=2u-4r^2 \\
%%   &d'=d+4p^3+2pc'=d+4p^3+2p(c-6p^2)=d+4p^3+2pc-12p^3=d-8p^3+2pc=-4rv \\
%%   &e'=e-p^4-p^2c'+pd' \\
%%   &=e-p^4-p^2(c-6p^2)+p(d-8p^3+2pc) \\
%%   &=e-p^4-p^2c+6p^4+pd-8p^4+2p^2c \\
%%   &=e-3p^4+p^2c+pd \\
%%   &=u^2-v^2 \\
%%   &w=2r^2 \\
%%   &\bm{r=\left(\frac{w}{2}\right)^\frac{1}{2}} \\
%%   &2u=4r^2+c' \\
%%   &u=2r^2+\frac{c'}{2} \\
%%   &=w+\frac{c'}{2} \\
%%   &d'^2=16v^2r^2 \\
%%   &\frac{d'^2}{8}=2v^2r^2 \\
%%   &=v^2w \\
%%   &e'w=u^2w-v^2w \\
%%   &=\left(w+\frac{c'}{2}\right)^2w-\frac{d'^2}{8} \\
%%   &=\left(w^2+c'w+\frac{c'^2}{4}\right)w-\frac{d'^2}{8} \\
%%   &=w^3+c'w^2+\frac{c'^2}{4}w-\frac{d'^2}{8} \\
%%   &\bm{w^3+c'w^2+\left(\frac{c'^2}{4}-e'\right)w-\frac{d'^2}{8}=0} \\
%%   &\bm{u=w+\frac{c'}{2}} \\
%%   &\bm{v=\frac{d'^2}{8w}}
%% \end{align*}  

\end{document}
