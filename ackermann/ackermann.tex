\documentclass{article}
\usepackage{amsmath}
\usepackage{amssymb}
\usepackage{graphicx}
\usepackage[margin=1in]{geometry}
\usepackage{hyperref}
\usepackage{caption}
\usepackage{float}
\usepackage[utf8]{inputenc}
\graphicspath{{images/}}
\hypersetup{
  colorlinks=true,
  urlcolor=blue,
}
\begin{document}

\title{Extending the Ackermann function}
\author{Aresh Pourkavoos}
\maketitle

The Ackermann function is an important function
in the history of computer science.
It is defined as follows,
where its arguments are natural numbers:
\begin{align*}
  A(0, n) &= n+1 \\
  A(m+1, 0) &= A(m, 1) \\
  A(m+1, n+1) &= A(m, A(m+1, n))
\end{align*}
Although it may look simple,
it was the first known example
of a function which is computable but not primitive recursive.
\footnote{
  The original Ackermann function was slightly more complicated,
  but this version takes roughly as long as the original to execute
  and returns roughly the same size of outputs.
}
Here, ``computable'' means that there is an algorithm
to calculate it that always takes a finite number of steps,
and ``primitive recursive'' means that such an algorithm
can be rewritten to use only \textit{bounded loops}.
In other words, if a block of code in a primitive recursive function
needs to be executed $n$ times (a loop),
a maximum possible value of $n$ needs to be known beforehand.
So using the definition above,
we can prove that the function will give an answer eventually
(i.e. it won't get stuck in an infinite loop),
but while the program is being executed,
we won't be able to tell how much longer is left.

For the first few values of $m$,
the output of $A(m, n)$ is as follows:
\begin{center}
  \begin{tabular}{|c|c|}
    \hline
    m & A(n, m) \\ \hline
    0 & $n+1$ \\ \hline
    1 & $n+2$ \\ \hline
    2 & $2n+3$ \\ \hline
    3 & $2^{n+3}-3$ \\ \hline
  \end{tabular}
\end{center}
It may be hard to see a pattern here,
but $m=4$ is where it gets interesting.
\[
A(4, n) = \underbrace{2^{2^{2^{...}}}}_{n+3 \text{ 2s}}-3,
\]
where parentheses are placed around the top,
i.e. $3^{3^3}=3^{(3^3)}=7625597484987$, not $(3^3)^3=19683$.
The function performs repeated exponentiation,
which allows for relatively large numbers to be made quickly.
For example, $A(4, 1)=2^{2^{2^2}}-3=2^{2^4}-3=2^{16}-3=65536-3=65533$,
and $A(4, 2)=2^{65536}-3$,
a number with almost 20000 decimal digits!

This repeated exponentiation is known as tetration
and may be denoted $a \uparrow \uparrow b$,
where $a$ is the number forming the power tower
and $b$ is its height,
so $A(4, n)$ may be written $2\uparrow\uparrow(n+3)-3$.
Two up arrows are used because one is reserved for exponents:
$A(3, n) = 2\uparrow(n+3)-3$.
Given the similarities,
we may also interpret the lower values of $m$:
$A(2, n) = 2(n+3)-3$, $A(1, n) = 2+(n+3)-3$, $A(0, n) = S(n+3)-3$,
where $S$ denotes the successor function,
which adds 1 to its argument.

Each operation is performed by repetition of the last:
addition is the repeated successor,
multiplication is repeated addition,
exponentiation is repeated multiplication,
and tetration is repeated exponentiation.
Larger values of $m$ continue this trend:
repeated tetration is known as pentation and denoted by 3 up arrows,
and $A(6, n) = 2\uparrow\uparrow\uparrow(n+3)-3$.

The reason that this function is not primitive recursive
is that primitive recursive functions can be shown to terminate with a single inductive proof:
show that $f(0)$ terminates,
and show that the termination of $f(n)$ implies that of $f(n+1)$ for all natural $n$.

Interpret arguments as coefficients of powers of $\omega$:
original function accepts ordinal $<\omega^2$
\begin{align*}
  A(n) &= n+1 \\
  A(\omega(m+1)) &= A(\omega m+1) \\
  A(\omega(m+1)+(n+1)) &= A(\omega m+A(\omega(m+1)+n))
\end{align*}

$X$: (possibly empty) list of naturals
$Y$: (possibly empty) list of zeros
$a, b$: naturals
\begin{align*}
  A(Y, a) &= a+1 \\
  A(X, b+1, Y, 0) &= A(X, b, Y, 1) \\
  A(X, b+1, Y, a+1) &= A(X, b, A(X, b+1, Y, a), Y) \\
\end{align*}
To find which of these 3 cases a given string of arguments falls into,
we look at all arguments except the last one.
If they are all 0 (or if there are no other arguments), case 1 applies.
Otherwise, there is at least one nonzero argument other than the last one.
Call the rightmost of these $b+1$,
so all arguments between it and the last argument are 0.
These go into the list $Y$,
while all arguments before $b+1$ go into $X$.
From here, the case is determined by the last argument:
if it's 0, case 2 applies,
and otherwise, case 3.

\begin{align*}
  A(1, 0, 0) &= A(0, 0, 1) = 2 \\
  A(1, 0, 1) &= A(0, A(1, 0, 0), 0) = A() \\
\end{align*}

Extension to higher ordinals

%% can give any ordinal $< \omega^\omega$
%% Leading zeros are discarded (as in the original Ackermann function):
%% $A(0, n)=A(n)=n+1$
%% Recursive case relies on function calls that are lexicographically earlier:
%% $k$-tuples of natural numbers are well-ordered (by $w^k$)


\end{document}
