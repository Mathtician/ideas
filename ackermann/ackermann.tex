\documentclass{article}
\usepackage{amsmath}
\usepackage{amssymb}
\usepackage{graphicx}
\usepackage[margin=1in]{geometry}
\usepackage{hyperref}
\usepackage{caption}
\usepackage{float}
\graphicspath{{images/}}
\hypersetup{
    colorlinks=true,
    urlcolor=blue,
}
\begin{document}

\title{Extending the Ackermann function}
\author{Aresh Pourkavoos}
\maketitle

Interpret arguments as coefficients of powers of $\omega$:
original function accepts ordinal $<\omega^2$
\begin{align*}
A(n) &= n+1 \\
A(\omega(m+1)) &= A(\omega m+1) \\
A(\omega(m+1)+(n+1)) &= A(\omega m+A(\omega(m+1)+n)) \\
\end{align*}

%% can give any ordinal $< \omega^\omega$
%% Leading zeros are discarded (as in the original Ackermann function):
%% $A(0, n)=A(n)=n+1$
%% Recursive case relies on function calls that are lexicographically earlier:
%% $k$-tuples of natural numbers are well-ordered (by $w^k$)


\end{document}
