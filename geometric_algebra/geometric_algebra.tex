\documentclass{article}
\usepackage{amsmath}
\usepackage{amssymb}
\usepackage{graphicx}
\usepackage[margin=1in]{geometry}
\usepackage{hyperref}
\usepackage{caption}
\usepackage{float}
\usepackage{bookmark}
\graphicspath{{images/}}
\hypersetup{
  colorlinks=true,
  urlcolor=blue,
}
\begin{document}

\title{Notes on Geometric Algebra}
\author{Aresh Pourkavoos}
\date{}
\maketitle

\tableofcontents

\section{Introduction}
Geometric algebras (or real Clifford algebras) provide a unified framework
for diverse topics in math and physics,
mostly centered on transformations in $n$-dimensional space.
These include but are not limited to
the familiar reflections, rotations, and translations of Euclidean space;
the hyperbolic rotations, AKA Lorentz boosts, which appear in special relativity;
the circle inversions used in compass-and-straightedge constructions
and their higher-dimensional analogs;
and the perspective transformations which find applications in computer vision.
They can represent area, volume, and beyond
in the same way that vectors represent length.
When Clifford algebras are constructed using complex numbers instead of reals,
they capture the unitary transformations and spinors of quantum mechanics.
When combined with calculus, geometric algebra encompasses the theory of differential forms.

These notes assume prior knowledge of introductory linear algebra,
i.e. real vector spaces, matrix multiplication,
diagonalization, inner products, etc.

\section{Definition}

An algebra over a vector space $V$ is created

\subsection{Example: the exterior algebra}

\section{Properties}

\subsection{Conic sections: a tangent}

\section{Symmetries of spacetime}
\subsection{Classical mechanics}
Classical (Newtonian) mechanics relies on three principles
governing the symmetries of space and time.
For this section, ``frames of reference'' refer specifically to inertial frames of reference,
i.e. those which are not accelerating.
\begin{enumerate}
\item Shifts between frames of reference are performed by linear transformations.
\item The laws of physics remain the same for all frames of reference.
\item The passage of time is absolute for all frames of reference.
\end{enumerate}
\subsection{Special relativity}
% Don't write anything here until the other section is done!

\section{From \texorpdfstring{$\mathbb{R}$}{R}
  to \texorpdfstring{$\mathbb{C}$}{C}
  to \texorpdfstring{$\mathbb{H}$}{H}}

So far, the components of multivectors have been real numbers,
and the Clifford algebra has been generated by a quadratic form.
With a few tricks related to representations, we can extend what we know
about Clifford algebras over the real numbers $\mathbb{R}$
to related algebras over the complex numbers $\mathbb{C}$
and algebra-like objects over the quaternions $\mathbb{H}$.

\subsection{\texorpdfstring{$\mathbb{R}$}{R} to \texorpdfstring{$\mathbb{C}$}{C}}

A complex number $a+bi$ can be represented by the $2 \times 2$ real matrix
$\begin{bmatrix} a & -b \\ b & a \end{bmatrix}$
so that complex addition, multiplication, and inverses
map to the corresponding operations on matrices.
For example,
\begin{align*}
  (a+bi) (c+di) &= (ac-bd)+(bc+ad)i; \\
  \begin{bmatrix} a & -b \\ b & a \end{bmatrix}
  \begin{bmatrix} c & -d \\ d & c \end{bmatrix} &=
  \begin{bmatrix} ac-bd & -(bc+ad) \\ bc+ad & ac-bd \end{bmatrix}.
\end{align*}
Likewise, an $m \times n$ complex matrix can be represented as a $2m \times 2n$ real matrix,
with each $2 \times 2$ block of the real matrix representing a single entry of the complex matrix.
Since breaking down matrices into blocks, multiplying them as units,
then adding the results together is a valid method of matrix multiplication,
the complex matrix product is preserved with this real representation.

\subsection{\texorpdfstring{$\mathbb{C}$}{C} to \texorpdfstring{$\mathbb{H}$}{H}}

\end{document}
