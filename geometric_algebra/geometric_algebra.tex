\documentclass{article}
\usepackage{amsmath}
\usepackage{amssymb}
\usepackage{graphicx}
\usepackage[margin=1in]{geometry}
\usepackage{hyperref}
\usepackage{caption}
\usepackage{float}
\graphicspath{{images/}}
\hypersetup{
    colorlinks=true,
    urlcolor=blue,
}
\begin{document}

\title{Notes on Geometric Algebra}
\author{Aresh Pourkavoos}
\date{}
\maketitle

\tableofcontents

\section{Introduction}
Geometric algebras (or real Clifford algebras) provide a unified framework
for diverse topics in math and physics,
mostly centered on transformations in $n$-dimensional space.
These include but are not limited to
the familiar reflections, rotations, and translations of Euclidean space;
the hyperbolic rotations, AKA Lorentz boosts, which appear in special relativity;
the circle inversions used in compass-and-straightedge constructions
and their higher-dimensional analogs;
and the perspective transformations which find applications in computer vision.
They can represent area, volume, and beyond
in the same way that vectors alone represent length.
When Clifford algebras are constructed using complex numbers instead of reals,
they capture the unitary transformations and spinors of quantum mechanics.
When combined with calculus, geometric algebra encompasses the theory of differential forms.

These notes assume prior knowledge of introductory linear algebra,
i.e. real vector spaces, matrix multiplication,
diagonalization, inner products, etc.

\section{Definition}

An algebra over a vector space $V$ is created

\subsection{Example: the exterior algebra}

\section{}

\end{document}
