\documentclass{article}
\usepackage{amsmath}
\usepackage{amssymb}
\usepackage{graphicx}
\usepackage[margin=1in]{geometry}
\usepackage{hyperref}
\usepackage{caption}
\usepackage{float}
\graphicspath{{images/}}
\hypersetup{
    colorlinks=true,
    urlcolor=blue,
}
\begin{document}

\title{Recurrence Relations and Differential Equations}
\author{Aresh Pourkavoos}
\maketitle

What do these two have in common?
\begin{align*}
  f''(x) &= f'(x)+f(x) \\
  F_{n+2} &= F_{n+1}+F_n
\end{align*}
The first is a differential equation,
which relates the value of a function to its derivatives,
and the other is a recurrence relation,
a rule for building a number sequence by looking at previous entries.
Beyond their visual similarities,
their solutions both involve the golden ratio and its conjugate,
\begin{align*}
  \varphi &= \frac{1+\sqrt{5}}{2} \approx 1.618 \\
  \overline{\varphi} &= \frac{1-\sqrt{5}}{2} \approx -0.618,
\end{align*}
the two solutions of the polynomial
$x^2=x+1$.
Specifically, the general solution to the first equation is
\[f(x) = a\exp(\varphi x)+b\exp(\overline{\varphi}x),\]
and the second is
\[F_n = a\varphi^n+b\overline{\varphi}^n.\]
Both have two degrees of freedom,
which makes sense considering the equations themselves:
in the differential equation,
$f(0)$ and $f'(0)$ may be chosen freely,
and $f''(0)$ and everything else are uniquely determined.
In the recurrence relation,
$F_0$ and $F_1$ may be chosen freely,
and $F_2$ and all other terms follow from them.
Additionally, both equations are linear,
meaning that it is possible to multiply a solution by a number
to obtain another solution,
and solutions may also be added together.
Given the two degrees of freedom
and the linearity,
all we need to do in both cases
is to find two solutions
which are not multiples of each other,
and all other solutions may be made from them
through multiplication and addition.
In linear algebra terms,
the solutions form a two-dimensional vector space,
so any independent set of two vectors
forms a basis of this vector space,
of which all other vectors are linear combinations.
The question then becomes: how do we find such a basis?

There is an easy way to find \textit{a} basis
which gives little information about the actual solution:
set one of the degrees of freedom to 1 and the other to 0,
extrapolate to get a full solution,
repeat for both degrees of freedom.
For example, in the case of $F$,
the two solutions starting with $1, 0$ and $0, 1$
are $1, 0, 1, 1, 2, 3, \ldots$ and 

% Repeated roots

% Zero roots

% How are they both connected to diagonalization?

\end{document}
