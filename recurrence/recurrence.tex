\documentclass{article}
\usepackage{amsmath}
\usepackage{amssymb}
\usepackage{graphicx}
\usepackage[margin=1in]{geometry}
\usepackage{hyperref}
\usepackage{caption}
\usepackage{float}
\graphicspath{{images/}}
\hypersetup{
    colorlinks=true,
    urlcolor=blue,
}
\begin{document}

\title{Recurrence Relations and Differential Equations}
\author{Aresh Pourkavoos}
\maketitle

What do these two have in common?
\begin{align*}
  f''(x) &= f'(x)+f(x) \\
  F_{n+2} &= F_{n+1}+F_n
\end{align*}
The first is a differential equation,
which relates the value of a function to its derivatives,
and the other is a recurrence relation,
a rule for building a number sequence by looking at previous entries.
Beyond their visual similarities,
their solutions both involve the golden ratio and its conjugate,
\begin{align*}
  \varphi &= \frac{1+\sqrt{5}}{2} \approx 1.618 \\
  \overline{\varphi} &= \frac{1-\sqrt{5}}{2} \approx -0.618,
\end{align*}
the two solutions of the polynomial
$x^2=x+1$.
Specifically, the general solution to the first equation is
\[f(x) = a\exp(\varphi x)+b\exp(\overline{\varphi}x),\]
and the second is
\[F_n = a\varphi^n+b\overline{\varphi}^n.\]

% Repeated roots

% Zero roots

% How are they both connected to diagonalization?

\end{document}
