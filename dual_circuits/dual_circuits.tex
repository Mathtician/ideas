\documentclass{article}
\usepackage{amsmath}
\usepackage{graphicx}
\usepackage[margin=1in]{geometry}
\usepackage{hyperref}
\usepackage{caption}
\usepackage{float}
\usepackage{tikz}
\usepackage{circuitikz}
\usepackage[rightcaption]{sidecap}
\sidecaptionvpos{figure}{m}
\usetikzlibrary{arrows}
\graphicspath{{images/}}
\hypersetup{
  colorlinks=true,
  urlcolor=blue,
}
\begin{document}

\title{Dual Circuits}
\author{Aresh Pourkavoos}
\maketitle

When I was learning about DC electronic circuits in high school physics,
I noticed a strange kind of symmetry in the rules
used to calculate voltage, current, and resistance.
I've written some of them below in a way that emphasizes this symmetry.

\begin{itemize}
\item
  The total voltage difference around a loop of wire is zero.
  % This is a form of the law of conservation of energy.
\item
  The total current flowing into and out of a node (an intersection of wires) is zero.
  % This is a form of the law of conservation of charge.
\item
  Resistors in series add their resistances.
\item
  Resistors in parallel add their conductances (reciprocal of resistance).
\item
  Resistors in series split the voltage drop across them proportionally to resistance,
  but they receive the same current.
\item
  Resistors in parallel split the current through them proportionally to conductance,
  but they receive the same voltage.
\end{itemize}

Essentially, each pair of laws can be interchanged by switching the following pairs of words:

\begin{align*}
  \text{Voltage} & \leftrightarrow \text{Current} \\
  \text{Loop} & \leftrightarrow \text{Node} \\
  \text{Series} & \leftrightarrow \text{Parallel} \\
  \text{Resistance} & \leftrightarrow \text{Conductance} \\
\end{align*}

Voltage and current can be considered duals,
as can loops and nodes, etc.
Rules 1 and 2 are Kirchhoff's circuit laws.
The other four are consequences of these and Ohm's law:
\[V=IR.\]
Ohm's law itself can be flipped using these dualities to obtain
\[I=V\frac{1}{R},\]
which is just the law itself rearranged.
In this way, Ohm's law is self-dual.

Initially, I didn't know what to make of this duality.
Then, I had the idea of constructing a \textit{dual circuit}.
Given a circuit which obeys Kirchhoff's laws and Ohm's law,
it may be possible to construct a second circuit
whose voltages come from currents of the original,
whose components in parallel come from components in series of the original, etc,
and \textit{which also obeys Kirchhoff's laws and Ohm's law.}
In a way, every circuit (which can lie flat without crossing wires
and is made with only DC batteries and resistors) has a twin!

\begin{SCfigure}[0.8][htb]
  \centering
  \begin{circuitikz}
    \draw (0,1)
    to[short] (0,3)
    to[battery] (2,3)
    to[R=$R4$] (2,1)
    to[short] (0,1);
    \draw (2,3)
    to[R=$R1$] (4,3)
    to[R=$R2$] (4,1)
    to[R=$R3$] (2,1);
    \draw[fill=black] (2,1) circle (1mm);
    \draw[fill=black] (2,3) circle (1mm);
    \draw[fill=black] (4,1) circle (1mm);
    \draw[fill=black] (4,3) circle (1mm);
    \draw[fill=white] (1,2) circle (1mm);
    \draw[fill=white] (3,2) circle (1mm);
    \draw[fill=white] (5,2) circle (1mm);
  \end{circuitikz}

  \caption*{
    R1: 1$\Omega$, 2A, 2V \\
    R2: 2$\Omega$, 2A, 4V \\
    R3: 3$\Omega$, 2A, 6V \\ 
    R4: 3$\Omega$, 4A, 12V \\
    Battery: 6A, 12V
  }
\end{SCfigure}

We will construct the dual of the circuit below as an example.
There are two parts to the process: numerical and topological.
The numerical part consists of finding the voltages, resistances, etc. of its components.
The topological part consists of finding its shape:
where its loops and nodes are and which components are in series or in parallel.
The numerical part is relatively easy;
voltage becomes current, and resistance is reciprocated to become conductance.
For example, R3 is a 3$\Omega$ resistor 
that has 2A passing through it for a drop of 6V.
Its dual would be a $\frac{1}{3}\Omega$ resistor
that has 6A passing through it for a drop of 2V.
The battery supplies 12V to an overall 2$\Omega$ load drawing 6A of current,
so its dual would be a 6V battery with a $\frac{1}{2}\Omega$ load drawing 12A.
The self-duality of Ohm's law shines through.

The more interesting part of constructing a dual circuit is topological.
How exactly can one switch loops (white dots) and nodes (black dots)
in a circuit like the one above without loose ends or shorts?
The answer, as it turns out, comes from graph theory.

A circuit can be represented by a graph
whose edges correspond to circuit components
and whose vertices correspond to nodes.
Loops, then, are faces of the graph:
contiguous areas that are surrounded completely by edges.
There is an operation on graphs
that does exactly what we want for dual circuits,
fittingly called the \textit{dual graph}.
Faces that share an edge in the original graph
become nodes connected by an edge in the dual, and vice versa.
(This operation works only on graphs which can lie in the plane without edges crossing,
known as planar graphs, but many circuits are planar.)

\begin{tikzpicture}[main node/.style={circle,draw}]
  \hspace{-1cm}
    \node[main node,fill=black] at (0, 1) (1) {};
    \node[main node,fill=black] at (0, 3) (2) {};
    \node[main node,fill=black] at (2, 1) (3) {};
    \node[main node,fill=black] at (2, 3) (4) {};
    \path
    (1) edge node {} (2)
    edge [bend left=90,looseness=3] node {} (2)
    (2) edge node {} (4)
    (3) edge node {} (1)
    (4) edge node {} (3);

    \node[main node,fill=black] at (5, 1) (1) {};
    \node[main node,fill=black] at (5, 3) (2) {};
    \node[main node,fill=black] at (7, 1) (3) {};
    \node[main node,fill=black] at (7, 3) (4) {};
    \node[main node,fill=white] at (4, 2) (5) {};
    \node[main node,fill=white] at (6, 2) (6) {};
    \node[main node,fill=white] at (8, 2) (7) {};
    \path
    (1) edge node {} (2)
    edge [bend left=90,looseness=3] node {} (2)
    (2) edge node {} (4)
    (3) edge node {} (1)
    (4) edge node {} (3)
    (5) edge node {} (6)
    edge [bend left=135,looseness=3] node {} (7)
    (6) edge node {} (7)
    edge [bend left=90,looseness=3] node {} (7)
    edge [bend right=90,looseness=3] node {} (7);

    \node[main node,fill=white] at (10, 2) (5) {};
    \node[main node,fill=white] at (12, 2) (6) {};
    \node[main node,fill=white] at (14, 2) (7) {};
    \path
    (5) edge node {} (6)
    edge [bend left=135,looseness=3] node {} (7)
    (6) edge node {} (7)
    edge [bend left=90,looseness=3] node {} (7)
    edge [bend right=90,looseness=3] node {} (7);
  \end{tikzpicture}

Above are the graph of the original circuit (left), its dual (right),
and their superimposition (center).
Note that while the original circuit appears to have only two loops,
there is a third loop around the outside, which contains the battery and resistors 1-3,
hence the third white vertex on the right.
Note also that each edge of one graph is crossed by exactly one edge of the other---its dual edge---so
the total number of edges remains the same.
This is good news for dual circuits,
since the number of components (one per edge) should not change either.
Combining this graph with the numerical flipping
produces a completely valid circuit.

\begin{SCfigure}[0.8][htb]
  \centering
  \begin{circuitikz}[baseline=0]
    \draw (3,2)
    to[R=$R3$] (3,0)
    to[short] (5,0)
    to[short] (5,2)
    to[short] (5,4)
    to[short] (3,4)
    to[R=$R1$] (3,2);
    \draw (1,2)
    to[R=$R4$] (3,2)
    to[R=$R2$] (5,2)
    to[short] (6,2)
    to[short] (6,5)
    to[short] (5,5)
    to[short] (1,5)
    to[short] (1,4)
    to[battery] (1,2);
    \draw[fill=black] (2,1) circle (1mm);
    \draw[fill=black] (2,3) circle (1mm);
    \draw[fill=black] (4,1) circle (1mm);
    \draw[fill=black] (4,3) circle (1mm);
    \draw[fill=white] (1,2) circle (1mm);
    \draw[fill=white] (3,2) circle (1mm);
    \draw[fill=white] (5,2) circle (1mm);
  \end{circuitikz}
     \caption*{
    R1: $1\Omega$, 2A, 2V \\
    R2: $\frac{1}{2}\Omega$, 4A, 2V \\
    R3: $\frac{1}{3}\Omega$, 6A, 2V \\ 
    R4: $\frac{1}{3}\Omega$, 12A, 4V \\
    Battery: 12A, 6V
  }
\end{SCfigure}

Success! Here we can see the principles of duality at work:
\begin{itemize}
\item
  Where all four resistors were in a loop in the original circuit,
  they connect to a node in the dual circuit.
\item
  Where the battery and R4 were in parallel in the original circuit,
  they are in series in the dual circuit.
\item
  Where the battery's 6A of current was split into 4A for R4 and 2A for the rest in the original circuit,
  its 6V are split into 4V for R4 and 2V for the rest in the dual circuit.
\end{itemize}

A switch can be thought of as an extreme case of a resistor.
An open switch has infinite resistance and no conductance,
and a closed one has no resistance and infinite conductance.
We can add another duality to the list:

\[\text{Open switch} \leftrightarrow \text{Closed switch}\]



While researching for this article, I found aspects of dual circuits in a few places:

\begin{itemize}
\item
  \href{https://en.wikipedia.org/wiki/Topology\_(electrical\_circuits)#Duality}{This Wikipedia article}
  talks about the use of duals in analyzing parts of circuits
  but does not mention the duality between resistance and conductance.
\item
  \href{https://en.wikipedia.org/wiki/Dual\_graph#Applications}{This article}
  mentions the use of duals in creating networks of logic gates,
  where the dual turns AND gates (switches in series)
  into OR gates (switches in parallel) and vice versa.
  However, this is only a topological dual,
  since flipping the state of every switch and interpreting the output as its inverse
  (as numerical duality suggests)
  would restore the logic gates to their former functions.\\
  See \href{https://en.wikipedia.org/wiki/De\_Morgan\'s\_laws}{De Morgan's Laws}: \ \ 
  $\lnot (A \land B) \equiv (\lnot A) \lor (\lnot B)$, $\lnot (A \lor B) \equiv (\lnot A) \land (\lnot B)$.
\item
  In \href{https://www.numberphile.com/videos/squares-and-tilings}{this video},
  Fields medalist Andrei Okounkov constructs a pair of dual circuits from a square tiling.
  However, since all of the resistors are 1$\Omega$,
  the duality between resistance and conductance is obscured.
  
\end{itemize}

After formulating the ideas presented here,
I began to look for the duals of other electrical components,
such as capacitors, and finally found
\href{https://en.wikipedia.org/wiki/Duality\_(electrical\_circuits)}{a more comprehensive list}
of duals in circuits, including all of the ideas I independently derived and more.
Apparently, the dualities of electrical components come from that between electricity and magnetism,
which can itself be explained by special relativity,
but I hardly understand any of that yet.
I'm surprised that, although this duality has been known for over a hundred years,
I never came across it in school or otherwise.
Duality seems like a valuable teaching tool,
since students would only have to memorize half of the usual number of rules!
     
\end{document}


