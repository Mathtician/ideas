\documentclass{article}
\usepackage{amsmath}
\usepackage{amssymb}
\usepackage{graphicx}
\usepackage[margin=1in]{geometry}
\usepackage{hyperref}
\usepackage{caption}
\usepackage{float}
\graphicspath{{images/}}
\hypersetup{
    colorlinks=true,
    urlcolor=blue,
}
\begin{document}

\newcommand{\mobius}{M{\"o}bius\ }

\title{\mobius Transformations and the Complex Projective Line}
\author{Aresh Pourkavoos}
\maketitle

A \mobius transformation is a function on complex numbers
\[f(z)=\frac{az+b}{cz+d},\]
where $a$, $b$, $c$, and $d$ are complex.
Before continuing, watching
\href{http://www-users.math.umn.edu/~arnold//moebius/}{this video}
will give a good idea of how these transformations look on the complex plane.
They include translation, rotation, scaling, and \textit{inversion},
a strange-looking process which turns the plane inside out.

There are two points usually made about these transformations.
First, the parameters $a$, $b$, $c$, and $d$ are restricted
so that the output of the function is not constant
(when it is defined).
For example,
\[g(z)=\frac{2z+1}{6z+3}=3\]
for all $z$ except $-\frac{1}{2}$, where $g(z)=\frac{0}{0}$, which is undefined.
Thus, $g$ is not a \mobius transformation.
In general, $az+b$ and $cz+d$ can't be multiples of each other.
It can be shown that this is equivalent to the statement

\[ad-bc \neq 0.\]

Next, note that when the denominator $cz+d=0$,
the output is not defined within the complex numbers.
For this reason, the transformation's range is often extended
to the \textit{Riemann sphere}, the set of complex numbers with $\infty$.
(This is what the narrator of the video means by
``taking a cue from Bernhard Riemann.'')
$\infty$, roughly speaking, is any nonzero number divided by zero.
($\frac{0}{0}$ is still undefined.)
Then, $f(z)=\infty$ when $cz+d=0$.
(Since $az+b$ must not be a multiple of $cz+d$,
$\frac{0}{0}$ never occurs.)

Let $(s, t) \in$ proj. line:
value is $\infty$ when $t=0$ and $s/t$ otherwise \\
Condition: $0/0$ is undefined, so $(s, t) \neq (0, 0)$\\
\[\frac{a\frac{s}{t}+b}{c\frac{s}{t}+d}=\frac{as+bt}{cs+dt}\]
(this also works for edge case of $\infty$)\\
Applying $f$ to $(s, t)$ gives $(as+bt, cs+dt)$, which is equiv. to matrix multiplication:
\[\begin{bmatrix} a & b \\ c & d \end{bmatrix}
\begin{bmatrix} s \\ t \end{bmatrix} =
\begin{bmatrix} as+bt \\ cs+dt \end{bmatrix}\]
Condition on $f$ is equiv. to invertibility of matrix,
condition on $(s, t)$ is equiv. to $\neq\vec{0}$\\
Invertibility Theorem: $(as+bt, cs+dt)\neq\vec{0}$,
so output is always $\in$ proj. line

%% \begin{figure}[H]
%%   \centering
%%   \includegraphics[width=0.5\linewidth]{/home/a/d/images/z.jpeg}
%%   \caption*{What is this thing? \textit{Generated by \href{https://artbreeder.com}{Artbreeder}}}
%% \end{figure}
%% Use the writeidea command to edit.

\end{document}
