\documentclass{article}
\usepackage[margin=1in]{geometry}
\usepackage{amsmath}
\usepackage{amssymb}
\usepackage{graphicx}
\graphicspath{images/}

\newcommand{\R}{\mathbb{R}}
\DeclareMathOperator{\softmax}{softmax}
\DeclareMathOperator{\soft}{soft}
\DeclareMathOperator{\argmax}{argmax}

\begin{document}

\title{Smooth Sorting}
\author{Aresh Pourkavoos}
\maketitle

$\softmax: \R^n \rightarrow [0, 1]^n$,
$\softmax(v)_i = \frac{\exp(v_i)}{\sum_{j=1}^n \exp(v_j)}$ \\
Gives smoothed version of argmax,
which returns vector with $1$ in position of largest element
and $0$ elsewhere \\
Entries have sum $1$:
may be interpreted as probability that given element of $v$ is the largest \\
Want to generalize to $\soft_k(v)$ for $k \in \{1, \ldots, n\} = [n]$:
probability that $v_j$ is the $k^\text{th}$-largest \\
Desired properties:
\begin{enumerate}
\item
  $\soft_1 = \softmax$
\item
  $\soft_{n+1-k}(v) = \soft_k(-v)$
\item
  The largest entry of $\soft_k(v)$
  is in the same place as the $k^\text{th}$-largest entry of $v$
\item
  $\sum_{j=1}^n\soft_k(v)_j = 1$
  (exactly one element of $v$ is at position $k$)
\item
  $\sum_{k=1}^n\soft_k(v)_j = 1$
  ($v_j$ is in exactly one position)
\item
  $\soft_k$ is smooth
\item
  $\soft_k(v) = \soft_k(u)$ if there is $c \in \mathbb{R}$
  such that for all $i$, $u_i = v_i+c$, \\
  i.e. $\soft_k$ does not change when adding a constant to all elements of the input
\end{enumerate}
Define $S \in \R^{n \times n}$ by $S_{jk} = \soft_k(v)_j$ \\
Properties 4 and 5: $S$ is doubly stochastic \\
Birkhoff's theorem: $S = \sum_\pi P(\pi)M_\pi$ over permutations $\pi$ of $[n]$, \\
$M_\pi$ has $1$ in position $(\pi(i), i)$ and $0$ elsewhere,
$P(\pi) \in [0, 1]$, $\sum_\pi P(\pi)=1$ \\
Interpretation: $P(\pi)=$ probability of
$v_{\pi(k)}$ being $k^\text{th}$-largest for all $k$ \\
Justification: $S_{jk} = \sum_{\pi(k)=j}P(\pi)$:
$v_j$ is $k^\text{th}$-largest iff
a permutation where $v_j$ is $k^\text{th}$-largest is chosen \\
Reframe problem: define a distribution over permutations,
get 4 and 5 for free \\
Equivalently, define procedure to pick a random permutation:
``smooth'' sorting \\

First attempt: property 1 may be rewritten $P(\pi(1)=j) = \softmax(v)_j$ \\
Can ensure by picking $\pi(1) = a$ first,
then permuting the remaining $n-1$ elements recursively:
\begin{itemize}
\item
  Define $\sigma: [n] \setminus \{a\} \leftrightarrow [n-1]$
  by $\sigma(i) = i$ if $i < a$ and $i-1$ if $i > a$:
  map between indices
\item
  Let $v' \in \R^{n-1}$ where $v'_i = v_{\sigma(i)}$
  (i.e. $v'$ is $v$ with $v_a$ removed)
\item
  Find permutation $\pi'$ of $[n-1]$ based on $v'$
\item
  Define $\pi(i) = a$ if $i=1$ and $\sigma^{-1}(\pi'(i-1))$ otherwise
\end{itemize}
Base case $n=0$: identity permutation \\
Since softmax satisfies independence of irrelevant alternatives (IIA),
this is equivalent to:
\begin{itemize}
\item
  Repeatedly pick $i \in [n]$ with probability $\softmax(v)_i$
\item
  If already seen, discard; else, add to list
\item
  Once list has length $n$, $\pi(k) = $ element at $k^\text{th}$ position
\end{itemize}
However, this procedure does not satisfy property 2:
\begin{itemize}
\item
  Let $n=3$, $v_1=2$, $v_2=1$, $v_3=0$
\item
  Then $\soft_3(v)_3 = \soft_1(-v)_3 = \frac{\exp(0)}{\exp(-2)+\exp(-1)+\exp(0)} \approx 0.67$ \\
\item
  Procedure picks $v_3$ last iff either ($v_1$ first, then $v_2$) or ($v_2$ first, then $v_1$)
\item
  Probability $= \frac{\exp(2)}{\exp(2)+\exp(1)+\exp(0)}\frac{\exp(1)}{\exp(1)+\exp(0)}
  +\frac{\exp(1)}{\exp(2)+\exp(1)+\exp(0)}\frac{\exp(2)}{\exp(2)+\exp(0)}
  \approx 0.70$
\end{itemize}
New procedure: should act on $v$ ``symmetrically'' to satisfy property 2 \\
Idea: pick $\pi(1) = a$ and $\pi(n) = b$ at once,
then find middle recursively using $v$ without $v_a$ or $v_b$ \\
(Base cases are now $n=0$ or $1$: still identity) \\
First attempt: pick $a$ by $\softmax(v)$ and $b$ by $\softmax(-v)$ independently \\
Equivalently:
\begin{itemize}
  \item
    Define matrix $E$ by $E_{ab} = \exp(v_a-v_b)$
  \item
    Pick $(a, b)$ with probability proportional to $E_{ab}$,
    i.e. equal to $E_{ab}/$(sum of entries of $E$)
\end{itemize}
$E$ stores non-normalized joint distribution of $(a, b)$ \\
Non-normalized marginal distribution of $a$ given by
$\sum_b E_{ab} = \exp(v_a)\sum_b \exp(-v_b)$ \\
Proportional to $\exp(v_a)$, which is proportional to $\softmax(v)_a$,
so property $1$ satisfied \\
Problem: $a$ could equal $b$ since $E_{aa} = \exp(v_a-v_a) = 1 \neq 0$:
can't have $\pi(1)=\pi(n)$ \\
Need to fix $E$ by making diagonal $0$
without changing marginal distributions (ratios of row/column sums) \\
First attempt:
\begin{itemize}
  \item
    Just zero out diagonal
  \item
    Equivalently, try to pick $a$ and $b$ again with the same probabilities
    until $a \neq b$
  \item
    Sum of each row decreases by $1$,
    but since rows can have different sums,
    their ratio can change
  \item
    Thus marginal distribution of $a$ not preserved (same for $b$)
  \item
    Scaling $E$ to restore sum of entries doesn't help
    since ratios of row/column sums don't change
\end{itemize}
Second attempt:
\begin{itemize}
\item
  Zero out diagonal, add $\frac{1}{n-1}$ to all other entries
\item
  Equivalently, pick $a$ and $b$ as usual,
  but if $a=b$, pick a distinct pair \textit{uniformly} instead
\item
  Row/column sums decrease by $1$ in $1$ spot
  and increase by $\frac{1}{n-1}$ in $n-1$ spots:
  no change
\item
  Thus marginal distributions of $a$ and $b$ preserved: success!
\end{itemize}
Recall this is a single iteration:
overall process finds $\pi(1)$ and $\pi(n)$, then $\pi(2)$ and $\pi(n-1)$, $\ldots$ \\
Use probabilities of picking each $\pi$ to define $\soft_k$ as described by Birkhoff's thm. \\
Satisfies properties 1, 2, 4, 5, 6, and 7 by virtue of its construction \\
However, procedure suggests that closed form of $\soft_k$ must case on whether $k < n/2$ \\
(Also expensive by naive method, but not currently a concern)

% What about property 3? By property 2, suffices to show for $k \leq n/2$ \\

\end{document}
