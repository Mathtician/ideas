\documentclass{beamer}
% \usepackage[margin=1in]{geometry}
\usepackage{amsmath}
\usepackage{amssymb}
\usepackage{graphicx}
\graphicspath{images/}

\begin{document}

\title{CRF Diminishings of the 600-Cell}
\author{Mathtician}
\maketitle

\begin{frame}
  \frametitle{Table of Contents}
  \tableofcontents
\end{frame}

\section{Prior work}

\begin{frame}
  \frametitle{Definitions}
  \begin{itemize}
  \item
    \textbf{CRF}: convex regular-faced
    \begin{itemize}
    \item
      \textbf{Convex}: may be formed as the face lattice of a bounded convex set,
      i.e. all elements are strictly convex
    \item
      \textbf{Face}: 2D element (not of any dimension, as is used elsewhere)
    \end{itemize}
  \item
    \textbf{600-cell (ex)}:
    the convex regular polychoron with 120 vertices
    and 600 tetrahedral cells, 5 to an edge
  \item
    \textbf{Diminishing of ex}:
    convex hull of a subset of the vertices of ex
    which are not all on the same hyperplane
    \begin{itemize}
    \item
      Same hyperplane $\rightarrow$ lower dimension
    \end{itemize}
  \end{itemize}
\end{frame}

\begin{frame}
  \frametitle{CRF Polyhedra}
  \begin{itemize}
  \item
    5 Platonic solids
  \item
    13 Archimedean solids
  \item
    $\infty$ prisms
    \begin{itemize}
    \item
      $n \in [3, \infty)$ sides, $n = 4 \rightarrow$ cube
    \end{itemize}
  \item
    $\infty$ antiprisms
    \begin{itemize}
    \item
      $n \in [3, \infty)$ sides, $n = 3 \rightarrow$ octahedron
    \end{itemize}
  \item
    92 Johnson solids\footnote{Norman Johnson, 1966}\footnote{Victor Zalgaller, 1969}
  \end{itemize}
\end{frame}

\begin{frame}
  \frametitle{Blind Polytopes}
  \begin{itemize}
  \item
    \textbf{Blind polytope}: convex polytope with regular facets
    \begin{itemize}
    \item
      Achtung: short ``i'' sound!
    \end{itemize}
  \item
    Blind polychora have regular (Platonic solid) cells
  \item
    There are 314248357 Blind polychora(!)
  \item
    All but 13 of them are \textbf{special cuts} of ex
  \end{itemize}
\end{frame}
%% \begin{frame}
%%   \frametitle{Uniform Blind Polychora}
%%   \begin{itemize}
%%   \item
%%     6 convex regular polychora
%%     \begin{itemize}
%%     \item
%%       5-cell (pen)
%%     \item
%%       Tesseract (tes)
%%     \item
%%       16-cell (hex)
%%     \item
%%       24-cell (ico)
%%     \item
%%       120-cell (hi)
%%     \item
%%       600-cell (ex)
%%     \end{itemize}
%%   \item 3 other uniform polychora
%%     \begin{itemize}
%%     \item
%%       Rectified 5-cell (rap)
%%     \item
%%       Rectified ex (rox)
%%     \item
%%       Snub 24-cell (sadi)
%%     \end{itemize}
%%   \end{itemize}
%% \end{frame}
%% \begin{frame}
%%   \frametitle{Non-uniform Blind Polychora}
%%   \begin{itemize}
%%   \item
%%     Tetrahedral bipyramid
%%   \item
%%     Octahedral pyramid
%%   \item
%%     Icosahedral pyramid
%%   \item
%%     Icosahedral bipyramid
%%   \item
%%     Augmented rectified 5-cell
%%     \begin{itemize}
%%     \item
%%       Octahedral cell is augmented
%%     \end{itemize}
%%   \item
%%     314248343 other \textbf{special cuts of ex}
%%   \end{itemize}
%% \end{frame}

\begin{frame}
  \frametitle{Special Cuts}
  \begin{itemize}
  \item
    A special cut is ex, diminished by a (nonempty) independent set of vertices
  \item
    Diminishing a vertex replaces 20 tetrahedra with an icosahedron
  \item
    Diminishing two adjacent vertices would mean the new cells are no longer icosahedral: remember this for later
  \item
    There are 314248344 nonempty independent sets of ex up to symmetry,
    so there are the same number of special cuts
    \begin{itemize}
    \item
      One independent set, the vertices of an inscribed 24-cell,
      creates a uniform special cut, the snub 24-cell
    \end{itemize}
  \end{itemize}
\end{frame}

\begin{frame}
  \frametitle{CRF Polychora}
  \begin{itemize}
  \item
    The other main way to generalize Johnson solids (or the family they complete)
    to 4D
  \item
    Too many to count! At least $10^\text{hundreds}$, maybe $10^\text{thousands}$
  \item
    This class contains even more ex diminishings than the special cuts
  \item
    But how many?
  \end{itemize}
\end{frame}

\section{Classifying Diminishings}

\begin{frame}
  \frametitle{Edge Length}
  \begin{itemize}
  \item
    CRFs have all edges the same length
  \item
    8 different distances between verts of ex: possible edge lengths
  \item
    Longest edges have no CRFs: too cramped
  \item
    Intermediate edges have some CRFs
    \begin{itemize}
    \item
      Diminishings of inscribed polychora e.g. 24-cell
    \end{itemize}
  \item
    Vast majority of CRFs have same edge length as ex
    \begin{itemize}
    \item
      Focus of this presentation
    \end{itemize}
    %%   where $\varphi := \frac{1+\sqrt{5}}{2}$:
    %%   \begin{itemize}
    %%   \item
    %%     $\sqrt{1+0\varphi} = 1$
    %%   \item
    %%     $\sqrt{1+1\varphi} = \varphi \approx 1.618$
    %%   \item
    %%     $\sqrt{2+1\varphi} \approx 1.902$
    %%   \item
    %%     $\sqrt{2+2\varphi} \approx 2.288$
    %%   \item
    %%     $\sqrt{2+3\varphi} = 1+\varphi \approx 2.618$
    %%   \item
    %%     $\sqrt{3+3\varphi} \approx 2.803$
    %%   \item
    %%     $\sqrt{3+4\varphi} \approx 3.078$
    %%   \item
    %%     $\sqrt{4+4\varphi} = 2\varphi \approx 3.236$
    %%   \end{itemize}
    %% \item
    %%   Polychoron with largest (unique) $\frac{\text{edge}}{\text{radius}}$ is the 5-cell $\left(\frac{\sqrt{10}}{2}\right)$
    %% \item
    %%   With given radius of $\varphi$, edge $\leq 2.559$
    %% \item
    %%   For remaining edge lengths, can find possible angles/faces
  \end{itemize}
\end{frame}
\begin{frame}
  \frametitle{Faces and Cells}
  \begin{itemize}
  \item
    Faces: triangles, pentagons, decagons
  \item
    Cells, in order of increasing circumradius:
    \begin{itemize}
    \item
      Tetrahedron
    \item
      Icosahedron, its 4 ``shallow'' diminishings, and the pentagonal pyramid (its ``deep'' diminishing)
    \item
      Dodecahedron
    \item
      Icosidodecahedron and its ``half'', the pentagonal rotunda
    \end{itemize}
  \item
    Higher circumradius $\rightarrow$ deeper cut, up to icosidodecahedron at the equator
  \item
    Can also make cuts deeper than the equator
  \item
    Most CRF diminishings of ex consist of only the shallowest cuts,
    which diminish 1 vertex each
    \begin{itemize}
    \item
      Intuition: deeper cuts remove more vertices,
      so there are fewer options for where to place the remaining shallow cuts
    \end{itemize}
  \end{itemize}
\end{frame}
\begin{frame}
  \frametitle{Shallow Cuts}
  \begin{itemize}
  \item
    Diminishing 1 vertex creates an icosahedron, as in the special cuts
  \item
    Diminishing 2 adjecent vertices creates 2 diminished icosahedra,
    joined at the pentagon
  \item
    Diminishing a triangle creates icosahedra with 2 adjacent vertices removed,
    which have trapezoidal faces: not allowed
  \item
    Otherwise, any triangle-free set may be diminished to produce a CRF
    \begin{itemize}
    \item
      Other diminishings of the icosahedron may be created
      by diminishing a vertex and an independent set of its neighbors
    \end{itemize}
  \end{itemize}
\end{frame}
\end{document}
