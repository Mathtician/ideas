\documentclass{article}
\usepackage{amsmath}
\usepackage{graphicx}
\usepackage[margin=1in]{geometry}
\usepackage{hyperref}
\usepackage{caption}
\usepackage{float}
\graphicspath{{images/}}
\hypersetup{
  colorlinks=true,
  urlcolor=blue,
}
\begin{document}

\title{Minimal Polynomials of Real Parts of Roots of Unity}
\author{Aresh Pourkavoos}
\maketitle

\newcommand{\cis}{\text{cis}}

What are the coordinates of the vertices
of a regular $n$-gon of radius 1?
The simple answer comes from trigonometry:
the vertices are points on the unit circle,
separated by angles of $\frac{\tau}{n}$
(where $\tau$ is the number of radians in a full turn),
so their coordinates are
\[\left(\cos\left(\frac{m}{n}\tau\right), \sin\left(\frac{m}{n}\tau\right)\right)\]
for all $0 \leq m < n$.
However, this formula makes symbolic manipulation unwieldy
when trying to reduce expressions involving these terms to their simplest form.
For example, when $n=5$ and $m=1$,
the x-coordinate of the point is
$\cos\left(\frac{1}{5}\tau\right) = \frac{1}{2}(1-\varphi)$,
where $\varphi = \frac{1+\sqrt{5}}{2}$ is the golden ratio,
the positive solution to $\varphi^2 = \varphi+1$.
Writing the coordinate in terms of $\varphi$ is easier than the trigonometric way,
since products involving it can easily be simplified
by replacing $\varphi^2$ with $\varphi+1$.

\begin{align*}
  r &= 2\cos\left(\frac{m}{n}\tau\right) \\
  p(r) &= 0 \\ 
  z &= \cis\left(\frac{m}{n}\tau\right) \\
  z^{-1} &= \cis\left(-\frac{m}{n}\tau\right) \\
  r &= z+z^{-1} \\
  q(z) &= 0 \\
\end{align*}

\begin{align*}
  n &= 7 \\
  0 &= z^3+z^2+z+1+z^{-1}+z^{-2}+z^{-3} \\
  &= a+br+cr^2+dr^3 \\
  &= a+b(z+z^{-1})+c(z+z^{-1})^2+d(z+z^{-1})^3 \\
  &= a+b(z+z^{-1})+c(z^2+2+z^{-2})+d(z^3+3z+3z^{-1}+z^{-3}) \\
  &= dz^3+cz^2+(b+3d)z+(a+2c)+(b+3d)z^{-1}+cz^{-2}+dz^{-3}
\end{align*}

\begin{align*}
  \begin{bmatrix}
    1 & 0 & 2 & 0 \\
    0 & 1 & 0 & 3 \\
    0 & 0 & 1 & 0 \\
    0 & 0 & 0 & 1 \\
  \end{bmatrix}
  \begin{bmatrix}
    a \\ b \\ c \\ d
  \end{bmatrix}
  &=
  \begin{bmatrix}
    1 \\ 1 \\ 1 \\ 1
  \end{bmatrix}
  \\
  \begin{bmatrix}
    1 & 2 \\
    0 & 1 \\
  \end{bmatrix}
  \begin{bmatrix}
    a \\ c
  \end{bmatrix}
  &=
  \begin{bmatrix}
    1 \\ 1
  \end{bmatrix}
  \\
  \begin{bmatrix}
    1 & 3 \\
    0 & 1 \\
  \end{bmatrix}
  \begin{bmatrix}
    b \\ d
  \end{bmatrix}
  &=
  \begin{bmatrix}
    1 \\ 1
  \end{bmatrix}
  \\
  \begin{bmatrix}
    a \\ c
  \end{bmatrix}
  &=
  \begin{bmatrix}
    -1 \\ 1
  \end{bmatrix}
  \\
  \begin{bmatrix}
    b \\ d
  \end{bmatrix}
  &=
  \begin{bmatrix}
    -2 \\ 1
  \end{bmatrix}
\end{align*}

\begin{align*}
  r^3+r^2-2r-1 &= 0 \\
\end{align*}

%% This is a pretty good idea.
%% \begin{figure}[H]
%%   \centering
%%   \includegraphics[width=0.5\linewidth]{/home/a/d/images/z.jpeg}
%%   \caption*{What is this thing? \textit{Generated by \href{https://artbreeder.com}{Artbreeder}}}
%% \end{figure}
%% Use the writeidea command to edit.

\begin{tabular}{|c|c|}
  \hline
  1 & $r-2$ \\ \hline
  2 & $r+2$ \\ \hline
  3 & $r+1$ \\ \hline
  4 & $r$ \\ \hline
  5 & $r^2+r-1$ \\ \hline
  6 & $r-1$ \\ \hline
  7 & $r^3+r^2-2r-1$ \\ \hline
  8 & $r^2-2$ \\ \hline
  9 & $r^2-3r+1$ \\ \hline
  10 & $r^2 \pm r-1$ \\ \hline
\end{tabular}
  
\end{document}
