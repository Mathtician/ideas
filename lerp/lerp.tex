\documentclass{article}
\usepackage[margin=1in]{geometry}
\usepackage{amsmath}
\usepackage{amssymb}
\usepackage{graphicx}
\graphicspath{images/}

\begin{document}

\title{Monotone Cubic Interpolation}
\author{Aresh Pourkavoos}
\maketitle

Given a set of data points $(x_1, y_1), \ldots, (x_n, y_n)$
in which both coordinates strictly increase moving down the list
(i.e. $x_1 < \ldots < x_n$ and $y_1 < \ldots < y_n$),
how can we define a function $f(x) = y$ which passes through all points
and which has positive slope everywhere?
Drawing line segments between adjacent points almost works,
but the derivative is generally undefined at each point,
since the slopes of the segments on either side may be different.
The apparent solution would be to use cubic interpolation instead,
which allows for the derivative at each point to be set,
and cubic curves between each pair of adjacent points may be found.

(Hermite polynomials)

This raises the question of what the derivatives should be.

A common approach to cubic interpolation
is, for every point,
to connect the points on its left and right
and use the slope of the resulting segment.
If the point is on the far left or right,
the slope of the segment between it and its adjacent point is used.
However, this approach does not guarantee monotonicity for monotone inputs.

(Example)

On the other hand, the derivative could be 0 at every data point,
which guarantees that the function is increasing (and in fact, strictly increasing).
However, this solution is not ideal because
the derivative should be positive everywhere,
which it is everywhere except the points themselves.
It is possible to choose a small positive slope $\epsilon$ to place at each point
(assuming finitely many points)
which avoids the non-increasing segments seen previously.
But this does not produce a straight line if the points are collinear,
which an ideal ``interpolation'' solution would.

Instead of trying to find an explicit formula for the derivatives,
we can define the task of finding them as an optimization problem.
In other words, given a potential assignment of derivatives to each point,
we can define a way to evaluate how ``well'' it interpolates between them,
and select the derivatives which perform the best.
A common approach for these types of problems is to define the ``energy'' for a given curve,
which usually measures deviation from a straight line in some way.
The name ``energy'' comes from the fact that a rod of elastic material ``wants'' to remain straight,
and bending it stores mechanical potential energy within it.

\end{document}
