\documentclass{article}
\usepackage[margin=1in]{geometry}
\usepackage{amsmath}
\usepackage{amssymb}
\usepackage{graphicx}
\graphicspath{images/}

\begin{document}

\title{Cubic Interpolation, Monotone and Otherwise}
\author{Aresh Pourkavoos}
\maketitle

Given a set of data points $(x_1, y_1), \ldots, (x_n, y_n)$
in which both coordinates strictly increase moving down the list
(i.e. $x_1 < \ldots < x_n$ and $y_1 < \ldots < y_n$),
how can we define a function which passes through all points
and which has positive slope everywhere?
In other words, we want to find $f : [x_1, x_n] \rightarrow \mathbb{R}$
such that $f(x_i) = y_i$ for all $i \in \{1, \ldots, n\}$
and $f'(x) > 0$ for all $x \in [x_1, x_n]$.

Drawing line segments between adjacent points almost works,
but the derivative is generally undefined at each point,
since the slopes of the segments on either side may be different.
The apparent solution would be to use cubic interpolation instead,
which allows for the derivative at each point to be set,
and cubic curves between each pair of adjacent points may be found.

(Hermite polynomials)

This raises the question of what the derivatives should be.

A common approach to cubic interpolation
is, for every point,
to connect the points on its left and right
and use the slope of the resulting segment.
If the point is on the far left or right,
the slope of the segment between it and its adjacent point is used.
However, this approach does not guarantee monotonicity for monotone inputs.

(Example)

On the other hand, the derivative could be 0 at every data point,
which guarantees that the function is increasing.
However, this solution is not ideal because
the derivative should be positive everywhere.
It is possible to choose a very small positive slope at each point,
which still avoids the decreasing segments seen previously.
But this does not produce a straight line if the points are collinear,
which an ideal solution would.

Instead of checking different formulas for the function,
we can think about finding it as an optimization problem.
In other words, given a function that passes through the points,
we can define a way to evaluate how ``well'' it interpolates between them,
and select the function which perform the best.
A common approach for these types of problems is to define the ``energy'' for a given curve,
which usually measures deviation from a straight line.
The name comes from the fact that
an elastic rod stores mechanical potential energy when bent
and tries to minimize its energy by straightening when released.

One of the ways to define a straight line is that its derivative is constant,
i.e. its second derivative is 0 everywhere.
Thus, we might try to keep the second derivative of our function as close to 0 as possible,
to prevent it from changing slope too quickly.
A natural way to measure distance from 0 is to square the second derivative,
disincentivizing both negative and positive values.
Since every point along the graph of the function should be taken into account,
we can integrate the squared second derivative over the entire interval:

\[J(f) = \int_{x_1}^{x_n}f''(x)^2 dx.\]

This is a problem in variational calculus,
which is calculus on functionals,
i.e. functions whose inputs are themselves functions.
$J$ defined above is a functional,
since it accepts a function $f$ and returns its energy.
Here, we will restrict $f$ to be $C^1$, or continuously differentiable.
This means that $f$ must have a derivative on the open interval $(x_1, x_n)$
(as well as one-sided derivatives at the endpoints $x_1$ and $x_n$),
and $f'$ must itself be a continuous function.

The set of such functions that passes through an arbitrary set of points
$(x_1, y_1), \ldots, (x_n, y_n)$ is difficult to parameterize.
Instead, we take advantage of the definition of $C^1$
to break the problem into more manageable parts.
Suppose we already know $f'(x_1), \ldots, f'(x_n)$,
i.e. the derivative of the function at each data point.
Call these values $c_1, \ldots, c_n$.
Then $f$ may be split into pieces $f_1, \ldots, f_{n-1}$,
where the domain of $f_i$ is the interval between adjacent data points $[x_i, x_{i+1}]$
for all $i \in \{1, \ldots, n-1\}$.
It follows that each $f_i$ is also $C^1$ and
\begin{align*}
  f_i(x_i) &= y_i, \\
  f_i(x_{i+1}) &= y_{i+1}, \\
  f_i'(x_i) &= c_i, \\
  f_i'(x_{i+1}) &= c_{i+1},
\end{align*}
where the endpoints have one-sided derivatives.
The important thing, though, is that once the derivatives are set,
the different possible curves of each piece $f_i$
may be swapped out independently of one another.
This is because the derivatives of adjacent pieces will always meet up at the endpoints,
preserving the continuity of $f'$ across the entire domain $[x_1, x_n]$.
This means that whatever the derivatives $c_i$ happen to be,
$f$ must necessarily be made of segments which are individually optimal for those $c_i$.
If $f$ had a suboptimal segment in it,
then that segment could be replaced with one with a lower value of $J$,
bringing down the total value for $f$.
Thus $f$ can be optimized as follows:
\begin{itemize}
\item
  Let the derivatives $c_1, \ldots, c_n$ be arbitrary.
\item
  For each interval $[x_i, x_{i+1}]$, find the optimal $C^1$ function $f_i$
  with endpoint values $y_i$ and $y_{i+1}$
  and endpoint derivatives $c_i$ and $c_{i+1}$.
\item
  Assemble $f$ using these $f_i$ and calculate $J(f)$ in terms of the various $c_i$.
\item
  Since $J$ now depends only on a handful of values $c_i$,
  it may be minimized using multivariable calculus.
\end{itemize}

The first two steps, like the original problem, are still part of variational calculus,
but they are now more manageable.
For $f_i$, the relevant variables are
$x_i$, $x_{i+1}$, $y_i$, $y_{i+1}$, $c_i$, and $c_{i+1}$,
and the goal is to minimize
\[\int_{x_i}^{x_i+1}f_i''(x)^2 dx.\]

There are a few techniques that could be used,

\newcommand{\dx}{\, \mathrm{d}x}
\newcommand{\du}{\, \mathrm{d}u}
\newcommand{\ddx}[1]{\frac{\mathrm{d}#1}{\mathrm{d}x}}

\newpage
\begin{align*}
  %% \frac{\partial J_i}{\partial c_i}
  %% &= \frac{\partial}{\partial c_i} \int_{x_i}^{x_{i+1}} f_i''(x)^2 \mathrm{d}x \\
  %% &= \int_{x_i}^{x_{i+1}} \frac{\partial}{\partial c_i} f_i''(x)^2 \mathrm{d}x \\
  %% \frac{\partial}{\partial c_i} f_i''(x)^2
  %% &= 2f_i''(x) \frac{\partial}{\partial c_i} f_i''(x) \\
  %% f_i''(x)
  %% &= \frac{\mathrm{d}}{\mathrm{d}x}f_i'(x) \\
  %% f_i'(x)
  %% &= \frac{\mathrm{d}}{\mathrm{d}x}f_i(x) \\
  %% f_i(x)
  %% &= g_i\left(\frac{x-x_i}{w_i}\right) \\
  %% &= g_i(u) \\
  h_{00}(u)
  &= 2u^3-3u^2+1 \\
  h_{01}(u)
  &= -2u^3+3u^2 \\
  h_{10}(u)
  &= u^3-2u^2+u \\
  h_{11}(u)
  &= u^3-u^2
\end{align*}
\begin{align*}
  g_1(u)
  &= y_1h_{00}(u)+y_2h_{01}(u)+w_1c_1h_{10}(u)+w_1c_2h_{11}(u) \\
  &= y_1(2u^3-3u^2+1)+y_2(-2u^3+3u^2)+w_1c_1(u^3-2u^2+u)+w_1c_2(u^3-u^2) \\
  &= (2y_1-2y_2+w_1c_1+w_1c_2)u^3+(-3y_1+3y_2-2w_1c_1-w_1c_2)u^2+w_1c_1u+y_1 \\
  &= (-2(y_2-y_1)+w_1c_1+w_1c_2)u^3+(3(y_2+3y_1)-2w_1c_1-w_1c_2)u^2+w_1c_1u+y_1 \\
  &= (-2h_1+w_1c_1+w_1c_2)u^3+(3h_1-2w_1c_1-w_1c_2)u^2+w_1c_1u+y_1 \\
  &= pu^3+qu^2+w_1c_1u+y_1 \\
  \frac{\partial p}{\partial c_1}
  &= w_1 \\
  \frac{\partial q}{\partial c_1}
  &= -2w_1 \\
  g_1'(u)
  &= 3pu^2+2qu+w_1 \\
  g_1''(u)
  &= 6pu+2q
\end{align*}
\begin{align*}
  f_1(x)
  &= g_1\left(\frac{x-x_1}{w_1}\right) \\
  &= g_1(u) \\
  \ddx{u}
  &= \frac{1}{w_1} \\
  w_1\mathrm{d}u
  &= \mathrm{d}x \\
  f_1'(x)
  &= \ddx{} \left(g_1(u)\right) \\
  &= \ddx{u} g_1'(u) \\
  &= \frac{1}{w_1}g_1'(u) \\
  f_1''(x)
  &= \ddx{} \left(\frac{1}{w_1}g_1'(u)\right) \\
  &= \frac{1}{w_1} \ddx{} (g_1'(u)) \\
  &= \frac{1}{w_1} \ddx{u} g_1''(u) \\
  &= \frac{1}{w_1} \frac{1}{w_1} g_1''(u) \\
  &= \frac{1}{w_1^2} g_1''(u) \\
  &= \frac{1}{w_1^2}(6pu+2q) \\
  &= \frac{2}{w_1^2}(3pu+q)
\end{align*}
\begin{align*}
  J_1
  &= \int_{x=x_1}^{x_2}f_1''(x)^2\dx \\
  \frac{\partial J_1}{\partial c_1}
  &= \frac{\partial}{\partial c_1} \int_{x=x_1}^{x_2} f_1''(x)^2 \dx \\
  &= \int_{x=x_1}^{x_2} \frac{\partial}{\partial c_1} f_1''(x)^2 \dx \\
  &= \int_{u=0}^{1} \frac{\partial}{\partial c_1} \left(\frac{2}{w_1^2}(3pu+q)\right)^2 w_1 \du \\
  &= \frac{4}{w_1^3} \int_{u=0}^{1} \frac{\partial}{\partial c_1} (3pu+q)^2 \du \\
  &= \frac{4}{w_1^3} \int_{u=0}^{1} 2(3pu+q) \frac{\partial}{\partial c_1} (3pu+q) \du \\
  &= \frac{4}{w_1^3} \int_{u=0}^{1} 2(3pu+q) \left(3\frac{\partial p}{\partial c_1}u+\frac{\partial q}{\partial c_1}\right) \du \\
  &= \frac{4}{w_1^3} \int_{u=0}^{1} 2(3pu+q)(3w_1u-2w_1) \du \\
  &= \frac{4}{w_1^2} \int_{u=0}^{1} 2(3pu+q)(3u-2) \du \\
  &= \frac{4}{w_1^2} \int_{u=0}^{1} 18pu^2+(6q-12p)u-4q \du \\
  &= \frac{4}{w_1^2} (6pu^3+(3q-6p)u^2-4qu)\Bigr|_0^1 \\
  &= \frac{4}{w_1^2} ((6p1^3+(3q-6p)1^2-4q1)-(6p0^3+(3q-6p)0^2-4q0)) \\
  &= \frac{4}{w_1^2} (6p+(3q-6p)-4q) \\
  &= \frac{4}{w_1^2} (-q) \\
  &= \frac{4}{w_1^2} (-(3h_1-2w_1c_1-w_1c_2)) \\
  &= \frac{4}{w_1^2} (-3h_1+2w_1c_1+w_1c_2) \\
  &= 4\left(-\frac{3h_1}{w_1^2}+\frac{2}{w_1}c_1+\frac{1}{w_1}c_2\right)
\end{align*}
\begin{align*}
  \frac{\partial J_i}{\partial c_i}
  &= 4\left(-\frac{3h_i}{w_i^2}+\frac{2}{w_i}c_i+\frac{1}{w_i}c_{i+1}\right) \\
  \frac{\partial J_i}{\partial c_{i+1}}
  &= 4\left(-\frac{3h_i}{w_i^2}+\frac{1}{w_i}c_i+\frac{2}{w_i}c_{i+1}\right) \\
  \frac{\partial J}{\partial c_1}
  &= \frac{\partial J_1}{\partial c_1} \\
  &= 4\left(-\frac{3h_1}{w_1^2}+\frac{2}{w_1}c_1+\frac{1}{w_1}c_2\right) \\
  \frac{\partial J}{\partial c_n}
  &= \frac{\partial J_{n-1}}{\partial c_n} \\
  &= 4\left(-\frac{3h_{n-1}}{w_{n-1}^2}+\frac{1}{w_{n-1}}c_{n-1}+\frac{2}{w_{n-1}}c_n\right) \\
  \frac{\partial J}{\partial c_i}
  &= \frac{\partial J_{i-1}}{\partial c_i}
  + \frac{\partial J_i}{\partial c_i} \\
  &= 4\left(-\frac{3h_{i-1}}{w_{i-1}^2}+\frac{1}{w_{i-1}}c_{i-1}+\frac{2}{w_{i-1}}c_i\right)
  + 4\left(-\frac{3h_i}{w_i^2}+\frac{2}{w_i}c_i+\frac{1}{w_i}c_{i+1}\right) \\
  &= 4\left(-\frac{3h_{i-1}}{w_{i-1}^2}-\frac{3h_i}{w_i^2}
  +\frac{1}{w_{i-1}}c_{i-1}
  +\left(\frac{2}{w_{i-1}}+\frac{2}{w_i}\right)c_i
  +\frac{1}{w_i}c_{i+1}\right)
\end{align*}
\begin{align*}
  4\left(-\frac{3h_1}{w_1^2}+\frac{2}{w_1}c_1+\frac{1}{w_1}c_2\right)
  &= 0 \\
  -\frac{3h_1}{w_1^2}+\frac{2}{w_1}c_1+\frac{1}{w_1}c_2
  &= 0 \\
  \frac{2}{w_1}c_1+\frac{1}{w_1}c_2
  &= \frac{3h_1}{w_1^2} \\
  4\left(-\frac{3h_{i-1}}{w_{i-1}^2}-\frac{3h_i}{w_i^2}
  +\frac{1}{w_{i-1}}c_{i-1}
  +\left(\frac{2}{w_{i-1}}+\frac{2}{w_i}\right)c_i
  +\frac{1}{w_i}c_{i+1}\right)
  &= 0 \\
  -\frac{3h_{i-1}}{w_{i-1}^2}-\frac{3h_i}{w_i^2}
  +\frac{1}{w_{i-1}}c_{i-1}
  +\left(\frac{2}{w_{i-1}}+\frac{2}{w_i}\right)c_i
  +\frac{1}{w_i}c_{i+1}
  &= 0 \\
  \frac{1}{w_{i-1}}c_{i-1}
  +\left(\frac{2}{w_{i-1}}+\frac{2}{w_i}\right)c_i
  +\frac{1}{w_i}c_{i+1}
  &= \frac{3h_{i-1}}{w_{i-1}^2}+\frac{3h_i}{w_i^2} \\
  4\left(-\frac{3h_{n-1}}{w_{n-1}^2}+\frac{1}{w_{n-1}}c_{n-1}+\frac{2}{w_{n-1}}c_n\right)
  &= 0 \\
  -\frac{3h_{n-1}}{w_{n-1}^2}+\frac{1}{w_{n-1}}c_{n-1}+\frac{2}{w_{n-1}}c_n
  &= 0 \\
  \frac{1}{w_{n-1}}c_{n-1}+\frac{2}{w_{n-1}}c_n
  &= \frac{3h_{n-1}}{w_{n-1}^2}
\end{align*}
\begin{align*}
  \begin{bmatrix}
    \frac{2}{w_1} & \frac{1}{w_1} & 0 \\
    \frac{1}{w_1} & \frac{2}{w_1}+\frac{2}{w_2} & \frac{1}{w_2} \\
    0 & \frac{1}{w_2} & \frac{2}{w_2} \\
  \end{bmatrix}
  \begin{bmatrix}
    c_1 \\ c_2 \\ c_3
  \end{bmatrix}
  &=
  \begin{bmatrix}
    \frac{3h_1}{w_1^2} \\
    \frac{3h_1}{w_1^2}+\frac{3h_2}{w_2^2} \\
    \frac{3h_2}{w_2^2}
  \end{bmatrix} \\
  \begin{bmatrix}
    \frac{2}{w_1} & \frac{1}{w_1} & 0 & 0 \\
    \frac{1}{w_1} & \frac{2}{w_1}+\frac{2}{w_2} & \frac{1}{w_2} & 0 \\
    0 & \frac{1}{w_2} & \frac{2}{w_2}+\frac{2}{w_3} & \frac{1}{w_3} \\
    0 & 0 & \frac{1}{w_3} & \frac{2}{w_3}
  \end{bmatrix}
  \begin{bmatrix}
    c_1 \\ c_2 \\ c_3 \\ c_4
  \end{bmatrix}
  &=
  \begin{bmatrix}
    \frac{3h_1}{w_1^2} \\
    \frac{3h_1}{w_1^2}+\frac{3h_2}{w_2^2} \\
    \frac{3h_2}{w_2^2}+\frac{3h_3}{w_3^2} \\
    \frac{3h_3}{w_3^2}
  \end{bmatrix}
\end{align*}

%% \begin{align*}
%%   K &= \sum_{i=1}^{n-1}\frac{w_i^2}{h_i}J_i \\
%%   \frac{\partial K}{\partial c_1}
%%   &= \frac{w_1^2}{h_1}\frac{\partial J_1}{\partial c_1} \\
%%   &= \frac{w_1^2}{h_1}4\left(-\frac{3h_1}{w_1^2}+\frac{2}{w_1}c_1+\frac{1}{w_1}c_2\right) \\
%%   &= 4\left(-3+2\frac{w_1}{h_1}c_1+\frac{w_1}{h_1}c_2\right) \\
%%   \frac{\partial K}{\partial c_n}
%%   &= \frac{w_{n-1}^2}{h_{n-1}}\frac{\partial J_{n-1}}{\partial c_n} \\
%%   &= \frac{w_{n-1}^2}{h_{n-1}}4\left(-\frac{3h_{n-1}}{w_{n-1}^2}+\frac{1}{w_{n-1}}c_{n-1}+\frac{2}{w_{n-1}}c_n\right) \\
%%   &= 4\left(-3+\frac{w_{n-1}}{h_{n-1}}c_{n-1}+2\frac{w_{n-1}}{h_{n-1}}c_n\right) \\
%%   \frac{\partial K}{\partial c_i}
%%   &= \frac{w_{i-1}^2}{h_{i-1}}\frac{\partial J_{i-1}}{\partial c_i}
%%   + \frac{w_i^2}{h_i}\frac{\partial J_i}{\partial c_i} \\
%%   &= \frac{w_{i-1}^2}{h_{i-1}}4\left(-\frac{3h_{i-1}}{w_{i-1}^2}+\frac{1}{w_{i-1}}c_{i-1}+\frac{2}{w_{i-1}}c_i\right)
%%   + \frac{w_i^2}{h_i}4\left(-\frac{3h_i}{w_i^2}+\frac{2}{w_i}c_i+\frac{1}{w_i}c_{i+1}\right) \\
%%   &= 4\left(-6
%%   +\frac{w_{i-1}}{h_{i-1}}c_{i-1}
%%   +\left(2\frac{w_{i-1}}{h_{i-1}}+2\frac{w_i}{h_i}\right)c_i
%%   +\frac{w_i}{h_i}c_{i+1}\right)
%% \end{align*}

\begin{align*}
  K &= \sum_{i=1}^{n-1}w_i^uh_i^vJ_i \\
  \frac{\partial K}{\partial c_1}
  &= w_1^uh_1^v\frac{\partial J_1}{\partial c_1} \\
  &= w_1^uh_1^v4\left(-\frac{3h_1}{w_1^2}+\frac{2}{w_1}c_1+\frac{1}{w_1}c_2\right) \\
  &= 4\left(-3w_1^{u-2}h_1^{v+1}+2w_1^{u-1}h_1^vc_1+w_1^{u-1}h_1^vc_2\right) \\
  \frac{\partial K}{\partial c_n}
  &= w_{n-1}^uh_{n-1}^v\frac{\partial J_{n-1}}{\partial c_n} \\
  &= w_{n-1}^uh_{n-1}^v4\left(-\frac{3h_{n-1}}{w_{n-1}^2}+\frac{1}{w_{n-1}}c_{n-1}+\frac{2}{w_{n-1}}c_n\right) \\
  &= 4\left(-3w_{n-1}^{u-2}h_{n-1}^{v+1}+w_{n-1}^{u-1}h_{n-1}^vc_{n-1}+w_{n-1}^{u-1}h_{n-1}^vc_n\right) \\
  \frac{\partial K}{\partial c_i}
  &= w_{i-1}^uh_{i-1}^v\frac{\partial J_{i-1}}{\partial c_i}
  + w_i^uh_i^v\frac{\partial J_i}{\partial c_i} \\
  &= w_{i-1}^uh_{i-1}^v4\left(-\frac{3h_{i-1}}{w_{i-1}^2}+\frac{1}{w_{i-1}}c_{i-1}+\frac{2}{w_{i-1}}c_i\right)
  + w_i^uh_i^v4\left(-\frac{3h_i}{w_i^2}+\frac{2}{w_i}c_i+\frac{1}{w_i}c_{i+1}\right) \\
  &= 4\left(
  -3w_{i-1}^{u-2}h_{i-1}^{v+1}-3w_i^{u-2}h_i^{v+1}+\ldots
  \right)
\end{align*}
\end{document}
