\documentclass{article}
\usepackage[margin=1in]{geometry}
\usepackage{amsmath}
\usepackage{amssymb}
\usepackage{graphicx}
\graphicspath{images/}

\begin{document}

\title{Monotone Cubic Interpolation}
\author{Aresh Pourkavoos}
\maketitle

Given a set of data points $(x_1, y_1), \ldots, (x_n, y_n)$
in which both coordinates strictly increase moving down the list
(i.e. $x_1 < \ldots < x_n$ and $y_1 < \ldots < y_n$),
how can we define a function $f(x) = y$ which passes through all points
and which has positive slope everywhere?
Drawing line segments between adjacent points almost works,
but the derivative is generally undefined at each point,
since the slopes of the segments on either side may be different.
The apparent solution would be to use cubic interpolation instead,
which allows for the derivative at each point to be set,
and cubic curves between each pair of adjacent points may be found.
This raises the question of what the derivatives shoud be.

\end{document}
