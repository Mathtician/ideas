\documentclass{article}
\usepackage[margin=1in]{geometry}
\usepackage{amsmath}
\usepackage{amssymb}
\usepackage{graphicx}
\graphicspath{images/}

\begin{document}

\title{Monotone Cubic Interpolation}
\author{Aresh Pourkavoos}
\maketitle

Given a set of data points $(x_1, y_1), \ldots, (x_n, y_n)$
in which both coordinates strictly increase moving down the list
(i.e. $x_1 < \ldots < x_n$ and $y_1 < \ldots < y_n$),
how can we define a function which passes through all points
and which has positive slope everywhere?
In other words, we want to find $f : [x_1, x_n] \rightarrow \mathbb{R}$
such that $f(x_i) = y_i$ for all $i \in \{1, \ldots, n\}$
and $f'(x) > 0$ for all $x \in [x_1, x_n]$.

Drawing line segments between adjacent points almost works,
but the derivative is generally undefined at each point,
since the slopes of the segments on either side may be different.
The apparent solution would be to use cubic interpolation instead,
which allows for the derivative at each point to be set,
and cubic curves between each pair of adjacent points may be found.

(Hermite polynomials)

This raises the question of what the derivatives should be.

A common approach to cubic interpolation
is, for every point,
to connect the points on its left and right
and use the slope of the resulting segment.
If the point is on the far left or right,
the slope of the segment between it and its adjacent point is used.
However, this approach does not guarantee monotonicity for monotone inputs.

(Example)

On the other hand, the derivative could be 0 at every data point,
which guarantees that the function is increasing.
However, this solution is not ideal because
the derivative should be positive everywhere.
It is possible to choose a very small positive slope at each point,
which still avoids the decreasing segments seen previously.
But this does not produce a straight line if the points are collinear,
which an ideal solution would.

Instead of checking different formulas for the function,
we can think about finding it as an optimization problem.
In other words, given a function that passes through the points,
we can define a way to evaluate how ``well'' it interpolates between them,
and select the function which perform the best.
A common approach for these types of problems is to define the ``energy'' for a given curve,
which usually measures deviation from a straight line.
The name comes from the fact that
an elastic rod stores mechanical potential energy when bent
and tries to minimize its energy by straightening when released.

One of the ways to define a straight line is that its derivative is constant,
i.e. its second derivative is 0 everywhere.
Thus, we might try to keep the second derivative of our function as close to 0 as possible,
to prevent it from changing slope too quickly.
A natural way to measure distance from 0 is to square the second derivative,
disincentivizing both negative and positive values.
Since every point along the graph of the function should be taken into account,
we can integrate the squared second derivative over the entire interval:

\[J = \int_{x_1}^{x_n}f''(x)^2 dx.\]

\end{document}
